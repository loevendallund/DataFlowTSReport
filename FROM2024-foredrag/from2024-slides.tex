\documentclass{beamer}
\usefonttheme{serif}
% math and typesetting
%\usepackage{mathtools}
\usepackage{amsmath,amssymb}
\usepackage{balance}

\usepackage[capitalise]{cleveref}

% tables, figures and boxes
\usepackage{float}
\usepackage{listings}
\usepackage{placeins}
\usepackage{subcaption}
\usepackage{wrapfig}
\usepackage{booktabs}
\usepackage{tabularx}
\usepackage[most]{tcolorbox}

\usepackage{soul,xcolor}

\usepackage{filecontents}
\usepackage{stmaryrd}

%etc
\usepackage{subfiles}

\usepackage{algpseudocode}
\usepackage{algorithm}
\usepackage[acronym]{glossaries}
\usepackage[T1]{fontenc}
\usepackage{semantic}

\usepackage{xspace}

%%% Local Variables:
%%% mode: latex
%%% TeX-master: t
%%% End:

\usepackage{transparent}
\input{slide-setup/macros-from2024-slides}

\mode<presentation>
{
  \usecolortheme[named=red]{structure}
  \usetheme{Dresden}
}

\title{\textcolor{yellow}{\textbf{A type system for data flow and alias analysis in ReScript}}}

\author[N. Lund and H. Hüttel]{\textcolor{yellow}{Nicky Ask Lund} \inst{\textcolor{yellow}{1}} \and \textcolor{yellow}{Hans Hüttel} \inst{\textcolor{yellow}{2}}}

\institute[AAU and UCPH]{\inst{\textcolor{yellow}{1}}
  \textcolor{yellow}{Department of Computer Science, Aalborg University} \and \inst{\textcolor{yellow}{2}} \textcolor{yellow}{Department of Computer Science, University of Copenhagen}}

\date{\textcolor{yellow}{\textsl{Working Formal Methods
    Symposium 2024, Timi{\textcommabelow{s}}oara, Romania}}}

\begin{document}

{
\setbeamertemplate{navigation symbols}{}
\setbeamertemplate{footline}[page number]
\setbeamertemplate{background canvas}{\includegraphics
[width=\paperwidth,height=\paperheight]{timisoara-kopi.jpg}}
\maketitle
}

\section{Introduction}

\begin{frame}
  \frametitle{Dataflow analysis}

  The goal of dataflow analysis is to provide a safe static analysis
  of the flow information in a program. 

  A variable occurrence is \alert{live} at a program point $p$ if its value
  may be used at program points later than $p$.

  This information can then be used in compiler optimizations such as
  the elimination of dead code and for register allocation.
  
\end{frame}

\begin{frame}
  \frametitle{How it all began}

The initial work of (Kildall,1972) allows one to compute a safe approximation
of the program points where each variable in a program is live.

The approach uses a collection of dataflow equations. A solution for
them can be computed in time polynomial in the size of the program to
be analyzed.

There is no notion of aliasing analysis here.
\end{frame}

% \begin{frame}
%   \frametitle{Using a type system}

%   A good reason for using a type system to express this form of
%   analysis is that allows us to more easily combine the analysis with
%   other forms of static analysis.

%   $\Gamma_1 \vdash_1 e : \tau_1$ and $\Gamma_2 \vdash_2 e : \tau_2$
%   give us the conjunctive judgement $(\Gamma_1,\Gamma_2) \vdash e :
%   (\tau_1,\tau_2)$.
  
% \end{frame}

\begin{frame}
  \frametitle{Type-based approaches}

  Many more recent approaches to dataflow analysis use type systems
  and show how aliasing affects dataflow. Some of the type systems are
  \emph{substructural} in that they introduce linearity.
  
  \begin{itemize}
  \item (Smith et al., 2000) present a notion of alias types that allows
    functions to specify the shape of a store and to track the flow of
    pointers through a computation. The language is a simple
    location-based language.
  \item (Ahmed et al., 2005) use a language based on a linear
    $\lambda$-calculus to give an alternative formulation of alias
    types. Here, every well-typed program terminates.
  \end{itemize}
\end{frame}

\begin{frame}
  \frametitle{The goals of our work}

  \begin{itemize}
  \item To establish a type system for liveness analysis in the
    presence of aliasing
  \item To study this in the setting of an imperative functional
    scripting language, ReScript
  \item To see if this can be done without the need for introducing linearity
  \end{itemize}
\end{frame}

\section{ReScript}

\begin{frame}
  \frametitle{ReScript}

  ReScript is based on OCaml with a JavaScript-inspired syntax and a
  type system based on that of OCaml. 


  Just like OCaml, ReScript is imperative as well as functional. It
  allows for mutability through reference constructs for creation,
  reading, and writing.
  
\end{frame}

% \begin{frame} \frametitle{Syntax}

%   The formation rules of our abstract syntax are shown below.
% %
% \begin{align*}
% e &::= x \mid c \mid e_1\;e_2 \mid \lambda x.e \mid c \; e_1 \; e_2\\
% 			& \mid \mbox{let} \; x \; e_1 \; e_2 \mid
%                    \mbox{let rec} \; x \; e_1 \; e_2 \\
%   & \mid \mbox{case}
%                    \; e_1 \; \vec{\pi} \; \vec{e} \mid  \refc \; e \mid e_1 := e_2 \mid \; !e\\
% \pi &::= n \mid b \mid x \mid \_  \mid
%                                 (s_1,\cdots,s_n)
% \end{align*}

% \end{frame}

\begin{frame}
  \frametitle{Functional constructs of ReScript}

  ReScript contains an applied $\lambda$-calculus with recursion,
  let-bindings and pattern matching.
  
 \begin{description}
  \item[Abstractions] $\lambda\;x.e$ have a parameter $x$ and
body $o$.  

\item[Applications] are written $e_1\;e_2$. The expression occurrence
  $e_1$ is applied to the argument $e_2$.

\item[Local declarations] $\textsf{let} \; x \; e_1 \; e_2$ associate
the variable $x$ with the value $e_1$ within $e_2$.

\item[Recursive declarations]
$\textsf{let rec} \; f \; e_1 \; e_2$ allows us to define a recursive
function $f$ where $f$ may occur in $e_1$.

\item[Pattern matching] $\textsf{case} \; e_1 \; \vec{\pi} \; \vec{e}$
  matches an occurrence with an ordered set, $\vec{\pi}$, of
  patterns. 
\end{description}

\end{frame}

\begin{frame}
  \frametitle{Imperative constructs of ReScript}
  
  The presence of references makes ReScript an imperative language.

  \begin{description}
    \item[The reference construct] $\textsf{ref\;e}$ which creates a
      reference. A reference is a location containing the value of $e$.

    \item[Dereferencing] $!e$ lets us read from a reference $e$.

    \item[Assignment] $e_1 := e_2$ lets us assign the value of $e_2$
      to $e_1$ if $e_1$ is a reference.
\end{description}


\end{frame}

\section{Dependencies}

\begin{frame}[fragile]
  \frametitle{An annotated expression}

    We only annotate \emph{bound} occurrences.

  \begin{rescript}
   (let x (ref 484000$\p{1}$)$\p{2}$
     (let y (let z (5$\p{3}$)$\p{4}$
        (x$\p{5}$:=z$\p{7}$)$\p{8}$)$\p{9}$ (!x)$\p{10}$)$\p{11}$)$\p{12}$
\end{rescript}

\begin{itemize}
\item<1->This creates a reference to the constant 484000 and binds the reference to
  $x$ (so $x$ is an alias of this reference).
\item<2-> Next a binding of $z$ is made
to the constant 5 before writing to the reference, that $x$ is bound to, 
to the value that $z$ is bound to.
\item<3-> Then a binding for $y$ is made to the unit value,
  as the assignment evaluates to the unit value.
\item<4-> And finally we read the reference that $x$ is bound to.
\item<5-> We see that e.g. $x\p{5}$ depends on $z\p{7}$.
\end{itemize}

\end{frame}

  \newcommand{\desc}[2]{\ensuremath{\underbrace{#1}_{\text{\tiny{#2}}}}}

\begin{frame}
  \frametitle{The binding model of the semantics}

  Values are given by the formation rules
%
  \begin{align*}
    v & ::= \desc{c}{constants} \mid \desc{\ell}{locations} \mid
        \desc{()}{unit} \\
    & \mid  \desc{\langle x,e^{p'},env\rangle}{closure} \mid
      \desc{\langle x,f,e^{p''},env\rangle}{recursive closure}
   \end{align*}

   \begin{itemize}
   \item An environment $env$ is a partial function from variables to values.
\item A store $sto$ is a partial function from locations to values.
   \end{itemize}

\end{frame}
\begin{frame} \frametitle{Dependencies as described in the semantics}

  A dependency function $w$ is a map from occurrences to pairs of sets
  of program points.
  
  If we have that
  % 
\[ w (u^p)=(L,V) \]
%
the value of the element $u^p$ depends on
\begin{itemize}
\item the set of location occurrences
  $L=\{\loc_1^{p_1},\cdots,\loc_n^{p_n}\}$ and
\item the set of variable occurrences $V =\{x_1^{p'_1},\cdots,x_m^{p'_m}\}$

\end{itemize}

If the value of $u^p$ depends on the value of $v^q$, we write $p \leq q$.
\end{frame}

% \begin{frame}[fragile]
%   \frametitle{An ordering on program points}

%    \begin{rescript}
%    (let x (ref 484000$\p{1}$)$\p{2}$
%      (let y (let z (5$\p{3}$)$\p{4}$
%         (x$\p{5}$:=z$\p{7}$)$\p{8}$)$\p{9}$ (!x)$\p{10}$)$\p{11}$)$\p{12}$
% \end{rescript}

%       If we have a dependency function $w$, we can derive a partial order $\po$ on
%       program points.

%       In our example, the ordering is the least po containing
% %
%       \begin{align*}
%         2 \po 4 && 2 \po 9 && 5 \po 9 \\
%         2 \po 8 && 7 \po 8
%       \end{align*}

%       Likewise, any $\po$ gives rise to a dependency function $w$.
% \end{frame}

\begin{frame}
\frametitle{A collecting semantics}

The evaluation of an expression will always

\begin{itemize}
\item produce a new value
\item modify the store, if references are involved
\item reach a new program point
\item create further dependencies
\end{itemize}

So we have big-step transitions of the form
%
\begin{align*}
env\vdash\left\langle e^{p'},sto,(w,\sqleq_w),p\right\rangle\rightarrow\left\langle v,sto',(w',\sqleq_w'),(L,V),p''\right\rangle
\end{align*}

\end{frame}



 \section{The type system}

 \begin{frame}
   \frametitle{Locations and aliases at the type level}

In a program, references are separate entities that introduce
dependencies.

This means that we need a notion of \alert{internal variable} to
represent locations at the level of the type system.

Internal variables are denoted by
$\nu x,\nu y, \nu z \ldots \in\cat{IVar}$.

We use a partition of $\cat{IVar} \cup \cat{Var}$ to represent
aliasing.

Two (possibly internal) variables belong to the same partition if they are
aliases for the same location.

\end{frame}

 \begin{frame}
   \frametitle{The language of types}

   The set of types \cat{Types} is given by the formation rules
%
\[ T ::=(\delta,\kappa)\mid T_1 \rightarrow T_2 \]
%
\begin{itemize}
\item The base type $(\delta,\kappa)$ is the type of an occurrence $o$
  that is not a function.
  \begin{itemize}
  \item $\delta$ is the set of occurrences that the value of $o$ may depend
    upon
  \item $\kappa$ represents the set of variables and internal
    variables that the value of $o$ may depend upon.
  \end{itemize}
  
\item The arrow type $T_1 \rightarrow T_2$ is used for typing
  abstractions.
  \end{itemize}
\end{frame}

 
 
 \begin{frame}
   \frametitle{The binding model of the type system}

   In our type system, we assign a type to each occurrence and to each
   internal variable.

   A type environment $\Gamma$ is a partial function
   %
   \[
     \Gamma:\cat{Var}_{\mathbf{P}}\cup\cat{IVar}_{\mathbf{P}}\rightharpoonup\cat{Types} \]
   
 \end{frame}

 \begin{frame}
   \frametitle{Dependencies at the type level}

	An approximated order of program points $\Pi$ is a pair
	\[ \Pi=(\cat{P},\sqleq_\Pi) \]
	where
	\begin{itemize}
		\item \cat{P} is the set of program points in an occurrence,
		\item $\sqleq_\Pi\subseteq\cat{P}\times\cat{P}$
	\end{itemize}
	We say that $\Pi$ is a partial order if $\sqleq_\Pi$ is a partial order.

        The notion of immediate predecessor appears at the type level also,
        but now wrt. $\sqleq_\Pi$ .
        
      \end{frame}

      \begin{frame}
        \frametitle{Type judgements}

        Type judgements have the format
%
\[ \Gamma,\Pi\vdash e^p: T \]
%
and tell us that the occurrence $e^p$ has type $T$, given the
dependency bindings $\Gamma$ and the approximated order of program
points $\Pi$.



      \end{frame}
% \begin{frame}
%   \frametitle{The type rule for variables}

%   \begin{tabular}{ll}
%     \runa{T-Var} &
% 	\condinf{}
% 	{\Gamma,\Pi \vdash x^p:T \sqcup (\{x^p\},\emptyset)}{where 
%                    $x^{p'}=\IP_{ \sqleq_\Pi}(x,\Gamma)$ \alert{(most recent occurrence of $x$)}\\
%     and
%                    $\Gamma(x^{p'})=T$}
%     \end{tabular}

% \end{frame}

\begin{frame}
  \frametitle{The let type rule}

  \begin{tabular}{ll}
    \runa{T-Let-1} &
	\condinf{
		\Gamma,\Pi\vdash e_1^{p_1}:(\delta,\kappa) \\
		\Gamma',\Pi\vdash e_2^{p_2}:T_2
	}
	{\Gamma,\Pi\vdash [\mbox{let}\; x \; e_1^{p_1} \; e_2^{p_2}]^{p}:T_2}{where $\Gamma'=\Gamma[x^{p}:(\delta,\kappa\cup \{x\})]$ and
          $\kappa\neq\emptyset$}
  \end{tabular}
  
\end{frame}

\section{Results}

\begin{frame}
  \frametitle{Soundness, informally}

  We now have

  \begin{itemize}
  \item Notions of dependency and bindings at the level of the semantics, as
    described by $w$, $env$ and $sto$
  \item Notions of dependency and bindings at the level of the type system, as
    described by $\Pi$ and $\Gamma$
  \end{itemize}

  These two notions are closely related. Suppose $e^p$ is a well-typed occurrence.
  
If the type-level dependency information $(\Gamma,\Pi)$ agrees with
the information $(e^p,v,sto,w)$ at the level of the semantics, then this
will also be the case after the evaluation of $e^p$.
 
\end{frame}

\begin{frame}
  \frametitle{The type system is sound}

  \begin{theorem}[Soundness]
	Suppose $e^{p'}$ is an occurrence where
	\begin{itemize}
		\item $env\vdash\left\langle e^{p'},sto,(w,\sqleq_w),p\right\rangle\rightarrow\left\langle v,sto',(w',\sqleq_w'),(L,V),p''\right\rangle$,
		\item $\Gamma,\Pi\vdash e^{p'} : T$
		\item $\Gamma,\Pi\vdash env$
		\item $(env,sto,(w,\sqleq_w))\models(\Gamma,\Pi)$
	\end{itemize}
	Then we have that
	\begin{itemize}
		\item $\Gamma,\Pi\vdash v:T$   \alert{(type preservation)}
		\item $(env,sto',(w',\sqleq_w'))\models(\Gamma,\Pi)$
                  \alert{(agreement wrt. the ordering)}
                  
		\item $(env,(w',\sqleq_w'),v,(L,V))\models(\Gamma,T)$  \alert{(agreement wrt. the type)}
	\end{itemize}
      \end{theorem}
      
\end{frame}

\begin{frame}
  \frametitle{Type checking}

  Type checking involves
  
  \begin{itemize}
  \item annotating the code to be analyzed with program points
  \item specifying a dependency function $w$
  \item specifying the types of variable occurrences as a $\Gamma$
 \item specifying dependencies as a $\Pi$
  \item specifying the type $T$ of an occurrence $e^p$
  \item checking that $(\Gamma,\Pi) \vdash e^p : T$
  \end{itemize}

\end{frame}

\begin{frame}
  \frametitle{Type inference}

  We would probably rather

  \begin{itemize}
  \item try to compute a $\Gamma$, $\Pi$ and $T$
    given $e^p$   
  \item assume that initially $w = \emptyset$.
  \end{itemize}

  This is liveness analysis as we know it: Computing an ordering of
  program points ($\Pi$) and an overapproximation of which variables
  are alive at which program points ($\Gamma$).

  The design of an algorithm for this is future work.
\end{frame}

\begin{frame}
  \frametitle{Conclusions}

  \begin{itemize}
 
  \item We have established a type system for establishing the sets of
    live variables and location at every program point of a ReScript
    expression.
  \item The type system is a safe overapproximation of the dependency
    information found in the semantics.
  \item This is a \alert{backward analysis}, as type rules speak of immediate
    predecessors.
  \item The type system is not linear -- but references cannot be bound to
abstractions and that every abstraction is used at most used once, so
there is a "poor man's notion of linearity'' present.
  \end{itemize}
\end{frame}

\begin{frame}
  \frametitle{Future work}

  \begin{itemize}
  \item If we also classify variables according to high vs. low
    security levels, our dataflow analysis can be used for checking
    for non-interference. How does our type system compare to type
    systems for non-interference such as those by Volpano and Smith?
  \item Establish an algorithm for type inference. The constraints can
    probably be solved using usual algorithms used for dataflow analysis.
  \item Introduce notions of polymorphism to get rid of the "poor
    man's notion of linearity''. 
  \end{itemize}
\end{frame}
\end{document}

%%% Local Variables:
%%% mode: latex
%%% TeX-master: t
%%% End:
