\documentclass{llncs}
% math and typesetting
%\usepackage{mathtools}
\usepackage{amsmath,amssymb}
\usepackage{balance}

% referencing
\usepackage{hyperref}
\usepackage{natbib}
\usepackage{cleveref}

% tables, figures and boxes
\usepackage{float}
\usepackage{listings}
\usepackage{placeins}
\usepackage{subcaption}
\usepackage{wrapfig}
\usepackage{booktabs}
\usepackage{tabularx}
\usepackage[most]{tcolorbox}

\usepackage{filecontents}
\usepackage{stmaryrd}

%etc
\setlength{\marginparwidth}{2cm}
\usepackage{todonotes}
\usepackage{subfiles}

\usepackage{algpseudocode}
\usepackage{algorithm}
\usepackage[acronym]{glossaries}
\usepackage[T1]{fontenc}
\usepackage{semantic}

%\newcommand{\InfName}[1]{\ensuremath{\lbrack #1 \rbrack}}
\newcommand{\InfName}[1]{\textsc{(#1)}}
\newcommand{\InfVal}[1]{\ensuremath{\left\langle #1 \right\rangle}}

\newcommand{\runa}[1]{\textsc{(#1)}}
\newcommand{\cat}[1]{\textbf{#1}}

\newcommand{\loc}{\ensuremath{\ell}}
\newcommand{\sto}{\ensuremath{sto}}

\newcommand{\Loc}{\ensuremath{\mathbb{L}oc}}
\newcommand{\Sto}{\ensuremath{\mathbb{S}to}}
\newcommand{\Val}{\ensuremath{\mathbb{V}alues}}
\newcommand{\Env}{\ensuremath{\mathbb{E}nv}}
\newcommand{\Exp}{\ensuremath{\mathbb{E}xp}}
\newcommand{\Con}{\ensuremath{\mathbb{C}on}}
\newcommand{\Var}{\ensuremath{\mathbb{V}ar}}
\newcommand{\IVar}{\ensuremath{\mathbb{IV}ar}}
\newcommand{\AVar}{\ensuremath{\mathbb{AV}ar}}
\newcommand{\Lab}{\ensuremath{\mathbb{L}abel}}
\newcommand{\Point}{\ensuremath{\mathbb{P}oint}}
\newcommand{\ID}{\ensuremath{\mathbb{ID}}}

\newcommand{\Dep}{\ensuremath{\mathbb{A}}}
\newcommand{\Pow}[1]{\ensuremath{\mathbb{P}(#1)}}
\newcommand{\Live}{\ensuremath{\mathbb{L}}}
\newcommand{\live}{\ensuremath{a_l}}

\algnewcommand\algorithmicforeach{\textbf{for each}}
\algdef{S}[FOR]{ForEach}[1]{\algorithmicforeach\ #1\ \algorithmicdo}

\newcommand{\AllA}{\ensuremath{\mathcal{A}}}
\newcommand{\Rel}{\ensuremath{\mathcal{R}}}
\newcommand{\RelS}{\ensuremath{\mathcal{S}}}
\newcommand{\PSet}{\ensuremath{\mathcal{P}}}
\newcommand{\lex}[1][\mathcal{B}]{\ensuremath{\mathrel{\leq_{#1}}}}
\newcommand{\gsol}[1]{\ensuremath{\llbracket \! #1 \! \rrbracket}}

\newcommand{\defeq}{\stackrel{\textsf{def}}{=}}

%%% Local Variables:
%%% mode: latex
%%% TeX-master: "../pepm2024"
%%% End:

\usepackage{xspace}

\newcommand{\uf}{\ensuremath{\mathsf{IP}}}

\newcommand{\condinf}[3]{\begin{tabular}{l} \inference[]{#1}{#2} \\[5mm]
                          #3 \end{tabular}}

\newcommand{\sqleq}{\ensuremath{\sqsubseteq\xspace}}
\newcommand{\set}[1]{\ensuremath{\{ #1 \}}}
\newcommand{\setof}[2]{\set{ #1 \mid #2}}
\newcommand{\refc}{\ensuremath{\textrm{ref}\xspace}}
\newcommand{\points}{\ensuremath{\textsf{points}}}
\newcommand{\nexte}{\ensuremath{\textrm{next}}\xspace}

\newcommand{\IP}{\ensuremath{\textsf{IP}}\xspace}

%\input{frontmatter/acm.tex}

\begin{document}
\title{A type system for data flow and alias analysis in ReScript}

\author{Nicky Ask Lund \inst{1} \and Hans Hüttel \inst{2}} 
\institute{Department of Computer Science, Aalborg University, Denmark
  \email{loevendallund@gmail.com}
  \and Department of Computer Science, Copenhagen, Denmark \email{hans.huttel@di.ku.dk}}

\maketitle

\keywords{Data-flow analysis, Alias analysis, Program analysis, Programming languages, Type systems}

\begin{abstract}
	ReScript is a strongly typed language that targets
        JavaScript, as an alternative to gradually typed languages,
        such as TypeScript. In this paper, we present a sound type system
        for data-flow analysis for a subset of the ReScript language,
        more specifically for a $\lambda$-calculus with mutability and
        pattern matching. The type system is a local analysis that
        collects information about variables are used at each program
        point as well as alias information. 
\end{abstract}



\section{Introduction}

The goal of data-flow analysis is to provide a static analysis of the
flow information in a program that can be used in compiler
optimizations and for register allocation. The original approach is to
build a system of flow equations based on a graph representation of
the program and to compute a solution using an iterative algorithm
\cite{KildallGaryA1973Auat,RyderBarbara1988Idaa}. Other graph-free
approaches have also been considered
\cite{HorspoolR.Niegel2002AGAt}. A challenge in this setting is how to
deal with the aliasing that imperative language constructs introduce.

Type systems have often been used to provide static analyses of
programs in order to characterize specific run-time errors, including
ones caused by aliasing. In \cite{10.1007/3-540-46425-5_24} Smith et
al. present a notion of alias types that allows functions to specify
the shape of a store and to track the flow of pointers through a
computation. The language is a simple location-based language. Other
type systems are substructural. Ahmed et
al. \cite{DBLP:conf/tlca/MorrisettAF05} use a language based on a
linear $\lambda$-calculus to give an alternative formulation of alias
types. The type system crucially relies on linearity, and every
well-typed program terminates.

In this paper we present a type system for data flow analysis in the
presence of aliasing for a non-trivial fragment of the programming
language ReScript which is meant as an alternative to other typed
languages that target JavaScript. ReScript is based on OCaml with a
JavaScript-inspired syntax and a type system based on that of
OCaml\cite{rescript_rebrand}. ReScript is imperative and allows for
mutability through reference constructs for creation, reading, and
writing,

The fragment that we study incorporates both functional and imperative
features. We show that the type system is sound in the sense that
it correctly overapproximates the set of occurrences on which any
given occurrence depends. Moreover, the dependency information that
the type system provides can also be used to reason about information
flow properties such as non-interference.

\section{A fragment of ReScript}\label{sec:lang}

We consider a fragment of ReScript that contains a $\lambda$-calculus
with bindings, pattern matching, and mutability.
 
\subsection{Syntax}

In a data-flow analysis we must record information of where variables
are used.  Therefore, the language presented is extended with a notion
of \emph{program points} taken from a countably infinite set
$\cat{P}$.
%
Every subexpression is labelled with a unique \emph{program
  point}. \emph{Occurrences} $o \in \cat{Occ}$ are labelled
expressions $e^p$ where $e \in \cat{Exp}$ and $p \in \cat{P}$.We let
$\ell$ range over set set of locations $\cat{Loc}$ and $x,f$ range
over the set of variables $\cat{Var}$.  An occurrence is \emph{atomic}
if it is of the form $u^p$ where $u \in \cat{Var} \cup \cat{Loc}$. If
$u^p$ is atomic, we call any other occurrence $u^q$ a
\emph{$u$-occurrence}.

When given a syntactic category $\mathbf{C}$, we let
$\mathbf{C}_{\cat{P}}$ denote the pair $\mathbf{C}\times\cat{P}$, so
that e.g.  $\cat{Exp}_\cat{P}=\cat{Exp}\times\cat{P}$.  This means
that $\cat{Occ}=\cat{Exp}_{\cat{P}}$.
%As occurrences are labelled expressions, we have that $\cat{Occ}=\cat{Exp}_{\cat{P}}$.

The formation rules of our abstract syntax are shown below.
%
\begin{align*}
o &::= e^p \\
e &::= x \mid c \mid o_1\;o_2 \mid \lambda x.o \mid c \; o_1 \; o_2\\
			& \mid \mbox{let} \; x \; o_1 \; o_2 \mid
                   \mbox{let rec} \; x \; o_1 \; o_2 \mid \mbox{case}
                   \; o_1 \; \vec{\pi} \; \vec{o} \mid  \refc \; o \mid o_1 := o_2 \mid \; !o\\
\pi &::= n \mid b \mid x \mid \_  \mid
                                (s_1,\cdots,s_n)
\end{align*}
%
$\lambda\;x.o$ denotes an \emph{abstraction} with a parameter $x$ and
body $o$.  \emph{Constants} $c$ are either natural numbers $n$,
boolean values $b$, unit value $()$, or functional constants, such as
the arithmetic operations and the Boolean connectives.

$o_1\;o_2$ denotes an \emph{application}, and $c\;o_1\;o_2$ denotes a
\emph{functional application} where $c$ is a functional constant and
$o_1$, $o_2$ are its arguments.

$\mbox{let} \; x \; o_1 \; o_2$ and $\mbox{let rec} \; f \; o_1 \; o_2$
denotes immutable bindings that associate the variable $x$ with the value $o_1$ within $o_2$. 
The latter form is recursive; $x$ may occur in $o_1$. Mutability
in ReScript is introduced using the \emph{reference} construct
$\mbox{ref\;o}$. This construct creates a reference in the form of a
location and allows for binding locations to local declarations. We
can read from a reference $o$ by writing $!o$ and write to a
reference using the assignment construct $o_1\;:=\;o_2$. 

$\mbox{case} \; o_1 \; \vec{\pi} \; \vec{o}$ denotes pattern matching, that
matches an occurrence with the ordered set, $\vec{\pi}$, of patterns.
We denote the size of the tuple pattern $\pi$ by $|\vec{\pi}|$ and the
size of a tuple occurrence by $|\vec{o}|$, requiring that
$|\vec{\pi}| = |\vec{o}|$ such that for each pattern in $\vec{\pi}$
there is a clause in $\vec{o}$.

We assume that all bindings bind distinct names; this can be ensured
by means of $\alpha$-conversion. The notions of free and bound
variables are defined as expected. 

\begin{example}\label{ex:write}
Consider
\begin{lstlisting}[language=Caml, mathescape=true]
  (let x (ref 3$^1$)$^2$
  (let y (let z (5$^3$)$^4$
     (x$^5$:=z$^7$)$^8$)$^{9}$ (!x)$^{10}$)$^{11}$)$^{12}$
\end{lstlisting}
This creates a reference to the constant 3 and binds the reference to
$x$ (so $x$ is an alias of this reference). Next a binding of $z$ is made
to the constant 5 before writing to the reference, that $x$ is bound to, 
to the value that $z$ is bound to. Then a binding for $y$ is made to the unit value,
as the write construct evaluates to the unit value. Lastly the reference, that $x$ is bound to, is read.

%Next a binding for $y$ is made
%and after that a binding of $z$ to the constant 5 before writing
%to the reference, that $x$ is bound to, to the value that $z$ is bound
%to. Finally the reference that $x$ is bound to is read. We expect to
%retrieve the value $5$.
\end{example}

\subsection{The binding model}\label{sec:EnvSto}

Our binding model uses an environment that keeps
track of the bindings of variables to values. Values are given by the formation rules
%
\[ v ::= c \mid \ell \mid () \mid \langle x,e^{p'},env\rangle \mid \langle
  x,f,e^{p''},env\rangle \]
%
Constants $c$, locations $\ell$, and unit $()$ are values, as are closures, $\langle
x,e^{p'},env\rangle$ and recursive closures, $\langle x,f,e^{p''},env\rangle$.

An environment $env \in \cat{Env} $ is a partial function $env:
\cat{Var}\rightharpoonup\cat{Values}$ and we let $env^{-1}(v)=\{x\in dom(env)\mid env(x)=v\}$.
%
A store $sto \in \cat{Sto}$ is a partial function
$\cat{Sto}=\cat{Loc}\cup\{\nexte\}\rightharpoonup\cat{Values}$ where
$\nexte$ is a pointer to the next unused location -- this information
is needed when new locations are needed.

Moreover, we assume a function
$new:\cat{Loc}\rightarrow\cat{Loc}$, which given, a location, gives us
the next location.

For any function $f$ we let $f[u
\mapsto w]$ denote the function $f'$ such that $f(u') = f(u)$ for $u
\neq u'$ and $f'(u') = w$.

\subsection{Keeping track of dependencies}\label{sec:DepFunc}

The goal of the semantics that follows is to collect the semantic
dependencies as they appear in a computation. To this end, we use a
dependency function that will tell us for each variable and location
occurrence what other, previous occurrences they depend upon.

\begin{definition}[Dependency function]\label{def:DepFunc}
	The set of dependency functions, $\cat{W}$, is the set of
        partial functions from location and variable occurrences to a
        pair of dependencies, such that
\[
  \cat{W}=\cat{Loc}_{\mathbf{P}}\cup\cat{Var}_{\mathbf{P}}\rightharpoonup\Pow{\cat{Loc}_{\mathbf{P}}}\times\Pow{\cat{Var}_{\mathbf{P}}} \] 
\end{definition}
For a dependency function $w$ and a
$u^p\in\cat{Loc}_{\mathbf{P}}\cup\cat{Var}_{\mathbf{P}}$, the clause
%
\[ w (u^p)=(L,V) \]
%
tells us that the element $u^p$ is bound to a pair of location and
variable occurrences where $L$ is a set of location occurrences
$L=\{\loc_1^{p_1},\cdots,\loc_n^{p_n}\}$ and a set of variable
occurrences $V =\{x_1^{p'_1},\cdots,x_m^{p'_m}\}$, meaning that the
value of the element $u^p$ depends on the occurrences found in $L$ and
in $V$.

\begin{example}\label{ex:dep}
	Consider the occurrence from \cref{ex:write}, where we can
        infer the following bindings for a dependency function
        $w_{ex}$ over this occurrence: 
	\[
          w_{ex}= \begin{array}{l}
                    [x^2\mapsto(\emptyset,\emptyset),z^4\mapsto(\emptyset,\emptyset),y^9\mapsto(\emptyset,\{x^5\}),
                    \\ \; \loc^2\mapsto(\emptyset,\emptyset),\loc^8\mapsto(\emptyset,\{z^7\})] \end{array} \]
	where $\loc$ is the location created from the reference construct.
	The variable bindings are distinct, an the location $\loc$ is
        bound multiple times, for the program points $2$ and $8$. 

	If we want to read a variable or location in $w_{ex}$, we must
        also know for which program point since there can exists
        multiple bindings for the same variable or location. 
\end{example}

By considering \cref{ex:dep}, we would like to read the information
from the location, that $x$ is an alias to.  As it is visible from the
occurrence in \cref{ex:write}, we know that we should read from
$\loc^8$, since we wrote to that reference at the program point $8$.
From $w_{ex}$ alone it is not possible to know which occurrence to
read, since there is no order defined between the bindings.  We therefore
introduce a notion of ordering in the form of a binary relation over
program points.% :

% \begin{definition}[]\label{def:BinRel}
% 	Let \cat{P} be a set of program points in an occurrence.
% 	Then $\sqleq$ is a binary relation of \cat{P}, such that:
% 	$$\sqleq\subseteq\cat{P}\times\cat{P}$$
% \end{definition}

% Since we are interested in the ordering of the elements in a
% dependency function $w$, we will define an instantiation of
% \cref{def:BinRel}. 

\begin{definition}[Occurring program points]\label{def:OccPP}
	Let $O$ be a set of occurrences, then $\points(O)$ is given by:
	\[ \points(O)=\{p\in\cat{P}\mid\exists e^p\in O\} \]
        For a pair $(L,V)$ we let $\points(L,V) = \points(L) \cup \points(V)$.
\end{definition}

Any dependency function induces an ordering on program points as follows.

\begin{definition}\label{def:RelPoint}
	Let $w\in\cat{W}$ be a dependency function.
	Then the induced order $\sqleq_w$ is given by
	 \[ \sqleq_w= \{(p,p')\mid p \in \points(dom(w)), p' \in
           \points(w(p)) \} \]
         	We say that $w$ is a partial order if $\sqleq_w$ is a partial order.
\end{definition}

\begin{example}\label{ex:depRel}
	Consider the example from \cref{ex:dep}. Assume a binary
        relation over the dependency function $w_{ex}$ given by
	\[ \sqleq_{w_{ex}}=\{(2,4),(2,9),(5,9),(2,8),(7,8)\} \]
	From this ordering, it is easy to see the ordering of the elements.
	The ordering we present also respects the flow the occurrence from \cref{ex:write} would evaluate to.
	We then know that the dependencies for the reference (that $x$ is an alias to) is for the largest binding of $\loc$.
\end{example}

As presented in \cref{def:DepFunc} and \cref{def:RelPoint}, the
dependency function and the binary relation are used to define the
flow of information. As illustrated by \cref{ex:depRel}, we need to
lookup the greatest of $\sqleq_w$. 

The immediate predecessor $\IP(u,S)$ of an element $u^p$ wrt. a set of
occurrences $S$ is the most recent $u$-element in $S$ seen before $u$. 

\begin{definition}[Immediate predecessor]\label{def:GBind}
  Let $u$ be an element, let \sqleq\ be an ordering on program points
  and $S$ be a set of occurrences, then $\IP(u,S)$ is given by
\[ \IP(u^p,S)=\inf\{u^q\in S\mid q\sqleq p\} \]
\end{definition}

Based on \cref{def:GBind}, we can present an instantiation of the
function for the dependency function $w$ and an order over $w$,
$\sqleq_w$: 

\begin{definition}
  Let $w$ be a dependency function, $\sqleq_w$ be the induced order,
  and $u$ be an element, then $\IP_{\sqleq_w}$ is given by:
	$$\IP_{\sqleq_w}(u,w)=\inf\{u^p\in dom(w)\mid u^q\in dom(w).q\sqleq_w p\}$$
\end{definition}


\begin{example}\label{ex:deplookup}
  As a continuation of \cref{ex:depRel}, we can now find the greatest
  element for an element, e.g., a variable or location.  As we were
  interested in finding the greatest bindings a location is bound to
  in $w_ex$, we use the function $\IP_{\sqleq_w}$:
	\[ \IP_{\sqleq_{w_ex}}(\loc,w_ex)=\inf\{\loc^p\in dom(w)\mid
          \loc^q\in dom(w). q\sqleq_{w_ex} p\} \]
where the set we get for $\loc$ is 
$\{\loc^2,\loc^8\}$. From this, we find the greatest element: 
	\[ \loc^8=\inf\{\loc^2,\loc^8\} \]
	As we can see, from the $\IP_{w_ex}$ function, we got $\loc^8$ which were the occurrence we wanted.
\end{example}

\subsection{Collection semantics}\label{sec:sem}

We can now introduce the big-step semantics for our language that
collects dependency information. In this semantics, the transitions
are of the form
%
\begin{align*}
env\vdash\left\langle e^{p'},sto,(w,\sqleq_w),p\right\rangle\rightarrow\left\langle v,sto',(w',\sqleq_w'),(L,V),p''\right\rangle
\end{align*}
%
This should be read as: Given the store $sto$, a dependency function
$w$, a relation over $w$, and the previous program point $p$, the
occurrence $e^{p'}$ evaluates to a value $v$, an updated store $sto'$,
an updated dependency function $w'$, a relation over $w'$, the
dependency pair $(L,V)$, and the program point $p''$ reached after
evaluating $e^{p'}$, given the bindings in the environment $env$. 

A selection of the rules for $\rightarrow$ can be found in
\cref{fig:ColSem}.

The \runa{Var} rule  uses the environment to get the value $x$ is bound to and uses dependency function $w$ to get its dependencies.
		To lookup the dependencies, the function $\IP_{\sqleq_w}$ is used to get the greatest binding a variable is bound to, in respect to the ordering $\sqleq_w$.
		Since the occurrence of $x$ is used, it is added to the set of variable occurrences we got from the lookup of the dependencies for $x$.

The \runa{Let} rule for the occurrence $[\mbox{let}\;x\;e_1^{p_1}\;e_2^{p_2}]^{p'}$, creates a local binding that can be used in $e_2^{p_2}$.
		The \runa{Let} rule evaluate $e_1^{p_1}$, to get the value $v$, that $x$ will be bound to in the environment for $e_2^{p_2}$, and the dependencies used to evaluate $e_1^{p_1}$ are bound in the dependency function.
		As we reach the program point $p_1$ after evaluating $e_1^{p_1}$, and it is also the program point before evaluating $e_2^{p_2}$, the binding of $x$ in $w$ is to the program points $p_1$.	

The \runa{Ref} rule, for the occurrence $[\refc\;e^{p'}]^{p''}$,
creates a new location and binds it in the store $sto$, to the value
evaluated from $e^{p'}$. We record the dependencies from evaluating
the body $e^{p'}$ in $w$ at the program point $p''$.

\runa{Ref-read} evaluates the body $e^{p_1}$ to a value which
must be a location $\loc$, and reads the value of $\loc$ in the store.
		The \runa{Ref-read} rule looks up the dependencies for
                $\loc$ in $w$.
		As there could be multiple bindings for $\loc$, in $w$, at different program points, we use the $\IP_{\sqleq_{w'}}$ function to get greatest binding of $\loc$ with respect to the ordering $\sqleq_{w'}$, 
		and we also add the location occurrence $\loc^{p'}$ to the set of locations.

                Finally \runa{Ref-write} tells us that we must
                evaluate $e_1^{p_1}$ to a location $\loc$ and
                $e_2^{p_2}$ to a value $v$, and bind $\loc$ in the
                store $sto$ to the value $v$.  We pass the program
                point $p'$ and the dependency function
                is also updated with a new binding for $\loc$.

\begin{table*}
%	\setlength\tabcolsep{8pt}
	\begin{tabular}{l}
          \runa{Var}\\[0.2cm]
	\inference[]{}
	{env\vdash \left\langle x^{p'},sto,(w,\sqleq_w),p \right\rangle \rightarrow \left\langle v,sto,(w,\sqleq_w),(L,V\cup\{x^{p'}\}),p' \right\rangle}\\[6mm]
	where $env(x)=v$, $x^{p''}=\IP_{\sqleq_w}(x,w)$, and
          $w(x^{p''})=(L,V)$ \\[4mm]
	\inference[]{}
	{env\vdash \left\langle x^{p'},sto,(w,\sqleq_w),p \right\rangle \rightarrow \left\langle v,sto,(w,\sqleq_w),(L,V\cup\{x^{p'}\}),p' \right\rangle}\\[6mm]
	where $env(x)=v$, $x^{p''}=\IP_{\sqleq_w}(x,w)$, and $w(x^{p''})=(L,V)$
          \\[4mm]
                    \runa{Let}\\[0.2cm]
	\inference[]
	{
		env\vdash \left\langle e_1^{p_1},sto,(w,\sqleq_w),p \right\rangle \rightarrow \left\langle v_1,sto_1,(w_1,\sqleq_w^1),(L_1,V_1),p_1 \right\rangle &\\
		env[x\mapsto v_1]\vdash \left\langle e_2^{p_2},sto_1,(w_2,\sqleq_w^1),p_1 \right\rangle \rightarrow \left\langle v,sto',(w',\sqleq_w'),(L,V),p_2 \right\rangle
	}
	{env\vdash \left\langle \left[\mbox{let}\;x\;e_1^{p_1}\;e_2^{p_2}\right]^{p'},sto,(w,\sqleq_w),p \right\rangle \rightarrow \left\langle v,sto',(w',\sqleq_w'),(L,V),p' \right\rangle}\\[6mm]
	where $w_2=w_1[x^{p_1}\mapsto(L,V)]$
          \\[4mm]
          \runa{Ref}\\
	\inference[]
	{env \vdash \left\langle e^{p'},sto,(w,\sqleq_w),p \right\rangle \rightarrow \left\langle v,sto',(w',\sqleq_w'),(L,V),p' \right\rangle}
	{env\vdash \left\langle \left[\refc\;e^{p'}\right]^{p''},sto,(w,\sqleq_w),p \right\rangle \rightarrow \left\langle \loc,sto'',(w'',\sqleq_w'),(\emptyset,\emptyset),p'' \right\rangle}\\[6mm]
	where $\loc=next$, $sto''=sto'[next\mapsto new(\loc),\loc\mapsto v]$, and\\
          $w''=w'[\loc^{p'}\mapsto (L,V)]$ \\[4mm]
 \runa{Ref-read}\\
	\inference[]
	{env \vdash \left\langle e^{p_1},sto,(w,\sqleq_w),p \right\rangle \rightarrow \left\langle \loc,sto',(w',\sqleq_w'),(L_1,V_1),p_1 \right\rangle}
	{env\vdash \left\langle \left[!e^{p_1}\right]^{p'},sto,(w,\sqleq_w),p \right\rangle \rightarrow \left\langle v,sto',(w',\sqleq_w'),(L\cup L_1\cup\{\loc^{p''}\},V\cup V_1),p' \right\rangle}\\[6mm]
	where $sto'(\loc)=v$,
          $\loc^{p''}=\IP_{\sqleq_w'}(\loc,w')$, and
          $w'(\loc^{p''})=(L,V)$ \\[4mm]
\runa{Ref-write}\\
	\inference[]
	{
		env \vdash \left\langle e_1^{p_1},sto,(w,\sqleq_w),p \right\rangle \rightarrow \left\langle \loc,sto_1,(w_1,\sqleq_w^1),(L_1,V_1),p_1 \right\rangle &\\
		env \vdash \left\langle e_2^{p_2},sto_1,(w_1,\sqleq_w^1),p_1 \right\rangle \rightarrow \left\langle v,sto_2,(w_2,\sqleq_w^2),(L_2,V_2),p_2 \right\rangle
	}
	{env\vdash \left\langle \left[e_1^{p_1}:=e_2^{p_2}\right]^{p'},sto,(w,\sqleq_w),p \right\rangle \rightarrow \left\langle (),sto',(w',\sqleq_w'),(L_1,V_1),p' \right\rangle}\\[6mm]
	where $sto'=sto_2[\loc\mapsto v]$, $\loc^{p''}=inf_{\sqleq_w^2}(\loc,w_2)$,\\
	$w'=w_2[\loc^{p'}\mapsto(L_2,V_2)]$, and
          $\sqleq_w'=\sqleq_w^2\cup(p'',p')$ \\[6mm]
	\end{tabular}
	\caption{Selected rules from the semantics}
	\label{fig:ColSem}
\end{table*}

\iffalse
\begin{example}[Data-flow for abstractions]\label{ex:DFAbs}
The following program creates a local binding to the identity function and applies it twice.

\begin{lstlisting}[language=Caml, mathescape=true]
(let x ($\lambda$ y.(y$^1$))$^2$ (let z (x$^3$ 1$^4$)$^5$ (x$^6$ 2$^7$)$^{8}$)$^{9}$)$^{10}$
\end{lstlisting}
The transition tree can be found in \cref{FigEx.Abs}.
In the transition tree, we see that $\psi$ is extended a couple of times, where we will take a look at a couple of interesting extensions to $\psi$
The first time we evaluate the abstraction body, $\psi$ is on the following form:

$$\psi_2=(w_2=[x^{2}\mapsto(\emptyset,\emptyset), y^{4}\mapsto(\emptyset,\emptyset)],\sqleq_w^2=\emptyset)$$
Here, the lookup of the parameter $y$ is simple, as there are only one occurrence, where we then know that $inf_{\psi_2}(y)=4$.

The second time we evaluate the body of the abstraction, $\psi$ is 
%
\[ \psi_3=(w_3=[x^{2}\mapsto(\emptyset,\emptyset),
  y^{4}\mapsto(\emptyset,\emptyset),
  y^{7}\mapsto(\emptyset,\emptyset)],\sqleq_w^2=\{4,7\}) \]
Here, we now have two bindings for the parameter $y$, but since we also know that there are an ordering for the two occurrences of $y$, we then know that the program point $7$ is evaluated after $4$, as such we know that $inf_{\psi_3}(y)=7$.
\end{example}

\begin{landscape}
\subfile{../examples/DFAbs.tex}
\end{landscape}
\fi

\section{A type system for data-flow analysis}\label{sec:TypeSys}

The type system for data-flow analysis is an overapproximation of
the big-step semantics.

\subsection{An overview of the type system}

The system assigns types, presented in \cref{sec:types}, to
occurrences given a type environment (presented in
\cref{sec:Judge}) and a so-called basis (presented in
\cref{sec:basis}).

%Since the language contains local information as bindings, and global
%information as locations, the type system must be able to describe
%this.
As presented, the language contains local information as bindings and global information as locations.
Since locations a a notion of the semantics, and references do not need to be bound to variables, we user the notion of \emph{internal} variables to represent locations.
An internal variable is denoted by $\nu x,\; \nu y\in\cat{IVar}$, where $\cat{IVar}$ is the syntactic category for internal variables.
To captue the alias information, or "being an alias for", is done by means of a partition set of $\cat{IVar} \cup \cat{Var}$.
When a set of variables and internal variables are in the same set, in a partition, we say that they share the same location.

%As references do not need to be bound to a variable, and is thus represented by locations.
%In the type system we do not have semantic notions at our disposal, we instead use the notion of \emph{internal} variables to represent locations.
%These internal variables are denoted by $\nu x,\; \nu y\in\cat{IVar}$, where $\cat{IVar}$ is the syntactic category for internal variables.

%To capture the concept of "being an alias for", or binding a variable to a reference, is captured by the means of a partition of the set $\cat{IVar} \cup \cat{Var}$.
%In such a partition, we think of variables and internal variables in the same set within the partition as sharing a location.


%Locations are a notion of the semantics and are used to represent
%references. But since references need not be bound to variables, we
%must find a way to represent this information in the type system,
%where we do not have semantic notions at our disposal. In the system
%we therefore use a notion of \emph{internal} variables to represent
%them. Internal variables are denoted by $\nu x,\; \nu y\in\cat{IVar}$
%and \cat{IVar} is the syntactic category for internal variables. The
%concept of "being an alias for'' we capture by means of a partition of
%the set $\cat{IVar} \cup \cat{Var}$. In such a partition, we think of
%variables and internal variables in the same set within the partition
%as sharing a location.

In this paper, we will not introduce polymorphism into the type
system.  As such we will introduce the constraints that references
cannot be bound to abstractions and every abstraction can at most be
used once.
%We also impose the constraint on the type system that references
%cannot be bound to abstractions. Since we do not introduce
%polymorphism into our type system, every abstraction is affine -- it
%can be used at most once.

%To see the need for this assumption, consider the occurrence
%\begin{lstlisting}[language=Caml, mathescape=true]
%(let x ($\lambda$ y.y$^1$)$^2$ (x$^3$ (x$^4$ 1$^5$)$^6$)$^7$)$^8$
%\end{lstlisting}
%As can be seen, the variable $x$ is used in both program point 3 and 4.
%To type this, we first do an application with a constant as argument.
%In the second application, the argument type needs to contain the occurrence $x^4$, as it were used for the evaluation.

%To type the occurrences used in both places where we apply the
%abstractions, the type of the argument in the innermost application is
%empty, as it applies a constant. 
%For the second, and outermost, application, the argument type must
%contain the occurrence $x^4$, as it were used to evaluate the value
%for the argument. 

\subsection{Types}\label{sec:types}

The set of types \cat{Types} is defined by the following formation rules:
%
\[ T ::=(\delta,\kappa)\mid T_1 \rightarrow T_2 \]
%
The base type $(\delta,\kappa)$ contains a set, $\delta$, of occurrences upon the value of an occurrence can at most depend,
and $\kappa$ represents alias information in the form of the set of variables and internal variables upon which this occurrence may depend.
If an occurrence has type $\kappa \neq \emptyset$,
the occurrence must represent a location.

\begin{definition}[Type base for aliasing]
	For an occurrence $o$, let $\cat{Var}_o$ be the set of all
        variables found in $o$ and $\cat{IVar}_o$ be the set of all
        internal variables found in $o$.
	The type base $\kappa^0=\{\kappa^0_1,\cdots,\kappa^0_n\}$ is
        then a partition of $\cat{Var}_o \cup \cat{IVar}_o$, where
        $\kappa_i^0\cap\kappa_j^0=\emptyset$ for all $i\neq j$. 
\end{definition}
For a variable, $x$ to be an alias of an internal variable, $\nu y$,
there must exist a $\kappa^0_i$ where $x \in \kappa^0_i$ and $\nu y
\in \kappa^0_i$. 
As such, there can only be multiple variables in a $\kappa^0_i$, if
there also exists an internal variable in $\kappa^0_i$. 

%This partition represents the assumption about which variables are
%an alias to internal variables: If $x \in \kappa^0_1$ and
%$\nu y \in \kappa^0_i$, then $x$ is an alias of $\nu y$.
%As such multiple variables can only belong to the same element
%$\kappa^0_i\in\kappa^0$, if there also exists an internal variable in
%$\kappa_0^i$. 

The arrow type is introduced to type abstractions.
If either $T_1$ or $T_2$ in an arrow type $T_1 \to T_2$ is a base type where $\kappa \neq \emptyset$,
then the abstraction must either take a reference as input or return a reference.

%In order to be able to type functions, we introduce arrow types
%$T_1 \rightarrow T_2$. If an occurrence has the
%abstraction type where either $T_1$ or $T_2$ are base types with alias
%information, then the abstraction either takes a reference as input or
%returns a reference.

%\begin{example}
%Consider the following occurrence:
%\begin{lstlisting}[language=Caml, mathescape=true]
%(let x (3$^1$)$^2$ (let y (ref x$^3$)$^4$ (!y)))
%\end{lstlisting}
%Here, we can type $x$ with $(\emptyset,\emptyset)$ as $x$ is bound to a constant and there a no variables or internal variables used.
%$y$ can then be given the type $(\set{x},\{x,\nu y\})$, as the reference construct \refc creates a new reference, which $y$ is then an alias to, e.g., $y$ is bound to a location.
%Her $\nu y$ represents the reference from $ref$, and can thus be given the type $(\{x^3\},\emptyset)$, where $\nu y$ is bound to a constant, because of $x$, but the occurrence $x^3$ were used, so it should be part of the set of occurrences $\delta$.
%\end{example}

Since the type system approximates the occurrences used to evaluate an occurrence, we need a notion of combining types.
%Since the type system approximates the occurrences used to evaluate an occurrence, we introduce the notion of union of types. %we introduce two unions.
%The first union is a simple union that expects the types to be similar, that is, only the base types are allowed to be different.

\begin{definition}
  	Let $T_1$ and $T_2$ be two types, then their union is defined as

  \[
T_1\cup T_2=
    \begin{cases}
      (\delta\cup\delta',\kappa\cup\kappa') & \begin{array}{l}
                                                T_1=(\delta,\kappa) \\
      T_2=(\delta',\kappa') \end{array} \\[3mm]
       (T_1'\cup T_2')\rightarrow (T_1''\cup T_2'') &
       \begin{array}{l} T_1=T_1'\rightarrow T_1'' \\;T_2=T_2'\rightarrow
       T_2''  \end{array}\\[3mm]
       (\delta' \cup \delta, \kappa \cup \kappa') & \begin{array}{l}
                                                      T_1 =
                                                      (\delta',\kappa') \\
                                                      T_2 =
                                                      (\delta,\kappa) \end{array}
                                                    \\[3mm]
       \text{undefined} & \text{otherwise}
    \end{cases}
\]
\end{definition}
% The second type union, is to add additional type information to an arbitrary type.
% This type union is used to add an occurrence to a type, e.g., in the \runa{Var} rule where the variable occurrence needs to be added to the type of that variable.
% \begin{definition}[Base type union]
% 	Let $T$ be an type and $(\delta,\kappa)$ be a base type, then the union of these are as follows:
% 	\begin{equation*}
% 		T\sqcup (\delta,\kappa)=
% 		\left\{\begin{matrix}
% 			\mbox{If } \; T=(\delta',\kappa')  & \mbox{then} \; (\delta\cup\delta',\kappa\cup\kappa')\\
% 			\mbox{else if } \; T=T_1\rightarrow T_2 & \mbox{then} \; T_1\rightarrow(T_2\sqcup (\delta,\kappa))
% 		\end{matrix}\right.
% 	\end{equation*}
% \end{definition}

\subsection{The binding model of the type system}\label{sec:basis}

The semantics will let us find dependency information, and the type 
system must approximate these semantic notions.

%As locations do not appear at the level of the type system, we
%introduce the notion of a type base. This is a partitioning of the
%sets of variables and internal variables. 

%\begin{definition}[Type Base for aliasing]
%	For an occurrence $o$, let $\cat{Var}_o$ be the set of all
%        variables found in $o$ and $\cat{IVar}_o$ be the set of all
%        internal variables in $o$ found in $o$.
%	The type base $\kappa^0=\{\kappa^0_1,\cdots,\kappa^0_n\}$ is
%        then a partition of $\cat{Var}_o \cup \cat{IVar}_o$, where
%        $\kappa_i^0\cap\kappa_j^0=\emptyset$ for all $i\neq j$. 
%\end{definition}

%This partition represents the assumption about which variables are
%actually an alias to internal variables: If $x \in \kappa^0_1$ and
%$\nu y \in \kappa^0_i$, then $x$ is an alias of $\nu y$.

%As such multiple variables can only belong to the same element
%$\kappa_0^i\in\kappa_0$, if there also exists an internal variable in
%$\kappa_0^i$. 

A type environment tells us the types of elements.

\begin{definition}[Type Environment]
  A type environment $\Gamma$ is a partial function $\Gamma:\cat{Var}_{\mathbf{P}}\cup\cat{IVar}_{\mathbf{P}}\rightharpoonup\cat{Types}$
\end{definition}

\begin{definition}[Updating a type environment]
	Let $\Gamma$ be a type environment, let $u^p$ be an element
        and let $T$ be a type.
	We write $\Gamma[u^p:T]$ to denote the type environment $\Gamma'$ where:
	\begin{align*}
		\Gamma'(y^{p'})=
		\left\{\begin{matrix}
			\Gamma(y^{p'}) & \mbox{if}\;y^{p'}\neq u^{p}\\\	 
			T & \mbox{if}\;y^{p'}=u^{p}
		\end{matrix}\right.
	\end{align*}
\end{definition}

We also assume an ordering of program points at type level.

\begin{definition}[Approximated order of program points]
	An approximated order of program points $\Pi$ is a pair
	\[ \Pi=(\cat{P},\sqleq_\Pi) \]
	where
	\begin{itemize}
		\item \cat{P} is the set of program points in an occurrence,
		\item $\sqleq_\Pi\subseteq\cat{P}\times\cat{P}$
	\end{itemize}
	We say that $\Pi$ is a partial order if $\sqleq_\Pi$ is a partial order.
\end{definition}

The notion of the immediate predecessor of a $u\in\cat{IVar} \cup \cat{Var}$ at the type level is relative to a type environment $\Gamma$ and the approximated order $\Pi$.

\begin{definition}[Immediate predecessor at type level]\label{def:GBindPi}
	%Let $u\in \cat{Var}\cup\cat{IVar}$, be either a variable or
        %internal variable, $\Gamma$ be a type environment, and $\Pi$
        %be the approximated order of program points that is a partial
        %order, then $\IP_{\sqleq_\Pi}$ is given by: 
	\[ \IP_{\sqleq_\Pi}(u,\Gamma)=\inf\{u^p\in
          dom(\Gamma)\mid u^q\in dom(\Gamma).q\sqleq_\Pi p\} \]
\end{definition}

A lookup of variable $u^p$ in the semantics is straightforward as its
value will be unique. In the type system, however, we need to
approximate over all possible branches in an occurrence.  To this end,
we consider chains wrt. our approximate order. A $p$-chain describes
the history, or a single possible evalution, behind an occurrence $u^p$.
A set of $p$-chains can thus be used to describe what an internal variable depends on.

%and can thus be used to describe what an internal variable depends on.

%In the type system, we do not have this luxary, as the language has pattern matching, where each pattern introduces a new branch.
%To handle this, we need to approximate for all possible branches (what we also call paths) an evaluation can take.

%We need to be able to explain from which other elements in a
%computation, an element $u^p$ obtained its value at $p$. Since our
%fragment of ReScript contains conditional expressions in the form of
%pattern matching, we need to express the different paths of evaluation
%that may arise -- where each pattern in the pattern matching construct
%introduces a new path).

%To this end, we consider chains wrt. our approximate order. 
%A $p$-chain describes the history behind an occurrence $u^p$, and can
%thus be used to describe what an internal variable depends on. 

\begin{definition}[$p$-chains]
	%Let $\Pi$ be an approximated order of program points, that is a partial order, and $p$ be a program point.
  A $p$-chain, denoted as $\Pi_p^{*}$, is a maximal chain
  wrt. $\sqleq_\Pi$ whose maximal element is $p$. We write
  $\Pi_p^{*}\in\Pi$, if $\Pi_p^{*}$ is a $\Pi$-chain. For any $p$, we
  let $\Upsilon_p$ denote the set of all $p$-chains in $\Pi$.
\end{definition}

% A $p$-chain is a total order, where $\Pi_p^{*}$ does not
% contain any pairs $(p,q)\in\sqleq_\Pi$, where $p\neq q$, then
% $(p,q)\notin\sqleq_{\Pi_p^{*}}$.

We can now define the immediate predecesor for the type system wrt. the set of $p$-chains.
This is done by taking the union of all $p$-chains for an occurrence $u^p$.
%We can now define what it means to be an immediate predecessor wrt. the set of $p$-chains.
%This immediate predecesor is found as the union of the immediate predecessors of $u$.

\begin{definition}\label{def:GBindUps}
	Let $u\in \cat{Var}\cup\cat{IVar}$, be either a variable or internal variable, $\Gamma$ be a type environment, and $\Upsilon_p$ be a set of $p$-chains, then $\IP_{\Upsilon_p}$ is given by:
\[
  \IP_{\Upsilon_p}(u,\Gamma)=\bigcup_{\Pi_p^{*}\in\Upsilon_p}\IP_{\Pi_p^{*}}(u,\Gamma) \]
\end{definition}



\subsection{The type system}\label{sec:Judge}
We will now present the judgement and type rules for the language, that is, how we assign types to occurrences.

Type judgements have the format
%
\[ \Gamma,\Pi\vdash e^p: T \]
%
and should be read as: the occurrence $e^p$ has type $T$, given the
dependency bindings $\Gamma$ and the approximated order of program
points $\Pi$. 

A highlight of type rules can be found in \cref{fig:TypeSys}.

\begin{description}
	%\item[\runa{T-Const}] rule, for occurrence $c^p$, is the simplest type rule, as there is nothing to track for constants, and as such it has the type $(\emptyset,\emptyset)$.

	\item[\runa{T-Var}] rule, for occurrence $x^p$, looks up the type for $x$ in the type environment, by finding the greatest binding using \cref{def:GBindPi}, and adding the occurrence $x^p$ to the type.

	\item[\runa{T-Let-1}] rule, for occurrence $[\mbox{let}\;x\;e_1^{p_1}\;e_2^{p_2}]^p$, creates a local binding for an internal variable, with the type of $e_1^{p_1}$ that can be used in $e_2^{p_2}$.
     As such, the rule assumes that the type of $e_1^{p_1}$ is a base type with alias information, i.e., $\kappa\neq\emptyset$.
		%The \runa{T-Let-1} rule assumes that the type of $e_1^{p_1}$ is a base type with alias information, i.e., $\kappa\neq\emptyset$.
		%If this is the case, then $e_1^{p_1}$ must evaluate to a location, in the semantics.
		The other cases, when $e_1^{p_1}$ is not a base type with alias information, are handled by the \runa{T-Let-2} rule.
		%Since a pattern can be a variable, we updates the type environment with the type of $e^p$.

	\item[\runa{T-Case}] rule, for occurrence $[\mbox{case}\;e^{p}\;\vec{\pi}\;\vec{o}]^{p'}$, is an over-approximation of all cases in the pattern matching expression, by taking an union of the type of each case.
		Since the type of $e^p$ is used to evaluate the pattern matching, we also add this type to the type of the pattern matching.

	\item[\runa{T-Ref-read}] rule, for occurrence $[!e^{p}]^{p'}$, is used to retrieve the type of references, where $e^p$ must be a base type with alias information.
     Since the language contains pattern matching, there can be multiple internal variables in $\kappa$ and multiple occurrences to read from.
     To do the lookup, we use $IP_{\Upsilon_{p'}}$ to look up all the $p'$-chains.
     %As such, we need to lookup for all the possible internal variables and program points, where we use $ip_{\Upsilon_{p'}}$ to look them up.
		%Since the type system is an over-approximation, there can be multiple internal variables in $\kappa$ and multiple occurrences we need to read from.
		%As such we need to lookup for all internal variables and also possible for multiple program points.
		%As such, we use the $uf_{Upsilon_{p'}}$ to lookup for all $p'$-chains.
\end{description}

% \begin{example}[Data-flow for abstractions]
% 	Consider the following occurrence for application:
% 	\begin{lstlisting}[language=Caml, mathescape=true]
% 		(($\lambda$ y.(PLUS 3$^1$ y$^2$)$^3$)$^4$ 5$^5$)$^6$
% 	\end{lstlisting}
% 	The derivation tree for the occurrence can be found in \cref{FigEx.TAbs}.
% 	Here, we show two applications, for \runa{T-App} and \runa{T-App-const}, where we create an abstraction that adds the constant $3$ to the argument of the abstraction, and applying the constant $5$ to the abstraction.

% 	When typing the abstraction, we need too make an assumption about the parameter $y$ and the body.
% 	As we are applying a constant to the argument, we can make an assumption that the type of the parameter should be $(\emptyset,\emptyset)$.

% 	Based on this assumption for the type, we can then type the body of the abstraction.
% 	As the body is an application for a functional constant, \runa{T-App-const}, we take a union for the types of each argument.
% \end{example}

% \begin{example}[Data-flow for references]
% 	Consider the following occurrence:
% 	\begin{lstlisting}[language=Caml, mathescape=true]
%           (let x (ref 1$^1$)$^2$
%           (let y (x$^3$) (!x$^4$)$^5$)$^6$)$^7$
% 	\end{lstlisting}
% 	The derivation tree for the occurrence can be found in \cref{FigEx.TRef}.
% 	Here, we show the typing of references where we create a reference and create 2 aliases for it before reading from the reference.
% 	When typing the reference, it modifies the base type $\Gamma$ with a new internal variable.
% 	From the type information, it is clear that only $x$ and the internal variable $\nu x$ is used.
% \end{example}

\begin{table*}[h]
	\setlength\tabcolsep{8pt}
	\begin{tabular}{ll}
% \runa{T-Const} &
% 	\inference[]{}
% 	{\Gamma,\Pi\vdash  c^{p}:(\emptyset,\emptyset)} \\
\runa{T-Var} &
	\condinf{}
	{\Gamma,\Pi \vdash x^p:T \sqcup (\{x^p\},\emptyset)}{where 
	$x^{p'}=uf_{ \sqleq_\Pi}(x,\Gamma)$, and
          $\Gamma(x^{p'})=T$} \\[15mm]
\runa{T-Let-1} &
	\condinf{
		\Gamma,\Pi\vdash e_1^{p_1}:(\delta,\kappa) \\
		\Gamma',\Pi\vdash e_2^{p_2}:T_2
	}
	{\Gamma,\Pi\vdash [\mbox{let}\; x \; e_1^{p_1} \; e_2^{p_2}]^{p}:T_2}{where $\Gamma'=\Gamma[x^{p}:(\delta,\kappa\cup \{x\})]$ and
          $\kappa\neq\emptyset$} \\
\runa{T-Case} &
	\condinf
	{
		\Gamma,\Pi\vdash e^{p}:(\delta,\kappa) \\
		\Gamma',\Pi\vdash e_i^{p_i}:T_i\;\;\;(1\leq i\leq|\vec{\pi}|)
	}
	{\Gamma,\Pi\vdash [\mbox{case}\;e^{p}\;\vec{\pi}\;\vec{o}]^{p'}:T\sqcup(\delta,\kappa)}
	{where $e_i^{p_i}\in\vec{o}$ and $s_i\in\vec{\pi}$ $T=\bigcup_{i=1}^{|\vec{\pi}|}T_i$, and\\
          $\Gamma'=\Gamma[x^p:(\delta,\kappa)]$ if $s_i=x$} \\[18mm]
\runa{T-Ref-read} &
	\condinf
	{\Gamma,\Pi\vdash  e^{p}:(\delta,\kappa)}
	{\Gamma,\Pi\vdash
                    [!e^{p}]^{p'}:T\cup(\delta\cup\delta',\emptyset)}
                   % {a}}
	{where $\left\{\begin{array}{l}\kappa\neq\emptyset$, $\delta'=\{\nu x^{p'}\mid\nu x\in\kappa\}$, $\nu x_1,\cdots,\nu x_n\in\kappa\\ 
	\{\nu x_1^{p_1},\cdots,\nu x_1^{p_m}\}=uf_{\Upsilon_{p'}}(\nu
                  x_1,\Gamma),\cdots,\\ \quad\{\nu x_n^{p_1'},\cdots,\nu
                  x_n^{p_s'}\}=uf_{\Upsilon_{p'}}(\nu x_n,\Gamma) \\
	T=\Gamma(\nu x_1^{p_1})\cup\cdots\cup\Gamma(\nu
                         x_1^{p_m})\cup\cdots\cup\\ \quad\Gamma(\nu
                         x_n^{p_1'})\cup\cdots\cup\Gamma(\nu
                         x_n^{p_s'})\end{array}\right\}$}
          \\
          & 
	\end{tabular}
	\caption{Selected rules from the type system}
	\label{fig:TypeSys}
\end{table*}

% \begin{landscape}
% \subfile{../examples/DFAbsT.tex}
% \end{landscape}

\section{Soundness}\label{sec:Soundness}

The type system is sound in that the type of
an occurrence correspond to the dependencies and the alias information
from the semantics. 
To show this, we will first introduce the
type rules for values and then describe  relation between the semantics and the
type system. 

\subsection{Type rules for values}

In our soundness theorem and its proof, values are mentioned. We
therefore state a collection of type rules for the values presented
int \cref{sec:EnvSto}. The type rules are given in
\cref{fig:ValTypeRules}. We describe the central ones here.

\begin{description}
\item[\runa{Constant}] differs from the rule \runa{T-Const}, since
  most occurrences can evaluate to a constant and as such we know that
  its type should be a base type. Constants can depend on other
  occurrences; we know that $\delta$ can be non-empty, but since
  constants are not locations, we also know that it cannot contain
  alias information, and as such $\kappa$ should be empty.

\item[\runa{Location}] types locations, and their type must be a base
  type. Since locations can depend on other occurrences, we know that
  $\delta$ can be non-empty.  As locations can contains alias
  information, and that a location is considered to always be an alias
  to itself, we know that $\kappa$ can never be empty, as it should
  always contain an internal variable.

	\item[\runa{Closure}] type rule represents abstraction, and as
          such we know that it should have the abstraction type,
          $T_1\rightarrow T_2$, where we need to make an assumption
          about the argument type $T_1$. 
		Since a closure contains the parameter, body, and the
                environment for an abstraction from when it were
                declared, we also need to handle those part in the
                type rule.
                
		The components of the closure are handled in the
                premises, where the environment must be well-typed.
                We also type the body of the abstraction in a type
                environment updated with the type $T_1$ of its
                parameter.

	% \item[\runa{RClosure}] is similar to the
        %   \runa{Closure} rule, but since this is a recursive closure,
        %   we also update the type environment with the
        %   name of the recursive binding to the type of the
        %   abstraction. 

	% \item[\runa{Unit}] simply assigns the base type, as this is
        %   not an abstraction and it also cannot have alias
        %   information. As the unit value is introduced from writing to
        %   references, $o=[o_1\;:=\;o_2]^p$, we know that from the type
        %   rule \runa{Ref-write} that the dependencies from the
        %   occurrence $o$ should also contain the set of occurrences. 
	% 	As such, the \runa{Unit} rule also contains a set of
        %         occurrences, $\delta$. 
\end{description}

As closures and recursive closures contain an environment, we also
need to define what it means to be a well-typed environment $env$
wrt. a type environment: Every variable bound in $env$ is bound to a
value that is well-typed wrt. $\Gamma$.

\begin{definition}[Well-typed environments]\label{def:TEnv}
	Let $v_1,\cdots,v_n$ be values such that $\Gamma,\Pi\vdash v_i:T_i$, for $1\leq i\leq n$.
	Let $env$ be an environment given by $env=[x_1\mapsto
        v_1,\cdots,x_n\mapsto v_n]$, $\Gamma$ be a type environment,
        and $\Pi$ be the approximated order of program points. 
	We say that:
	$$\Gamma,\Pi\vdash env$$
	iff 
	\begin{itemize}
		\item For all $x_i\in dom(env)$ then $\exists x_i^p\in dom(\Gamma)$ where $\Gamma(x_i^p)=T_i$ then 
			$$\Gamma,\Pi\vdash env(x_i):T_i$$
	\end{itemize}
\end{definition}

\begin{table*}
	\setlength\tabcolsep{8pt}
	\begin{tabular}{ll}
		\runa{Constant} &
			\inference[]{}
				{\Gamma,\Pi\vdash  c:(\delta, \emptyset)}\\[1cm]

		\runa{Location} &
			\inference[]{}
				{\Gamma,\Pi\vdash  \loc:(\delta, \kappa)}\\
				Where $\kappa\neq\emptyset$\\[1cm]

		\runa{Closure} &
			\inference[]
				{
					\Gamma,\Pi\vdash env \\
					\Gamma[x^{p}:T_1],\Pi\vdash e^{p'}:T_2
				}
				{\Gamma,\Pi\vdash \left\langle x^{p}, e^{p'}, env \right\rangle^{p''}:T_1\rightarrow T_2}% \\[1cm]

		% \runa{RClosure} &
		% 	\inference[]
		% 		{
		% 			\Gamma,\Pi\vdash env \\
		% 			\Gamma[x^{p}:T_1,f^{p'}:T_1\rightarrow T_2],\Pi\vdash e^{p''}:T_2
		% 		}
		% 		{\Gamma,\Pi\vdash \left\langle x^{p}, f^{p'}, e^{p''}, env \right\rangle^{p_3}:T_1\rightarrow T_2}\\[1cm]

		% \runa{Unit} &
		% 	\inference[]{}
		% 		{\Gamma,\Pi\vdash  ():(\delta,\emptyset)}\\[0.5cm]
	\end{tabular}
	\caption{Type rules for values}
	\label{fig:ValTypeRules}
\end{table*}

\subsection{Notions of agreement}

As our soundness theorem relates the type system to the semantics, we
must define what it means for instances of the binding models of the
semantics and the type system to agree.

First we define what it means for a set of occurrences $\delta$ to
faithfully represent the information from a dependency pair $(L,V)$
wrt. a environment $env$.

\begin{definition}[Dependency agreement]\label{def:DepAgree}
	% Let $env$ be an environment, $(L,V)$ be a dependency pair, and $\delta$ be a set of occurrences.
	We say that:
	$$(env,(L,V))\models\delta$$
	if
	\begin{itemize}
		\item $V\subseteq\delta$,
		\item For all $\loc^p\in L$ where
                  $env^{\loc}\neq\emptyset$, we then have $env^{-1} \loc \subseteq \kappa_i^0$ for some $\kappa_i^0\in\delta$
		\item For all $\loc^p\in L$ where $env^{\loc}=\emptyset$ then there exists a $\kappa_i^0\in\delta$ such that $\kappa_i^0\subseteq\cat{IVar}$
	\end{itemize}
      \end{definition}

      Since types can contain alias information $\kappa$, we also need
      to define what it means for the information in $\kappa$ to be
      known to an environment $env$.  If there exists alias
      information in $env$, then there exists an alias base
      $\kappa^0_i\in\kappa^0$ such that the alias information known to
      $env$ is included in that of $\kappa^0_i$, and there exists a
      $\nu x\in\kappa$, such that $\nu x\in \kappa^0_i$.  If there is
      no currently known alias information, we simply check that there
      exists a corresponding internal variable, that is part of an
      alias base.

      \begin{definition}[Alias agreement]\label{def:AliasAgree}
	% Let $env$ be an environment, $w$ be a pair of a dependency function, $\sqleq_w$ be a relation over $w$, $\loc$ be a location, and $\kappa$ be an alias set.
	We say that
	$$(env,(w,\sqleq_w),\loc)\models(\Gamma,\kappa)$$
	if
	\begin{itemize}
		\item $\exists \loc^p\in dom(w).\nu x^p\in dom(\Gamma)\Rightarrow\nu x\in\kappa$
		\item $env^{-1}(\loc)\neq\emptyset.\exists \kappa^0_i\in\kappa^0\Rightarrow
			(env^{-1}(\loc)\subseteq\kappa^0_i)\wedge(\exists \loc^p\in dom(w).\nu x^p\in dom(\Gamma)\Rightarrow\nu x\in\kappa^0_i\wedge\nu x\in\kappa)$
		\item $env^{-1}(\loc)=\emptyset.\exists \kappa^0_i\in\kappa^0\Rightarrow
			(\exists \loc^p\in dom(w).\nu x^p\in dom(\Gamma)\Rightarrow\nu x\in\kappa^0_i\wedge\nu x\in\kappa)$
	\end{itemize}
\end{definition}

%\subsubsection{Agreement between values and types}
% Next, we define what it means for a value $v$ to agree with a type $T$.
If a value $v$ is a location, then we check that both the set of occurrences agrees with the dependency pair, presented in \cref{def:DepAgree}, 
and check if the alias information agrees with the semantics, \cref{def:AliasAgree}.
If the value $v$ is not a location, then its type can either be an
abstraction type or a base type.
For the base type, we check that the agreement between the set of occurrences and the dependency pair agrees.
If the type is an abstraction, then we check that $T_2$ agrees with 
the binding model. 
We are only concerned about the return type $T_2$ for abstractions,
since if the argument parameter is used in the body of the
abstraction, then the dependencies would already be part of the return
type. 

\begin{definition}[Type agreement]\label{def:TAgree}
	% Let $env$ be an environment, $w$ be a dependency function, $\sqleq_w$ be a relation over $w$, $(L,V)$ be a dependency pair, and $T$ be a type.
	We say that
	$$(env,v,(w,\sqleq_w),(L,V))\models(\Gamma,T)$$
	iff
	\begin{itemize}
		\item $v\neq\loc$ and $T=T_1\rightarrow T_2$:
		\begin{itemize}
			\item $(env,v,(w,\sqleq_w),(L,V))\models(\Gamma,T_2)$
		\end{itemize}

		\item $v\neq\loc$ and $T=(\delta,\kappa)$:
		\begin{itemize}
			\item $(env,(L,V))\models\delta$
		\end{itemize}

		\item $v=\loc$ then $T=(\delta,\kappa)$ where:
		\begin{itemize}
			\item $(env,(L,V))\models\delta$
			\item $(env,(w,\sqleq_w),v)\models(\Gamma,\kappa)$
		\end{itemize}
	\end{itemize}
\end{definition}

% \subsubsection{Agreements between bindings}
% This definition of agreement will ensure that the bindings in the
% semantics and the type system speak of the same elements in the same way.
% We are now finally able to define agreement between the binding models, i.e.,
% define what it means for a type environment and an approximated order
% of program points $(\Gamma,\Pi)$ to be in agreement with an
% environment, a store, and a dependency function $(env,sto,w)$.

\begin{definition}[Environment agreement]\label{def:EnvAgree}
	% Let $(w,\sqleq_w)$ be a pair containing the dependency function and a relation over it, $env$ be an environment, $sto$ be the a store, $\Gamma$ be a type environment, and $\Pi$ be an approximated program point order.
	We say that:
	$$(env,sto,(w,\sqleq_w))\models(\Gamma,\Pi)$$
	if 
	\begin{enumerate}
		\item \label{prop:1} $\forall x\in dom(env).(\exists x^p\in dom(w))\wedge(x^p\in dom(w)\Rightarrow \exists x^p\in dom(\Gamma))$
		\item \label{prop:2} $\forall x^p\in dom(w).x^p\in dom(\Gamma)\Rightarrow env(x)=v\wedge w(x^p)=(L,V)\wedge\Gamma(x^p)=T.\\(env,v,(w,\sqleq_w),(L,V))\models (\Gamma,T)$
		\item \label{prop:3} $\forall \loc\in dom(sto).(\exists \loc^p\in dom(w))\wedge(\exists \nu x.\forall p\in\{p'\mid\loc^{p'}\in dom(w)\}\Rightarrow\\\nu x^p\in dom(\Gamma))$
		\item \label{prop:4} $\forall \loc^p \in dom(w).\exists\nu x^{p}\in dom(\Gamma)\Rightarrow w(\loc^p)=(L,V)\wedge\Gamma(\nu x^{p})=\\T.(env,\loc,(w,\sqleq_w),(L,V))\models T$
		\item \label{prop:5} if $p_1\sqleq_w p_2$ then $p_1\sqleq_\Pi p_2$
	\item \label{prop:6} $\forall \loc^p\in dom(w).\exists \nu
          x^p\in dom(\Gamma)\Rightarrow\exists
          p'\in\cat{P}.\IP_{\sqleq_w}(\loc,w)\in
          \IP_{\Upsilon_{p'}}(\nu x,\Gamma)$ 
	\end{enumerate}
\end{definition}

%  In
% particular, the variables mentioned in the environment $env$ and in
% $w$ have dependency information recorded in $\Gamma$ \eqref{prop:1}.
% It will also ensure that variables mentioned in $w$ have values that
% are in agreement with the type information \eqref{prop:2} -- this
% notion of value/type agreement is defined later.


% i.e., if the dependencies in the semantics is also captured in the
% type environment, the alias information is captured, that $\Pi$ is a
% good approximation, in respect to $w$, and the $p$-chains captures the
% global occurrence.  As such, the type environment focuses on three
% areas: \cat{1)} local information variables, \cat{2)} the global
% information for references, and \cat{3)} the approximated order of
% program points.  It should be noted at the agreement only relates the
% information known by $env$, $sto$, and $w$.

\begin{itemize}
	\item The agreement for local information only relates the information currently known by $env$, and that the information known by $w$ and $\Gamma$ agrees, in respect to \cref{def:TAgree}.
		This is ensured by \eqref{prop:1} and \eqref{prop:2}.

	\item We similarly handle agreement for the global information
          known, which is ensured by \eqref{prop:3} and
          \eqref{prop:4}. 
		Since $\Gamma$ contains the global information for
                references, we require that there exists a
                corresponding internal variable to the currently known
                locations, by comparing them by program points. 
		We also ensure that the dependency information for a location
                occurrence agrees with the type of a corresponding
                internal variable occurrence as given by \cref{def:TAgree}. 

	\item We also need to ensure that $\Pi$ is a good
          approximation of the order $\sqleq_w$ and the greatest
          binding function for $p$-chains ensures that we always get
          the necessary reference occurrences. 
	\eqref{prop:5} ensures that the ordering information
        $\sqleq_w$ agrees with that of $\Pi$.

   \item We finally need to ensure that for every location
        known, there exists a corresponding internal variable
        where, getting the greatest binding of this occurrence,
        $\loc^p$, there exists a program point $p'$, such that
        looking up all greatest bindings for the $p'$-chain, there
        exists an internal variable occurrence that corresponds to
        $\loc^p$. This is captured by \eqref{prop:6}.
\end{itemize}

% \subsection{Properties of the type system}

% The soundness proof that follows makes use of certain properties of
% the semantics and type system.

% The first property we present is for the dependency function, since
% the dependency function is global, and as such can contain information
% that arises from the side effects on an evaluation. The property
% states that if any new variable bindings are introduced to the
% dependency function by evaluating an occurrence $e^p$, those
% variables are not free in $e^p$.

% \begin{lemma}[History]\label{lemma:His}
% 	Suppose $e^p$ is an occurrence, that
% 	$$env\vdash\left\langle e^{p},sto,(w,\sqleq_w),p'\right\rangle\rightarrow\left\langle v,sto',(w',\sqleq_w'),(L,V),p''\right\rangle$$
% 		and $x^{p_1}\in dom(w')\backslash dom(w)$.
% 		Then $x\notin fv(e^{p})$
% \end{lemma}
% %The proof for \cref{lemma:His} can be found in \cref{app:HisProof}.

% %\subfile{HisProof/index.tex}

% The second property is the strengthening of the type environment, which states that if there is a binding the type environment, used to type an occurrence $e^p$, and the variables is not free in $e^p$ then the binding can be removed.

% \begin{lemma}[Strengthening]\label{lemma:Strength}
% 	If $\Gamma[x^{p'}:T'],\Pi\vdash e^{p}:T$ and $x\notin fv(e^p)$, then $\Gamma,\Pi\vdash e^{p}:T$
% \end{lemma}
% %The proof for \cref{lemma:Strength} can be found in \cref{app:StrProof}.
% %\subfile{StrProof/index.tex}

\subsection{The soundness theorem}

We can now present the soundness theorem for our type system.

\begin{theorem}[Soundness]
	Suppose $e^{p'}$ is an occurrence where
	\begin{itemize}
		\item $env\vdash\left\langle e^{p'},sto,(w,\sqleq_w),p\right\rangle\rightarrow\left\langle v,sto',(w',\sqleq_w'),(L,V),p''\right\rangle$,
		\item $\Gamma,\Pi\vdash e^{p'} : T$
		\item $\Gamma,\Pi\vdash env$
		\item $(env,sto,(w,\sqleq_w))\models(\Gamma,\Pi)$
	\end{itemize}
	Then we have that
	\begin{itemize}
		\item $\Gamma,\Pi\vdash v:T$
		\item $(env,sto',(w',\sqleq_w'))\models(\Gamma,\Pi)$
		\item $(env,(w',\sqleq_w'),v,(L,V))\models(\Gamma,T)$
	\end{itemize}
\end{theorem}
% \begin{proof}(Outline)
% 	The proof proceeds by induction on the height of the derivation tree for 
% 	$$env\vdash\left\langle e^{p'},sto,\psi,p\right\rangle\rightarrow\left\langle v,sto',\psi',(L,V),p''\right\rangle$$
% 	%We will only show the proof of four rules here, for \runa{Var}, \runa{Case}, \runa{Ref}, and \runa{Ref-write}.% , the full proof can be found in \cref{app:SoundnessProof}.

% 	% \begin{description}
% 	% 	\item[\runa{Var}] Here $e^{p'}=x^{p'}$, where
\begin{figure}[H]
	\setlength\tabcolsep{8pt}
	\begin{tabular}{l}
		\item[\runa{Var}] Here $e^{p'}=x^{p'}$, where
\begin{figure}[H]
	\setlength\tabcolsep{8pt}
	\begin{tabular}{l}
		\item[\runa{Var}] Here $e^{p'}=x^{p'}$, where
\begin{figure}[H]
	\setlength\tabcolsep{8pt}
	\begin{tabular}{l}
		\input{sections/appendix/ColRules/var.tex}
	\end{tabular}
\end{figure}
And from our assumptions, we have:
\begin{itemize}
	\item $\Gamma,\Pi\vdash x^{p'} : T$
	\item $\Gamma,\Pi\vdash env$
	\item $(env,sto,(w,\sqsubseteq_w))\models(\Gamma,\Pi)$
\end{itemize}
To type the occurrence $x^{p'}$ we use the rule \runa{T-Var}:
\begin{figure}[H]
	\setlength\tabcolsep{8pt}
	\begin{tabular}{l}
		\runa{T-Var}\\[0.2cm]
			\inference[]{}
			{\Gamma,\Pi \vdash x^p:T \sqcup (\{x^p\},\emptyset)}
	\end{tabular}
\end{figure}
Where $x^{p''}=uf_{\sqsubseteq_\Pi}(x,\Gamma)$, $\Gamma(x^{p''})=T$.

We need to show that \cat{1)} $\Gamma,\Pi\vdash c:T$, \cat{2)} $(env,sto',(w',\sqsubseteq_w'))\models(\Gamma,\Pi)$, and\\
\cat{3)} $(env,v,(w',\sqsubseteq_w'),(L,V))\models(\Gamma,T)$.
\begin{description}
	\item[1)] Since, from our assumption, we know that $\Gamma,\Pi\vdash env$, we can then conclude that $\Gamma,\Pi\vdash v:T$

	\item[2)] Since there are no updates to $sto$ and
          $(w,\sqsubseteq_w)$, we then know from our assumptions that
          it is the case that $(env,sto,(w,\sqsubseteq_w))\models(\Gamma,\Pi)$ holds after an evaluation.

	\item[3)] Since there are no updates to $sto'$ and
          $(w',\sqsubseteq_w')$, since $(L,V) = w(x^{p''})$ and since $T = \Gamma(x^{p''})$,
		we then know that $(env,v,(w',\sqsubseteq_w'),(L,V))\models(\Gamma,T)$.
		Due to \cref{def:TAgree} we can conclude that: 
		$$(env,v,(w',\sqsubseteq_w'),(L,V\cup\{x^{p''}\}))\models(\Gamma,T\sqcup \{x^{p''}\})$$
\end{description}

	\end{tabular}
\end{figure}
And from our assumptions, we have:
\begin{itemize}
	\item $\Gamma,\Pi\vdash x^{p'} : T$
	\item $\Gamma,\Pi\vdash env$
	\item $(env,sto,(w,\sqsubseteq_w))\models(\Gamma,\Pi)$
\end{itemize}
To type the occurrence $x^{p'}$ we use the rule \runa{T-Var}:
\begin{figure}[H]
	\setlength\tabcolsep{8pt}
	\begin{tabular}{l}
		\runa{T-Var}\\[0.2cm]
			\inference[]{}
			{\Gamma,\Pi \vdash x^p:T \sqcup (\{x^p\},\emptyset)}
	\end{tabular}
\end{figure}
Where $x^{p''}=uf_{\sqsubseteq_\Pi}(x,\Gamma)$, $\Gamma(x^{p''})=T$.

We need to show that \cat{1)} $\Gamma,\Pi\vdash c:T$, \cat{2)} $(env,sto',(w',\sqsubseteq_w'))\models(\Gamma,\Pi)$, and\\
\cat{3)} $(env,v,(w',\sqsubseteq_w'),(L,V))\models(\Gamma,T)$.
\begin{description}
	\item[1)] Since, from our assumption, we know that $\Gamma,\Pi\vdash env$, we can then conclude that $\Gamma,\Pi\vdash v:T$

	\item[2)] Since there are no updates to $sto$ and
          $(w,\sqsubseteq_w)$, we then know from our assumptions that
          it is the case that $(env,sto,(w,\sqsubseteq_w))\models(\Gamma,\Pi)$ holds after an evaluation.

	\item[3)] Since there are no updates to $sto'$ and
          $(w',\sqsubseteq_w')$, since $(L,V) = w(x^{p''})$ and since $T = \Gamma(x^{p''})$,
		we then know that $(env,v,(w',\sqsubseteq_w'),(L,V))\models(\Gamma,T)$.
		Due to \cref{def:TAgree} we can conclude that: 
		$$(env,v,(w',\sqsubseteq_w'),(L,V\cup\{x^{p''}\}))\models(\Gamma,T\sqcup \{x^{p''}\})$$
\end{description}

	\end{tabular}
\end{figure}
And from our assumptions, we have:
\begin{itemize}
	\item $\Gamma,\Pi\vdash x^{p'} : T$
	\item $\Gamma,\Pi\vdash env$
	\item $(env,sto,(w,\sqsubseteq_w))\models(\Gamma,\Pi)$
\end{itemize}
To type the occurrence $x^{p'}$ we use the rule \runa{T-Var}:
\begin{figure}[H]
	\setlength\tabcolsep{8pt}
	\begin{tabular}{l}
		\runa{T-Var}\\[0.2cm]
			\inference[]{}
			{\Gamma,\Pi \vdash x^p:T \sqcup (\{x^p\},\emptyset)}
	\end{tabular}
\end{figure}
Where $x^{p''}=uf_{\sqsubseteq_\Pi}(x,\Gamma)$, $\Gamma(x^{p''})=T$.

We need to show that \cat{1)} $\Gamma,\Pi\vdash c:T$, \cat{2)} $(env,sto',(w',\sqsubseteq_w'))\models(\Gamma,\Pi)$, and\\
\cat{3)} $(env,v,(w',\sqsubseteq_w'),(L,V))\models(\Gamma,T)$.
\begin{description}
	\item[1)] Since, from our assumption, we know that $\Gamma,\Pi\vdash env$, we can then conclude that $\Gamma,\Pi\vdash v:T$

	\item[2)] Since there are no updates to $sto$ and
          $(w,\sqsubseteq_w)$, we then know from our assumptions that
          it is the case that $(env,sto,(w,\sqsubseteq_w))\models(\Gamma,\Pi)$ holds after an evaluation.

	\item[3)] Since there are no updates to $sto'$ and
          $(w',\sqsubseteq_w')$, since $(L,V) = w(x^{p''})$ and since $T = \Gamma(x^{p''})$,
		we then know that $(env,v,(w',\sqsubseteq_w'),(L,V))\models(\Gamma,T)$.
		Due to \cref{def:TAgree} we can conclude that: 
		$$(env,v,(w',\sqsubseteq_w'),(L,V\cup\{x^{p''}\}))\models(\Gamma,T\sqcup \{x^{p''}\})$$
\end{description}

% 	% 	\input{sections/Proof/SoundProof/case.tex}
% 	% 	\input{sections/Proof/SoundProof/refread.tex}
% 	% \end{description}
% \end{proof}

% \subfile{SoundProof/index.tex}

\section{Conclusion}\label{sec:Conc}

We have introduced a type system for local data-flow analysis for a
subset of ReScript that includes functional as well as imperative
feature, notably that of references.

The type system provides a safe approximation of the data flow in an
expression. This also allows us to reason about security
properties. In particular, the notion of non-interference introduced
by Goguen and Meseguer \cite{goguen-meseguer} and studied in information-flow analysis can be
understood in this setting. A program satisfies the non-interference
property if the variables classified as \emph{low} cannot be affected
by variables classified as \emph{high}. This corresponds to the
absence of chains in $\Pi$ in which low occurrences appear below high
occurrences. A topic of further work is to understand the relative
expressive power of our system wrt. the systems of Volpano and Smith
\cite{volpano-smith-96,volpano-smith-97}.

On the other hand, the system contains slack. In particular, the type
system is monomorphic. This means that a locally declared abstraction
cannot be used at multiple places, even though this may be safe, as
this would mean it would contain occurrences at multiple program
points. Moreover, abstractions cannot be bound to references.

Polymorphism for the base type $(\delta,\kappa)$ would allow 
abstractions to be used multiple times in an occurrence. Consider as
an example

\begin{lstlisting}[language=Caml, mathescape=true]
(let x ($\lambda$ y.y$^1$)$^2$ (x$^3$ (x$^4$ 1$^5$)$^6$)$^7$)$^8$
\end{lstlisting}

Occurrences such as this would now become typable,
since when typing the applications, the type of the argument changes,
as the occurrence $x^4$ is present in the second application. 

The way references are defined currently in the type system, they cannot be bound to abstractions.
If this should be introduced a couple of questions need to be evaluated.
First there should be looked into base type polymorphism, the second
would be to look into type polymorphism, i.e., allow a reference to be bound to
an arrow type at one point and a base type at another.

%Moreover, if a reference had different types, e.g., if it had
%an abstraction type at one point and a base type at another point, we should
%either require references to always have the same type, e.g., with base type polymorphism.

%Here, either we should require references to always have the same
%type, e.g., with base type polymorphism.

%For the base type polymorphism consider the following occurrence:
%Consider the following occurrence:
%\begin{lstlisting}[language=Caml, mathescape=true]
%((!(case 1 (1$^1$) (let z 5$^2$ (ref ($\lambda$ y.(PLUS z$^3$ y$^4$)$^5$)$^6$))$^7$)$^8$)$^9$ 5)$^{10}$
%\end{lstlisting}
%This occurrence would create a reference to a local abstraction which depends on the locally declared variable $z$ before reading from the reference and applying the constant to it.
%In the semantics, the environment would be added to the abstraction closure, and when evaluating the body of the abstraction, in an application, it would use the environment in the closure.

A next step is to devise a type inference algorithm for the type
system. An inference algorithm must compute an approximated order of
program points, a proper $\kappa_0$ and the types for abstractions,
that is, find all the places where the parameter of an abstraction
should be bound.  We conjecture that such a type inference algorithm
for our system will be able to compute the information found in an
interative data flow analysis.

\bibliography{bib}
\bibliographystyle{plain}
\end{document}



%%% Local Variables:
%%% mode: latex
%%% TeX-master: t
%%% End:
