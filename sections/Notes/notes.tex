\documentclass[../../master.tex]{subfiles}
\begin{document}
\section{Notes-old}

\subsection{Free and bound variables}
We denote $\tau(s)$, for a pattern $s$, as
$$
	\tau(s)=
		\left\{\begin{matrix}
			\{x\} & \mbox{if}\;s=x\\ 
			\emptyset & \mbox{otherwise}
		\end{matrix}\right.
$$

\begin{definition}[Free variables]\label{def:fv}
	The set of free variables is given by:
	\begin{align*}
		fv(x^p)&=\{x\}\\
		fv(c^p)&=\emptyset\\
		fv([\lambda\;y.e_1^{p'}]^p)&=fv(e_1^{p''})\backslash\{y\}\\
		fv([e_1^{p'}\;e_2^{p''}]^p)&=fv(e_1^{p'})\cup fv(e_2^{p''})\\
		fv([\mbox{let}\;y\;e_1^{p'}\;e_2^{p''}]^p)&=fv(e_1^{p'})\cup fv(e_2^{p''})\backslash\{y\}\\
		fv([\mbox{let rec}\;f\;e_1^{p'}\;e_2^{p''}]^p)&=fv(e_1^{p'})\cup fv(e_2^{p''})\backslash\{f\}\\
		fv([\mbox{case}\;e^{p'}\;\pi^{p''}]^p)&=fv(e_1^{p'})\cup fv(\pi)\\
		fv([(s\;e^{p'})\;\pi])&=fv(e^{p'})\cup fv(\pi)\backslash\tau(s)\\
		fv([(s\;e^{p'})])&=fv(e^{p'})\backslash\tau(s)\\
		fv([\mbox{ref}\;e^{p'}]^p)&=fv(e^{p'})\\
		fv([!e^{p'}]^p)&=fv(e^{p'})\\
		fv([e_1^{p'}\;:=\;e_2^{p''}]^p)&=fv(e_1^{p'})\cup fv(e_2^{p''})\\
	\end{align*}
\end{definition}

\begin{definition}[Bound variables]
	The set of bound variables is given by:
	\begin{align*}
		bv(x^p)&=\emptyset\\
		bv(c^p)&=\emptyset\\
		bv([\lambda\;y.e_1^{p'}]^p)&=bv(e_1^{p'})\cup\{y\}\\
		bv([e_1^{p'}\;e_2^{p''}]^p)&=bv(e_1^{p'})\cup bv(e_2^{p''})\\
		bv([\mbox{let}\;y\;e_1^{p'}\;e_2^{p''}]^p)&=bv(e_1^{p'})\cup bv(e_2^{p''})\cup\{y\}\\
		bv([\mbox{let rec}\;f\;e_1^{p'}\;e_2^{p''}]^p)&=bv(e_1^{p'})\cup bv(e_2^{p''})\cup\{f\}\\
		bv([\mbox{case}\;e^{p'}\;\pi^{p''}]^p)&=bv(e_1^{p'})\cup bv(\pi)\\
		bv([(s\;e^{p'})\;\pi])&=bv(e^{p'})\cup bv(\pi)\cup\tau(s)\\
		bv([(s\;e^{p'})])&=bv(e^{p'})\cup\tau(s)\\
		bv([\mbox{ref}\;e^{p'}]^p)&=bv(e^{p'})\\
		bv([!e^{p'}]^p)&=bv(e^{p'})\\
		bv([e_1^{p'}\;:=\;e_2^{p''}]^p)&=bv(e_1^{p'})\cup bv(e_2^{p''})\\
	\end{align*}
\end{definition}


\subsection{Basis}
This section introduces the basis for the type checker, as such the type bases er assumptions.
\begin{definition}[Type Base for aliasing]
	For a program $p$, let $var\in\cat{Var}$ be a set of all variables in $p$ and $ivar\in\cat{IVar}$ be the set of all internal variables in $p$.
	The type base $\kappa^0=\{\kappa^0_1,\cdots,\kappa^0_n\}$ is a then partition of $var\cup ivar$, where $\kappa_i^0\cap\kappa_j^0=\emptyset$ for all $i\neq j$.
\end{definition}
The type base for aliasing is an assumption for which variables that shares the same internal variable, and as such, shares the same location.
As such, if we have a an assumption for a variable $x$, such that $\{x\}=\kappa_i$, then $x$ would not be an alias to a location.
On the otherhand if we have $\{y,\nu y\}=\kappa_j$, for a variable $y$ and internal variable $\nu y$, then $y$ is an alias for $\nu y$.

\subsection{Type environment}
\begin{definition}[Type Environment]
	A type environment $\Gamma$ is a partial function $\Gamma:Id^P\rightharpoonup TYPES$
\end{definition}
As such, $\Gamma$ represents the dependencies for variables and internal variables.

\begin{definition}[Approximated order of program points]
	An approximated order of program points $\Pi$, such that: 
	\begin{itemize}
		\item  $p\sqsubseteq p'\in\Pi$
		\item The order of program points is transitive, such that if $p\sqsubseteq p'\in\Pi$ and $p'\sqsubseteq p''\in\Pi$ then $p\sqsubseteq p''\in\Pi$.
	\end{itemize}
\end{definition}
The intuition behind an approximated order of program points is the execution order, and should as such follow all possible evaluation paths for a given program.

\subsection{Agreement}
\todo[inline]{This section talks about the relation between the semantics and type system}
\todo[inline]{$env,sto,w$ is a model of $\Gamma$}
The first agreement we present here is the agreement for the type environment $\Gamma$ and dependency function $w$.
The $\Gamma$ contains both local dependencies for variable occurrences and global dependencies for internal variable occurrences, while $w$ contains global dependencies for variable and location occurrences.

As such. it is clear that $\Gamma$ is not a direct model of $w$, but models the current state that is evaluated, thus it models both the environment, store, and dependency function.
This is clear, as both the environment and type environment both only contains local information for variables.

On the other hand, locations are treated as global information in the type system, which are represented as internal variables, as such if a location exists in the store it would thus the equivalent would exists in the type environment.
Since the type environment contains only occurrences, locations and internal variables can thus be compared by program points, as locations and internal variables are created and updated at the same program points.
This relation is done by comparing between between the store, dependency function and type environment.

Lastly, since the dependency function and type environment collects information about dependencies, the dependencies ar then compared.
Since there a both local and global information, and the dependency function collects all information globally, the dependency comparison is done only on the the information  for occurrences that exists in the domain of both $w$ and $\Gamma$.

\begin{definition}[Environment agreement]\label{def:EnvAgree}
	Let $w$ be a dependency function, $env$ be an environment, $sto$ be the a store, $\Gamma$ be a type environment, and $\Pi$ be an approximated program point order.
	We say that:
	$$(env,sto,w)\models(\Gamma,\Pi)$$
	if 
	\todo[inline]{Combine 1 and 2, and combine the 4 and 5}
	\begin{itemize}
		\item $\forall x\in dom(env).(\exists x^p\in dom(w))\wedge(x^p\in dom(w)\Rightarrow \exists x^p\in dom(\Gamma))$
		\item $\forall x^p\in dom(w).x^p\in dom(\Gamma)\Rightarrow w(x^p)=(L,V)\wedge\Gamma(x^p)=T.(w,env,(L,V))\models T$
		\item $\forall \loc\in dom(sto).(\exists \loc^p\in dom(w))\wedge(\exists \nu x.\forall p\in\{p'\mid\loc^{p'}\in dom(w)\}\Rightarrow\nu x^p\in dom(\Gamma))$
		\item $\forall \loc^p \in dom(w).\exists\nu x^{p}\in dom(\Gamma)\Rightarrow w(\loc^p)=(L,V)\wedge\Gamma(\nu x^{p})=T.(w,env,(L,V))\models T$
		\item if $p_1\sqsubseteq_w p_2$ then $p_1\sqsubseteq_\Pi p_2$
	\end{itemize}
\end{definition}

\begin{definition}[Environment agreement]\label{def:EnvAgree}
	Let $w$ be a dependency function, $env$ be an environment, $sto$ be the a store, and $\Gamma$ be a type environment.
	We say that:
	$$(env,sto,w)\models\Gamma$$
	if 
	\todo[inline]{Combine 1 and 2, and combine the 4 and 5}
	\begin{itemize}
		\item $\forall x\in dom(env)\Rightarrow\exists x^p\in dom(w)$
		\item $\forall x\in dom(env).x^p\in dom(w)\Rightarrow \exists x^p\in dom(\Gamma)$
		\item $\forall x^p\in dom(w).x^p\in dom(\Gamma)\Rightarrow w(x^p)=(L,V)\wedge\Gamma(x^p)=T.(w,env,(L,V))\models T$
		\item $\forall \loc\in dom(sto)\Rightarrow\exists \loc^p\in dom(w)$
		\item $\forall \loc\in dom(sto)\Rightarrow\exists \nu x.\forall p\in\{p'\mid\loc^{p'}\in dom(w)\}$ then $\nu x^p\in dom(\Gamma)$
		\item $\forall \loc^p \in dom(w).\exists\nu x^{p}\in dom(\Gamma)\Rightarrow w(\loc^p)=(L,V)\wedge\Gamma(\nu x^{p})=T.(w,env,(L,V))\models T$
	\end{itemize}
\end{definition}
In \cref{def:EnvAgree}, the agreement contains a lot of conditions, but in essence it states:
\begin{description}
	\item[variables] If a variable exists in $env$, then there exists an occurrence of that variable in both $w$ and $\Gamma$.
		And for all variable occurrences, $x^p$, that both $w$ and $\Gamma$ knows about, the dependency and type for $x^p$ agrees.
	\item[locations] The agreement is similar for locations, but there is an extra step as the type system represents locations as internal variables.
		The comparison is thus done by comparing program points and since this is global, the comparison is done for all program points that exists for a a location $\loc$ in $w$.
\end{description}

\begin{definition}[Type agreement]\label{def:TAgree}
	Let $w$ be a dependency function, $env$ be an environment, $(L,V)$ be a dependency pair, and $T$ be a type.
	We say that:
	$$(w,env,v,(L,V))\models(\Gamma,T)$$
	iff
	\begin{itemize}
		\item $v\neq\loc$ and $T=T_1\rightarrow T_2$:
		\begin{itemize}
			\item $(w,env,v,(L,V))\models(\Gamma,T_2)$
		\end{itemize}

		\item $v\neq\loc$ and $T=(\delta,\kappa)$:
		\begin{itemize}
			\item $(env,(L,V))\models\delta$
		\end{itemize}

		\item $v=\loc$ then $T=(\delta,\kappa)$ where:
		\begin{itemize}
			\item $(env,(L,V))\models\delta$
			\item $(w,env,v)\models(\Gamma,\kappa)$
		\end{itemize}
	\end{itemize}
\end{definition}

\begin{definition}[Dependency agreement]\label{def:DepAgree}
	Let $env$ be an environment, $(L,V)$ be a dependency pair, and $\delta$ be a set of variables.
	We say that:
	$$(env,(L,V))\models\delta$$
	if
	\begin{itemize}
		\item $V\subseteq\delta$,
		\item For all $\loc^p\in L$ where $\exists x\in dom(env).env(x)=\loc$, we then have $\{x\in dom(env)\mid env(x)=\loc\}\subseteq \kappa_i^0$ for a $\kappa_i^0\in\delta$
		\item For all $\loc^p\in L$ where $\not\exists x \in dom(env).env(x)=\loc$ then there exists a $\kappa_i^0\in\delta$ such that $\kappa_i^0\subseteq\cat{IVar}$
	\end{itemize}
\end{definition}

\begin{definition}[Alias agreement]\label{def:AliasAgree}
	Let $env$ be an environment, $\loc$ be a location, and $\kappa$ be an alias set.
	We say that:
	$$(env,w,\loc)\models(\Gamma,\kappa)$$
	if
	\begin{itemize}
		\item $\exists \loc^p\in dom(w).\nu x^p\in dom(\Gamma)\Rightarrow\nu x\in\kappa^0_i$
		\item $env^{-1}(\loc)\neq\emptyset$ then $\exists \kappa^0_i\in\kappa.(env^{-1}(\loc)\subseteq\kappa^0_1)\wedge(\exists \loc^p\in dom(w).\nu x^p\in dom(\Gamma)\Rightarrow\nu x\in\kappa^0_i)$
		\item $env^{-1}(\loc)=\emptyset$ then $\exists \kappa^0_i\in\kappa.(\kappa^0_i\subseteq\cat{IVar})\wedge(\exists \loc^p\in dom(w).\nu x^p\in dom(\Gamma)\Rightarrow\nu x\in\kappa^0_i)$
		%\item $\forall\loc^p\in dom(w).(env^{-1}(\loc)\cap\cat{Var}\neq\emptyset)\Rightarrow(\exists \kappa^0_i\in\kappa.\exists\nu x^p\in dom(\Gamma).\nu x\in\kappa^0_i\wedge env^{-1}(\loc)\subseteq\kappa^0_i)$
		%\item $\forall\loc^p\in dom(w).(env^{-1}(\loc)\subseteq\cat{IVar})\Rightarrow(\exists \kappa^0_i\in\kappa.\exists\nu x^p\in dom(\Gamma).\nu x\in env^{-1}(\loc)\subseteq\kappa^0_i)$
	\end{itemize}
\end{definition}

\subsection{Judgement}
\begin{definition}[Environment judgement]
	Let $v_1,\cdots,v_n$ be values such that $\Gamma';\Pi\vdash v_i:T_i$, for $1\leq i\leq n$.
	Let $env$ be an environment where $env=[x_1\mapsto v_1,\cdots,x_n\mapsto v_n]$, $\Gamma$ be a type environment, and $\Pi$ be the approximated order of program points.
	We say that:
	$$\Gamma;\Pi\vdash env$$
	iff 
	\begin{itemize}
		\item For all $x_i\in dom(env)$ then $\exists x_i^p\in dom(\Gamma)$ where $\Gamma(x_i^p)=T_i$ then 
			$$\Gamma,\Pi\vdash env(x_i):T_i$$
	\end{itemize}
\end{definition}

\subsection{Type rules for values}
For the sake of proving the type system, we present type rules for values.
The formation rules for values can be given by:
$$v::=c\mid\loc\mid\left\langle x^{p},e^{p'},env\right\rangle\mid\left\langle x^{p},f^{p'},e^{p''},env\right\rangle\mid ()$$
Where the type rules is given in \cref{fig:ValTypeRules}.
\begin{figure}[H]
	\setlength\tabcolsep{8pt}
	\begin{tabular}{l}
		\hline\\
		\runa{Constant}\\[0.4cm]
			\inference[]{}
				{\Gamma;\Pi\vdash  c:(\emptyset, \emptyset)}\\[1cm]

		\runa{Location}\\[0.4cm]
			\inference[]{}
				{\Gamma;\Pi\vdash  \loc:(\delta, \kappa)}\\[1cm]

		\runa{Closure}\\[0.4cm]
			\inference[]
				{
					\Gamma;\Pi\vdash env \\
					\Gamma,x^{p}:T_1;\Pi\vdash e^{p'}:T_2
				}
				{\Gamma;\Pi\vdash \left\langle x^{p}, e^{p'}, env \right\rangle^{p''}:T_1\rightarrow T_2}\\[1cm]

		\runa{Recursive closure}\\[0.4cm]
			\inference[]
				{
					\Gamma;\Pi\vdash env \\
					\Gamma,x^{p}:T_1,f^{p'}:T_1\rightarrow T_2;\Pi\vdash e^{p''}:T_2
				}
				{\Gamma;\Pi\vdash \left\langle x^{p}, f^{p'}, e^{p''}, env \right\rangle^{p_3}:T_1\rightarrow T_2}\\[1cm]

		\runa{Unit}\\[0.4cm]
			\inference[]{}
			{\Gamma;\Pi\vdash  ():(\delta,\kappa)}\\[0.5cm]
		\hline
	\end{tabular}
	\caption{Type rules for values}
	\label{fig:ValTypeRules}
\end{figure}

\subsection{Lemma}
\begin{lemma}[History]\label{lemma:His}
	If 
	$$env\vdash\left\langle e^{p'},w,sto,p\right\rangle\rightarrow\left\langle v,w',sto',(L,V),p''\right\rangle$$
		and $x^{p_1}\in dom(w')\backslash dom(w)$ then $x^{p_1}\notin fv(e)$
\end{lemma}
\begin{proof}
	The proof proceeds by induction on the height of the derivation tree for $env\vdash\left\langle e^{p'},w,sto,p\right\rangle\rightarrow\left\langle v,w',sto',(L,V),p''\right\rangle$.
	\begin{description}
		\item[\runa{Cons}] Here $e^{p'}=c^{p'}$, where
		$$env\vdash\left\langle c^{p'},w,sto,p\right\rangle\rightarrow\left\langle c,w',sto',(\emptyset,\emptyset),p'\right\rangle$$
		Where $w=w'$ and $sto'=sto$.
		This case follows immediate as $dom(w')\backslash dom(w)=\emptyset$.

		\item[\runa{Var}] Here $e^{p'}=x^{p'}$, where
		$$env\vdash\left\langle x^{p'},w,sto,p'\right\rangle\rightarrow\left\langle v,w',sto',(L,V),p'\right\rangle$$
		Where $w=w'$, $sto'=sto$, $env(x)=v$, $inf_{p'}(x,w)=p''$, and $w(x^{p''})=(L,V)$.
		This case follows immediate as $dom(w')\backslash dom(w)=\emptyset$.
	\end{description}
	Next, follows the induction step:

	\begin{description}
		\item[\runa{Abs}] Here $e^{p'}=[\lambda y.e_1^{p''}]^{p'}$, where
		$$env\vdash\left\langle [\lambda y.e_1^{p''}]^{p'},w,sto,p\right\rangle\rightarrow\left\langle v,w',sto',(\emptyset,\emptyset),p'\right\rangle$$
		Where $w=w'$, $sto'=sto$, and $v=\left\langle y,e_1^{p''},env\right\rangle$.
		This case follows immediate as $dom(w')\backslash dom(w)=\emptyset$.
		
		\item[\runa{App}] Here $e^{p'}=[e_1^{p_1}\;e_2^{p_2}]^{p'}$, where
		\begin{figure}[H]
			\setlength\tabcolsep{8pt}
			\begin{tabular}{l}
					\inference[]
					{env \vdash \left\langle e_1^{p'},sto,w,p \right\rangle \rightarrow \left\langle v,sto',w',(L,V),p' \right\rangle &\\
					env \vdash \left\langle e_2^{p''},sto',w',p' \right\rangle \rightarrow \left\langle v',sto'',w'',(L',V'),p'' \right\rangle &\\
					env[y\mapsto v'] \vdash \left\langle e_3^{p_1},sto'',w_3,p'' \right\rangle \rightarrow \left\langle v'',sto_3,w_4,(L'',V''),p_1 \right\rangle}
					{env\vdash \left\langle [e_1^{p'}\;e_2^{p''}]^{p_3},sto,w,p \right\rangle \rightarrow \left\langle v'',sto_3,w_4,(L\cup L'',V\cup V''),p_1 \right\rangle}\\
			\end{tabular}
		\end{figure}
		Where $v=\left\langle y,e_3^{p_1},env\right\rangle$ and $w_3=w''[y^{p''}\mapsto(L',V')]$.
		By our induction hypothesis, we have:
		\begin{itemize}
			\item if $x^{p_3}\in dom(w')\backslash dom(w)$ then $x\notin fv(e_1^{p'})$
			\item if $x^{p_3}\in dom(w'')\backslash dom(w')$ then $x\notin fv(e_2^{p''})$
			\item if $x^{p_3}\in dom(w_4)\backslash dom(w_3)$ then $x\notin fv(e_3^{p_1})$
		\end{itemize}
		To show that this holds, we need to show for the extension \runa{App} introduces to $w$.
		By \cref{def:fv}, we know that the first premise evaluates to a closure, and $fv([e_1^{p_1}\;e_2^{p_2}]^{p'})=fv(e_1^{p_1})\cup fv(e_2^{p_2})$.
		Since we know that $y\notin fv(e_1^{p_1})$, and all variables are destinct, we also know that $y\notin fv([e_1^{p_1}\;e_2^{p_2}]^{p'})$.
		We can thus conclude that:
		\begin{itemize}
			\item if $x^{p_3}\in dom(w_4)\backslash dom(w)$ then $x\notin fv(e^{p'})$
		\end{itemize}

	\item[\runa{Let}] Here $e^{p'}=[\mbox{let}\;x\;e_1^{p_1}\;e_2^{p_2}]^{p'}$, where
		\begin{figure}[H]
			\setlength\tabcolsep{8pt}
			\begin{tabular}{l}
			\inference[]
				{env\vdash \left\langle e_1^{p'},sto,w,p \right\rangle \rightarrow \left\langle v,sto',w',(L,V),p' \right\rangle &\\
				env[y\mapsto v]\vdash \left\langle e_2^{p''},sto',w'',p' \right\rangle \rightarrow \left\langle v',sto'',w_3,(L',V'),p'' \right\rangle}
				{env\vdash \left\langle [\mbox{let}\;y\;e_1^{p'}\;e_2^{p''}]^{p_3},sto,w,p \right\rangle \rightarrow \left\langle v',sto'',w_3,(L',V'),p_3 \right\rangle}
			\end{tabular}
		\end{figure}
		Where $w''=w'[y^{p'}\mapsto(L,V)]$.
		By our induction hypothesis, we have:
		\begin{itemize}
			\item if $x^{p_3}\in dom(w')\backslash dom(w)$ then $x\notin fv(e_1^{p'})$
			\item if $x^{p_3}\in dom(w_3)\backslash dom(w'')$ then $x\notin fv(e_2^{p''})$
		\end{itemize}
		Then by \cref{def:fv}, we must also have
		\begin{itemize}
			\item if $x^{p_3}\in dom(w_3)\backslash dom(w)$ then $x\notin fv(e^{p'})$
		\end{itemize}
		This also hold true for the extension, $w'[y^{p'}\mapsto(L,V)]$, since by \cref{def:fv} we know that $y\notin fv(e^{p'})$.

	\item[\runa{Let rec}] Here $e^{p'}=[\mbox{let rec}\;x\;e_1^{p_1}\;e_2^{p_2}]^{p'}$, where
		\begin{figure}[H]
			\setlength\tabcolsep{8pt}
			\begin{tabular}{l}
			\inference[]
				{env\vdash \left\langle e_1^{p'},sto,w,p \right\rangle \rightarrow \left\langle v,sto',w',(L,V),p' \right\rangle &\\
				env[f\mapsto\left\langle x,f,e_1^{p'},env''\right\rangle]\vdash \left\langle e_2^{p''},sto',w',p' \right\rangle \rightarrow \left\langle v',sto'',w'',(L',V'),p'' \right\rangle}
				{env\vdash \left\langle [\mbox{let rec}\;f\;e_1^{p'}\;e_2^{p''}]^{p_3},sto,w,p \right\rangle \rightarrow \left\langle v',sto,w_3,(L',V'),p_3 \right\rangle}
			\end{tabular}
		\end{figure}
		Where $v=\left\langle y,e_1^{p'},env''\right\rangle$ and $w_3=w''[f^{p''}\mapsto(L',V')]$.
		By our induction hypothesis, we have:
		\begin{itemize}
			\item if $x^{p_3}\in dom(w')\backslash dom(w)$ then $x\notin fv(e_1^{p'})$
			\item if $x^{p_3}\in dom(w_3)\backslash dom(w'')$ then $x\notin fv(e_2^{p''})$
		\end{itemize}
		Then by \cref{def:fv}, we must also have
		\begin{itemize}
			\item if $x^{p_3}\in dom(w_3)\backslash dom(w)$ then $x\notin fv(e^{p'})$
		\end{itemize}
		This also hold true for the extension, $w'[f^{p'}\mapsto(L,V)]$, since by \cref{def:fv} we know that $f\notin fv(e^{p'})$.

	\item[\runa{Case}] Here $e^{p'}=[\mbox{case}\;e_1^{p_1}\;\pi^{p_2}]^{p'}$, where
		\begin{figure}[H]
			\setlength\tabcolsep{8pt}
			\begin{tabular}{l}
			\inference[]
				{env \vdash \left\langle e_1^{p_1},sto,w,p \right\rangle \rightarrow \left\langle v,sto',w',(L,V),p_1 \right\rangle &\\
				env \vdash \left\langle (v,(L,V),\pi^{p_2}),sto',w',p_1 \right\rangle \rightarrow \left\langle v',sto'',w'',(L',V'),p_2 \right\rangle}
				{env\vdash \left\langle [\mbox{case}\;e_1^{p_1}\;\pi^{p_2}]^{p'},sto,w,p \right\rangle \rightarrow \left\langle v',sto'',w'',(L\cup L',V\cup V'),p' \right\rangle}
			\end{tabular}
		\end{figure}
		By our induction hypothesis, we have:
		\begin{itemize}
			\item if $x^{p_3}\in dom(w')\backslash dom(w)$ then $x\notin fv(e_1^{p'})$
			\item if $x^{p_3}\in dom(w'')\backslash dom(w')$ then $x\notin fv(\pi^{p''})$
		\end{itemize}
		Then by \cref{def:fv}, we must also have
		\begin{itemize}
			\item if $x^{p_3}\in dom(w'')\backslash dom(w)$ then $x\notin fv(e^{p'})$
		\end{itemize}

	\item[\runa{Match 1}] Here $e^{p'}=[(v,(L,V),(s\;e_1^{p_1})\;\pi^{p_2}]^{p'}$, where
		\begin{figure}[H]
			\setlength\tabcolsep{8pt}
			\begin{tabular}{l}
			\inference[]
				{env\sigma \vdash \left\langle e_1^{p_1},sto,w',p \right\rangle \rightarrow \left\langle v,sto',w'',(L,V),p_1 \right\rangle}
				{env\vdash \left\langle [(v,(L,V),(s\;e_1^{p_1})\;\pi^{p_2})]^{p'},sto,w',p \right\rangle \rightarrow \left\langle v,sto',w'',(L,V),p' \right\rangle}
			\end{tabular}
		\end{figure}
		Where $match(v,s)=\sigma$, $\phi=match_w(s,(L,V))$ and $w'=w\phi$.
		By our induction hypothesis, we have:
		\begin{itemize}
			\item if $x^{p_3}\in dom(w'')\backslash dom(w')$ then $x\notin fv(e_1^{p_1})$
		\end{itemize}
		Then by \cref{def:fv}, for $fv(e^{p'})$, we know that it holds for all $x\notin\tau(s)$.
		From \cref{def:fv} we know that for all $x\in\tau(s)$, then $fv([(v,(L,V),(s\;e^{p_1})\;\pi^{p_2})]^{p'})= fv(e^{p_1}) \cup fv(\pi^{p_2})\backslash\tau(s)$, as such the we can conclude that:
		\begin{itemize}
			\item if $x^{p_3}\in dom(w'')\backslash dom(w)$ then $x\notin fv(e^{p'})$
		\end{itemize}

	\item[\runa{Match 2}] Here $e^{p'}=[(v,(L,V),(s\;e_1^{p_1}))]^{p'}$, where
		\begin{figure}[H]
			\setlength\tabcolsep{8pt}
			\begin{tabular}{l}
			\inference[]
				{env\sigma \vdash \left\langle e_1^{p_1},sto,w',p \right\rangle \rightarrow \left\langle v,sto',w'',(L,V),p' \right\rangle}
				{env\vdash \left\langle [(v,(L,V),(s\;e_1^{p_1}))]^{p_3},sto,w',p \right\rangle \rightarrow \left\langle v,sto',w'',(L,V),p_3 \right\rangle}
			\end{tabular}
		\end{figure}
		Where $match(v,s)=\sigma$, $\phi=match_w(s,(L,V))$ and $w'=w\phi$.
		By our induction hypothesis, we have:
		\begin{itemize}
			\item if $x^{p_3}\in dom(w'')\backslash dom(w')$ then $x\notin fv(e_1^{p_1})$
		\end{itemize}
		Then by \cref{def:fv}, for $fv(e^{p'})$, we know that it holds for all $x\notin\tau(s)$.
		From \cref{def:fv} we know that for all $x\in\tau(s)$, then $fv([(v,(L,V),(s\;e^{p_1})\;\pi^{p_2})]^{p'})= fv(e^{p_1}) \cup fv(\pi^{p_2})\backslash\tau(s)$, as such the we can conclude that:
		\begin{itemize}
			\item if $x^{p_3}\in dom(w'')\backslash dom(w)$ then $x\notin fv(e^{p'})$
		\end{itemize}

	\item[\runa{Match $\perp$}] Here $e^{p'}=[(v,(L,V),(s\;e_1^{p_1})\;\pi^{p_2}]^{p'}$, where
		\begin{figure}[H]
			\setlength\tabcolsep{8pt}
			\begin{tabular}{l}
			\inference[]
				{env \vdash \left\langle (v,(L,V),\pi^{p''}),sto,w,p \right\rangle \rightarrow \left\langle v,sto',w',(L,V),p'' \right\rangle}
				{env\vdash \left\langle [(v,(L,V),(s\;e^{p'})\pi^{p''})]^{p_3},sto,w,p \right\rangle \rightarrow \left\langle v,sto',w',(L,V),p_3 \right\rangle}
			\end{tabular}
		\end{figure}
		Where $match(v,s)=\perp$
		By our induction hypothesis, we have:
		\begin{itemize}
			\item if $x^{p_3}\in dom(w'')\backslash dom(w')$ then $x\notin fv(\pi^{p_2})$
		\end{itemize}
		From \cref{def:fv} we know that is must also hold for:
		\begin{itemize}
			\item if $x^{p_3}\in dom(w'')\backslash dom(w)$ then $x\notin fv(e^{p'})$
		\end{itemize}

	\item[\runa{Ref}] Here $e^{p'}=[\mbox{ref}\;e_1^{p_1}]^{p'}$, where
		\begin{figure}[H]
			\setlength\tabcolsep{8pt}
			\begin{tabular}{l}
			\inference[]
				{env \vdash \left\langle e^{p_1},sto,w,p \right\rangle \rightarrow \left\langle v,sto',w',(L,V),p_1 \right\rangle}
				{env\vdash \left\langle [\mbox{ref}\;e^{p_1}]^{p'},sto,w,p \right\rangle \rightarrow \left\langle \loc,sto''[\loc\mapsto v],w'',(\emptyset,\emptyset),p' \right\rangle}
			\end{tabular}
		\end{figure}
		Where $sto''=sto'[next\mapsto new\;\loc]$ and $w''=w'[\loc^{p'}\mapsto (L,V)]$.
		By our induction hypothesis, we have:
		\begin{itemize}
			\item if $x^{p_3}\in dom(w')\backslash dom(w)$ then $x\notin fv(e_1^{p_1})$
		\end{itemize}
		Then by \cref{def:fv}, we must also have
		\begin{itemize}
			\item if $x^{p_3}\in dom(w')\backslash dom(w)$ then $x\notin fv(e^{p'})$
		\end{itemize}

	\item[\runa{Ref read}] Here $e^{p'}=[!e_1^{p_1}]^{p'}$, where
		\begin{figure}[H]
			\setlength\tabcolsep{8pt}
			\begin{tabular}{l}
			\inference[]
				{env \vdash \left\langle e^{p_1},sto,w,p \right\rangle \rightarrow \left\langle \loc,sto',w',(L,V),p_1 \right\rangle}
				{env\vdash \left\langle [!e^{p_1}]^{p'},sto,w,p \right\rangle \rightarrow \left\langle v,sto',w',(L\cup\{\loc^{p'}\},V),p' \right\rangle}
			\end{tabular}
		\end{figure}
		Where $sto'(\loc)=v$.
		By our induction hypothesis, we have:
		\begin{itemize}
			\item if $x^{p_3}\in dom(w')\backslash dom(w)$ then $x\notin fv(e_1^{p_1})$
		\end{itemize}
		Then by \cref{def:fv}, we must also have
		\begin{itemize}
			\item if $x^{p_3}\in dom(w')\backslash dom(w)$ then $x\notin fv(e^{p'})$
		\end{itemize}

	\item[\runa{Ref write}] Here $e^{p'}=[e_1^{p_1}:=e_2^{p_2}]^{p'}$, where
		\begin{figure}[H]
			\setlength\tabcolsep{8pt}
			\begin{tabular}{l}
			\inference[]
				{env \vdash \left\langle e_1^{p_1},sto,w,p \right\rangle \rightarrow \left\langle \loc,sto',w',(L,V),p_1 \right\rangle &\\
				env \vdash \left\langle e_2^{p_2},sto',w',p_1 \right\rangle \rightarrow \left\langle v,sto'',w'',(L',V'),p_2 \right\rangle}
				{env\vdash \left\langle [e_1^{p_1}:=e_2^{p_2}]^{p'},sto,w,p \right\rangle \rightarrow \left\langle (),sto_3,w_3,(L,V),p' \right\rangle}
			\end{tabular}
		\end{figure}
		Where $sto_3=sto''[\loc\mapsto v]$ and $w_3=w''[\loc^{p_3}\mapsto(L',V')]$.
		By our induction hypothesis, we have:
		\begin{itemize}
			\item if $x^{p_3}\in dom(w')\backslash dom(w)$ then $x\notin fv(e_1^{p_1})$
			\item if $x^{p_3}\in dom(w'')\backslash dom(w')$ then $x\notin fv(e_2^{p_2})$
		\end{itemize}
		Then by \cref{def:fv}, we must also have
		\begin{itemize}
			\item if $x^{p_3}\in dom(w'')\backslash dom(w)$ then $x\notin fv(e^{p'})$
		\end{itemize}
	\end{description}
\end{proof}

\begin{lemma}[Strengthening]\label{lemma:Strength}
	If $\Gamma,x^{p'}:T';\Pi\vdash e^{p}:T$ and $x\notin fv(e^p)$, then $\Gamma;\Pi\vdash e^{p}:T$
\end{lemma}
\begin{proof}
	The proof proceeds by induction on the height of the derivation tree for $\Gamma;\Pi\vdash e^{p}:T$.
	\begin{description}
		\item[\runa{Cons}] Here $e^p=c^p$, where
		$$\Gamma,x^{p'}:T';\Pi\vdash c^p : (\emptyset,\emptyset)$$
		and $x\notin fv(c^p)$.
		We can thus conclude that:
		$$\Gamma;\Pi\vdash c^p : (\emptyset,\emptyset)$$

		\item[\runa{Var}] Here $e^p=y^p$, where
		$$\Gamma,x^{p'}:T';\Pi\vdash y^p : T$$
		From our assumption we know that $x\neq y$, since if $x=y$ then we have $x\in fv(y^p)$ which is against our assumption.
		As such we have $x\notin fv(y^p)$, where we can thus conclude that:
		$$\Gamma;\Pi\vdash y^p : T$$
	\end{description}
	Next, follows the induction step:

	\begin{description}
		\item[\runa{Abs}] Here $e^p=[\lambda\;y.e_1^{p_1}]^p$, where
		\begin{figure}[H]
			\setlength\tabcolsep{8pt}
			\begin{tabular}{l}
			\inference[]
				{\Gamma,y^{p''}:T_1;\Pi\vdash  e_1^{p_1}:T_2}
				{\Gamma,x^{p'}:T';\Pi\vdash  [\lambda\;y.e_1^{p_1}]^{p}:T_1\rightarrow T_2}\\[1cm]
			\end{tabular}
		\end{figure}
		and $x\notin fv([\lambda\;y.e_1^{p_1}]^p)$.
		By \cref{def:fv} we know that $x\notin fv(e_1^{p_1})$, we can thus conclude that $\Gamma;\Pi\vdash[\lambda\;y.e_1^{p_1}]^{p}:T_1\rightarrow T_2$.

		\item[\runa{App}] Here $e^p=[e_1^{p_1}\;e_2^{p_2}]^p$, where
		\begin{figure}[H]
			\setlength\tabcolsep{8pt}
			\begin{tabular}{l}
			\inference[]
				{
					\Gamma;\Pi\vdash e_1^{p_1}:T_1\rightarrow T_2 &\\
					\Gamma;\Pi\vdash e_2^{p_2}:T_1
				}
				{\Gamma,x^{p'}:T';\Pi\vdash [e_1^{p_1} \; e_2^{p_2}]^{p}:T_2}\\[1cm]
			\end{tabular}
		\end{figure}
		and $x\notin fv([e_1^{p_1} \; e_2^{p_2}]^p)$.
		By \cref{def:fv}, we then know:
		\begin{itemize}
			\item $x\notin fv(e_1^{p_1})$,
			\item $x\notin fv(e_2^{p_2})$
		\end{itemize}
		Where we can then conclude that $\Gamma;\Pi\vdash [e_1^{p_1} \; e_2^{p_2}]^{p}:T_2$.

		\item[\runa{Let 1}] Here $e^p=[\mbox{let}\;y\;e_1^{p_1}\;e_2^{p_2}]^p$, where
		\begin{figure}[H]
			\setlength\tabcolsep{8pt}
			\begin{tabular}{l}
			\inference[]
				{\Gamma;\Pi\vdash e_1^{p_1}:(\delta,\kappa) &\\
				\Gamma,y^p:(\delta,\kappa\cup\{x\});\Pi\vdash e_2^{p_2}:T_2}
				{\Gamma,x^{p'}:T';\Pi\vdash [\mbox{let}\; y \; e_1^{p_1} \; e_2^{p_2}]^{p}:T_2}
			\end{tabular}
		\end{figure}
		and $x\notin fv([\mbox{let}\; y \; e_1^{p_1} \; e_2^{p_2}]^p)$.
		By \cref{def:fv}, we then know:
		\begin{itemize}
			\item $x\notin fv(e_1^{p_1})$,
			\item $x\notin fv(e_2^{p_2})$
		\end{itemize}
		Where we can then conclude that $\Gamma;\Pi\vdash [\mbox{let}\; y \; e_1^{p_1} \; e_2^{p_2}]^{p}:T_2$.

		\item[\runa{Let 2}] Here $e^p=[\mbox{let}\;y\;e_1^{p_1}\;e_2^{p_2}]^p$, where
		\begin{figure}[H]
			\setlength\tabcolsep{8pt}
			\begin{tabular}{l}
			\inference[]
				{\Gamma;\Pi\vdash e_1^{p_1}:T_1 &\\
				\Gamma,y^p:T_1;\Pi\vdash e_2^{p_2}:T_2}
				{\Gamma,x^{p'}:T';\Pi\vdash [\mbox{let}\; y \; e_1^{p_1} \; e_2^{p_2}]^{p}:T_2}
			\end{tabular}
		\end{figure}
		and $x\notin fv([\mbox{let}\; y \; e_1^{p_1} \; e_2^{p_2}]^p)$.
		By \cref{def:fv}, we then know:
		\begin{itemize}
			\item $x\notin fv(e_1^{p_1})$,
			\item $x\notin fv(e_2^{p_2})$
		\end{itemize}
		Where we can then conclude that $\Gamma;\Pi\vdash [\mbox{let}\; y \; e_1^{p_1} \; e_2^{p_2}]^{p}:T_2$.

		\item[\runa{Let rec}] Here $e^p=[\mbox{let rec}\;f\;e_1^{p_1}\;e_2^{p_2}]^p$, where
		\begin{figure}[H]
			\setlength\tabcolsep{8pt}
			\begin{tabular}{l}
			\inference[]
				{\Gamma;\Pi\vdash e_1^{p_1}:T_1\rightarrow T_2 &\\
				\Gamma,f^p:T_1\rightarrow T_2;\Pi\vdash e_2^{p_2}:T}
				{\Gamma,x^{p'}:T';\Pi\vdash [\mbox{let}\; y \; e_1^{p_1} \; e_2^{p_2}]^{p}:T}
			\end{tabular}
		\end{figure}
		and $x\notin fv([\mbox{let rec}\; f \; e_1^{p_1} \; e_2^{p_2}]^p)$.
		By \cref{def:fv}, we then know:
		\begin{itemize}
			\item $x\notin fv(e_1^{p_1})$,
			\item $x\notin fv(e_2^{p_2})$
		\end{itemize}
		We can then conclude that $\Gamma;\Pi\vdash [\mbox{let}\; f \; e_1^{p_1} \; e_2^{p_2}]^{p}:T$.

		\item[\runa{Case}] Here $e^p=[\mbox{case}\;e_1^{p_1}\;\pi^{p_2}]^p$, where
		\begin{figure}[H]
			\setlength\tabcolsep{8pt}
			\begin{tabular}{l}
			\inference[]
				{\Gamma;\Pi\vdash e_1^{p_1}:(\delta,\kappa) &\\
				\sigma(\Gamma,s_i,p,(\delta,\kappa));\Pi\vdash (s_i \; e^{p_i}):T_i\;\;\;(1\leq i\leq|\vec{\pi}|)}
				{\Gamma,x^{p'}:T';\Pi\vdash [\mbox{case}\;e_1^{p_1}\;\pi^{p_2}]^{p}:T\sqcup(\delta,\kappa)}
			\end{tabular}
		\end{figure}
		Where $T=\bigcup_{i}T_i$.
		and $x\notin fv([\mbox{case}\;e^{p} \vec{\pi}]^p)$.
		By \cref{def:fv}, we then know:
		\begin{itemize}
			\item $x\notin fv(e_1^{p_1})$,
			\item $x\notin fv(\pi^{p_2})$
		\end{itemize}
		We can then conclude that $\Gamma;\Pi\vdash [\mbox{case}\;e_1^{p_1}\;\pi^{p_2}]^{p}:T$.

		\item[\runa{Match}] Here $e^p=[(s\;e_2^{p_2})]^p$, where
		\begin{figure}[H]
			\setlength\tabcolsep{8pt}
			\begin{tabular}{l}
			\inference[]
				{\Gamma;\Pi\vdash e_1^{p_1}:T}
				{\Gamma,x^{p'}:T';\Pi\vdash [(s\;e_1^{p_1})]^{p}:T}
			\end{tabular}
		\end{figure}
		and $x\notin fv([(s\;e_1^{p_1})]^p)$.
		By \cref{def:fv}, we then know:
		\begin{itemize}
			\item $x\notin fv(e_1^{p_1})$,
		\end{itemize}
		We can then conclude that $\Gamma;\Pi\vdash [\mbox{case}\;e_1^{p_1}\;\pi^{p_2}]^{p}:T$.

		\item[\runa{Ref}] Here $e^p=[\mbox{ref}\;e_1^{p_1}]^p$, where
		\begin{figure}[H]
			\setlength\tabcolsep{8pt}
			\begin{tabular}{l}
			\inference[]
				{\Gamma;\Pi\vdash  e_1^{p_1}:T_1}
				{\Gamma,\nu y^{p'}:T_1,x^{p'}:T;\Pi\vdash [\mbox{ref}\;e_1^{p_1}]^{p}:(\emptyset,\kappa)}
			\end{tabular}
		\end{figure}
		Where $\kappa=\{\nu x\}$.
		and $x\notin fv([\mbox{ref}\;e_1^{p_1}]^p)$.
		By \cref{def:fv}, we then know:
		\begin{itemize}
			\item $x\notin fv(e_1^{p_1})$,
		\end{itemize}
		We can then conclude that $\Gamma;\Pi\vdash [\mbox{ref}\;e_1^{p_1}]^{p}:T$.

		\item[\runa{Ref read}] Here $e^p=[!e_1^{p_1}]^p$, where
		\begin{figure}[H]
			\setlength\tabcolsep{8pt}
			\begin{tabular}{l}
			\inference[]
				{\Gamma;\Pi\vdash  e_1^{p_1}:(\delta,\kappa)}
				{\Gamma,x^{p'}:T;\Pi\vdash [!e_1^{p_1}]^{p}:T\sqcup(\delta\cup\delta',\emptyset)}
			\end{tabular}
		\end{figure}
		Where $\delta'=\{\nu x^{p'}\mid\nu x\in\kappa\}$,\\ $T=\bigcup_{\nu x\in\kappa}(\Gamma(\nu x^{p''}))$ and $p''=\Lambda(x,p')$.
		and $x\notin fv([!e_1^{p_1}]^p)$.
		By \cref{def:fv}, we then know:
		\begin{itemize}
			\item $x\notin fv(e_1^{p_1})$,
		\end{itemize}
		We can then conclude that $\Gamma;\Pi\vdash [!e_1^{p_1}]^{p}:T$.

		\item[\runa{Ref write}] Here $e^p=[e_1^{p_1}\;:=\;e_2^{p_2}]^p$, where
		\begin{figure}[H]
			\setlength\tabcolsep{8pt}
			\begin{tabular}{l}
			\inference[]
				{\Gamma;\Pi\vdash  e_1^{p_1}:(\delta,\kappa)&\\
				\Gamma;\Pi\vdash  e_2^{p_2}:T_1}
				{\Gamma',x^{p'}:T;\Pi\vdash [e_1^{p_1}\;:=\;e_2^{p_2}]^{p}:(\delta,\kappa)}
			\end{tabular}
		\end{figure}
		Where $\Gamma'=\forall \nu x\in\kappa.\Gamma,\nu x^{p}:T_1$.
		and $x\notin fv([e_1^{p_1}\;:=\;e_2^{p_2}]^p)$.
		By \cref{def:fv}, we then know:
		\begin{itemize}
			\item $x\notin fv(e_1^{p_1})$,
			\item $x\notin fv(e_2^{p_2})$,
		\end{itemize}
		We can then conclude that $\Gamma';\Pi\vdash [e_1^{p_1}\;:=\;e_2^{p_2}]^{p}:T$.
	\end{description}
\end{proof}

\subsection{Proof}
\begin{theorem}[Soundness of type system]
	Suppose $e^{p'}$ is an occurrence where
	\begin{itemize}
		\item $env\vdash\left\langle e^{p'},sto,w,p\right\rangle\rightarrow\left\langle v,sto',w',(L,V),p''\right\rangle$,
		\item $\Gamma;\Pi\vdash e^{p'} : T$
		\item $(env,sto,w)\models(\Gamma,\Pi)$
		\item $\Gamma;\Pi\vdash env$
	\end{itemize}
	Then we have that:
	\begin{itemize}
		\item $\Gamma;\Pi\vdash v:T$
		\item $w'\models\Pi$
		\item $(env,sto',w')\models(\Gamma,\Pi)$
	\end{itemize}
\end{theorem}
\begin{proof}
	The proof proceeds by induction on the height for the derivation tree for $env\vdash\left\langle e^{p'},sto,w,p'\right\rangle\rightarrow\left\langle v,sto,w,(L,V),p\right\rangle$.
	In the base case we have the \runa{Cons} and \runa{Var} rule:
	\begin{description}
		\item[\runa{Cons}] Here $e^{p'}=c^{p'}$, where
			\begin{itemize}
				\item $env\vdash\left\langle c^{p'},sto,w,p\right\rangle\rightarrow\left\langle c,sto,w,(\emptyset,\emptyset),p'\right\rangle$
				\item $\Gamma,\Pi\vdash c^{p'}:(\emptyset,\emptyset)$, and 
				\item $(env,sto,w)\models\Gamma$
				\item $w\models\Pi$
			\end{itemize}
			Since $env$, $sto$, $w$, and $\Gamma$ remains unchanged when evaluating $e^{p'}$, the value is a constant $c$, and the both the dependency pari and type are $(\emptyset,\emptyset)$, we can thus conclude that:
			\begin{itemize}
				\item $\Gamma,\Pi\vdash c : (\emptyset,\emptyset)$
				\item $w\models\Pi$
				\item $(env,sto,w)\models\Gamma$
				\item $(env,w,(\emptyset,\emptyset))\models(\emptyset,\emptyset)$
			\end{itemize}

		\item[\runa{Var}] Here $e^{p'}=x^{p'}$, where
			\begin{itemize}
				\item $env\vdash\left\langle x^{p'},sto,w,p\right\rangle\rightarrow\left\langle v,sto,w,(\emptyset,\emptyset),p'\right\rangle$
				\item $\Gamma,\Pi\vdash x^{p'}:T\sqcup (\{x^p\},\emptyset)$, and 
				\item $(env,sto,w)\models\Gamma$
				\item $w\models\Pi$
			\end{itemize}
			Since $env$, $sto$, $w$, and $\Gamma$ remains unchanged when evaluating $e^{p'}$.
			Since we know $(env,sto,w)\models\Gamma$ and $w\models\Pi$, and that both $inf_p$ and $\Lambda$ finds the greatest lower bound program point for and occurrence $x^p$, i.e., the program point $p'$ that is closest, or the same, to $p$ where $x^{p'}$ is bound in $w$ and $\Gamma$ respectively.
			We can thus get:
			\begin{itemize}
				\item $\Gamma,\Pi\vdash v:T$
				\item $w\models\Pi$
				\item $(env,sto,w)\models\Gamma$
				\item $(env,w,(L,V))\models T$
			\end{itemize}
	\end{description}

	Next, follows the induction step:
	\begin{description}
		\item[\runa{Abs}] Here $e^{p'}=\left[\lambda\;x.e_1^{p_1}\right]^{p'}$, where
			\begin{figure}[H]
				\setlength\tabcolsep{8pt}
				\begin{tabular}{l}
					\InfName{Abs}\\[0.2cm]
						\inference[]{}
						{env\vdash\left\langle \left[\lambda\;x.e_1^{p_1}\right]^{p'},sto,w,p\right\rangle\rightarrow\left\langle v,sto,w,(\emptyset,\emptyset),p'\right\rangle}
				\end{tabular}
			\end{figure}
			Where $v=\left\langle x,e_1^{p_1},env\right\rangle$.
			And from our assumptions, we have that:
			\begin{itemize}
				\item $\Gamma,\Pi\vdash \left[\lambda\;x.e_1^{p_1}\right]^{p'}:T$, and 
				\item $(env,sto,w)\models(\Gamma,\Pi)$
				\item $\Gamma;\Pi\vdash env$
			\end{itemize}
			Where we know that, to type $e^{p'}$ we need to use the \runa{T-Abs} rule, where we then have:
			\begin{figure}[H]
				\setlength\tabcolsep{8pt}
				\begin{tabular}{l}
					\InfName{T-Abs}\\[0.2cm]
						\inference[]
						{\Gamma,x^{p_0}:T_1;\Pi\vdash  e_1^{p_1}:T_2}
						{\Gamma;\Pi\vdash  \left[\lambda\;x.e_1^{p_1}\right]^{p'}:T_1\rightarrow T_2}\\
				\end{tabular}
			\end{figure}
			Where $p'\sqsubseteq_\Pi p_0\wedge p_0\sqsubseteq_\Pi p_1$.
			We must show that \cat{(1)} $\Gamma,\Pi\vdash v:T$, \cat{(2)} $(env,sto',w')\models(\Gamma,\Pi)$, and \cat{(3)} $(env,w',(L,V))\models(\Gamma,T)$.

			\begin{description}
				\item[(1)] Since the value is a closure, we must type it with the \runa{Closure} rule:
					\begin{figure}[H]
						\setlength\tabcolsep{8pt}
						\begin{tabular}{l}
							\runa{Closure}\\[0.4cm]
								\inference[]
								{
									\Gamma;\Pi\vdash env \\
									\Gamma,x^{p_0}:T_1;\Pi\vdash e_1^{p_1}:T_2
								}
								{\Gamma;\Pi\vdash \left\langle x^{p_0}, e_1^{p_1}, env \right\rangle^{p'}:T_1\rightarrow T_2}
						\end{tabular}
					\end{figure}
					Where we know $\Gamma;\Pi\vdash env$ from our assumption and $\Gamma,x^{p_0}:T_1;\Pi\vdash e_1^{p_1}:T_2$ from the premise of \runa{Abs} type rule.
				\item[(2)] Is immediate since $sto'=sto$ and $w'=w$.
				\item[(3)] Is immediate by \cref{def:TAgree}, since we know that the value $v$ is a closure and the dependency pair is $(\emptyset,\emptyset)$.
			\end{description}

		\item[\runa{App}] Here $e^{p'}=\left[e_1^{p'}\;e_2^{p''}\right]^{p'}$, where
			\begin{figure}[H]
				\setlength\tabcolsep{8pt}
				\begin{tabular}{l}
					\InfName{App}\\[0.2cm]
					\inference[]
						{env \vdash \left\langle e_1^{p_1},sto,w,p \right\rangle \rightarrow \left\langle v_1,w_1, sto_1,(L_1,V_1),p_1 \right\rangle &\\
						env \vdash \left\langle e_2^{p_2},w_1, sto_1,p_1 \right\rangle \rightarrow \left\langle v_2,w_2, sto_2,(L_2,V_2),p_2 \right\rangle &\\
						env'[x\mapsto v_2] \vdash \left\langle e_3^{p_3}, sto_2, w_2', p_2 \right\rangle \rightarrow \left\langle v,w', sto',(L_3,V_3),p_3 \right\rangle}
						{env\vdash \left\langle \left[e_1^{p_1}\;e_2^{p_2}\right]^{p'},sto,w,p \right\rangle \rightarrow \left\langle v,sto',w',(L,V),p' \right\rangle}
				\end{tabular}
			\end{figure}
			Where $v_1=\left\langle x,e_3^{p_3},env'\right\rangle$, $w_2'=w_2[x^{p_2}\mapsto (L_2,V_2)]$, and $(L,V)=(L_1\cup L_3,V_1\cup V_3)$
			And from our assumptions, we have that:
			\begin{itemize}
				\item $\Gamma;\Pi\vdash \left[e_1^{p_1}\;e_2^{p_2}\right]^{p'}:T$,
				\item $w\models\Pi$,
				\item $(env,sto,w)\models\Gamma$,
				\item $\Gamma;\Pi\vdash env$
			\end{itemize}
			Where we know that, to type $e^{p'}$ we need to use the \runa{App} type rule, where we then have:
			\begin{figure}[H]
				\setlength\tabcolsep{8pt}
				\begin{tabular}{l}
					\runa{App}\\[0.2cm]
						\inference[]
						{
							\Gamma;\Pi\vdash e_1^{p_1}:T_1\rightarrow T &\\
							\Gamma;\Pi\vdash e_2^{p_2}:T_1
						}
						{\Gamma;\Pi\vdash [e_1^{p_1} \; e_2^{p_2}]^{p}:T}\\
				\end{tabular}
			\end{figure}
			We must show that \cat{(1)} $\Gamma,\Pi\vdash v:T$, \cat{(2)} $w'\models\Pi$, \cat{(3)} $(env,sto',w')\models\Gamma$, and \cat{(4)} $(env,w',(L,V))\models(\Gamma,T)$.

			Due to the first premise: since $w\models\Pi$, $(env,sto,w)\models\Gamma$, $\Gamma;\Pi\models env$, and due to the first premise from the type rule \runa{App}, we get from our induction hypothesis:
			\begin{itemize}
				\item $\Gamma;\Pi\vdash e_1^{p_1}:T_1\rightarrow T$,
				\item $w_1\models\Pi$,
				\item $(env,sto_1,w_1)\models\Gamma$,
				\item $(env,w_1,(L_1,V_1))\models(\Gamma,T_1\rightarrow T)$
			\end{itemize}
			And due to the second premise, from the first premise, and the second premise of the type rule for \runa{App} we get from our induction hypothesis:
			\begin{itemize}
				\item $\Gamma;\Pi\vdash e_2^{p_2}:T_1$,
				\item $w_2\models\Pi$,
				\item $(env,sto_2,w_2)\models\Gamma$,
				\item $(env,w_2,(L_2,V_2))\models(\Gamma,T_1)$
			\end{itemize}

			For the third premise, we need to show that there is a $\Gamma_1$ such that: \cat{a)} $\Gamma_1;\Pi\vdash e_3^{p_3}:T$, \cat{b)} $(env'[x\mapsto v_2],sto_2,w_2')\models\Gamma_1$, and \cat{c)} $w_2'\models\Pi$.
			\begin{description}
				\item[a)] Since $v_1$ is a closure where $\left\langle x^{p_0},e_3^{p_3},env'\right\rangle$ we must have concluded this with the \runa{Closure} rule, such that:
					\begin{figure}[H]
						\setlength\tabcolsep{8pt}
						\begin{tabular}{l}
							\runa{Closure}\\[0.4cm]
								\inference[]
								{
									\Gamma;\Pi\vdash env' \\
									\Gamma,x^{p_0}:T_1;\Pi\vdash e_3^{p_3}:T
								}
								{\Gamma;\Pi\vdash \left\langle x^{p_0}, e_3^{p_3}, env' \right\rangle:T_1\rightarrow T}
						\end{tabular}
					\end{figure}
					From our assumption, we know that $\Gamma;\Pi\vdash env$, and $env=env'[env'']$, we then know that $\Gamma;\Pi\vdash env'$.
					We then need to show for the second premise.
					Here, we can see that $\Gamma_1$ must be $\Gamma,x^{p_0}:T_1$, as such we need to show that there is a $x^{p_0}:T_1$.
					From the second premise of \runa{App} we know that $\Gamma;\Pi\vdash v_2:T_1$, and we know that $w_2'=w_2[x^{p_2}\mapsto(L_2,V_2)]$ we the know that $p_0=p_2$, and as such we then get a $\Gamma_1$ such that $\Gamma,x^{p_2}:T_1$.
				\item[b)] We know that $(env,sto_2,w_2)\models\Gamma$ from the second premise.
					We then need to show that we can get $env'$, $w_2'$, and $\Gamma_1$.
					From \cat{a)} we know that $env=env'[env'']$ as such we know that $(env',sto_2,w_2)\models\Gamma$ holds.
					We also know that $env'[x\mapsto v_2]$, $w_2'=w_2[x^{p_2}\mapsto(L_2,V_2)]$ and $\Gamma_1=\Gamma,x^{p_2}:T_1$.
					Since we have $\Gamma;\Pi\vdash v_2:T_1$ we then know, from \cref{def:EnvAgree}, that $(env'[x\mapsto v_2],sto_2,w_2')\models\Gamma_1$ also holds.
				\item[c)] Since we know that $w_2\models\Pi$, we only need to show for the extension $x^{p_2}\mapsto(L_2,V_2)$.
					From \cref{def:popular}, $w_2\models\Pi$, and that all program points in $e_2^{p_2}$ must be earlier program points, based on the semantics.
					We then know that $w_2'\models\Pi$ must hold.
			\end{description}
			From \cat{a)}, \cat{b)}, \cat{c)}, and the third premise of \runa{App} rule, we can get from our induction hypothesis:
			\begin{itemize}
				\item $\Gamma_1;\Pi\vdash v:T$,
				\item $w'\models\Pi$,
				\item $(env'[x\mapsto v_2],sto',w')\models\Gamma$,
				\item $(env'[x\mapsto v_2],w_3,(L_3,V_3))\models(\Gamma,T_2)$
			\end{itemize}
			We can thus conclude the following:
			\begin{description}
				\item[(1)] $\Gamma,\Pi\vdash v:T$
				\item[(2)] $w'\models\Pi$
			\end{description}

			We then need to show that we can get \cat{(3)} and \cat{(4)}.
			From \cref{lemma:His} we know the follwing:
			\begin{itemize}
				\item if $x^{p''}\in dom{w_1}\backslash dom(w) then x^{p''}\notin fv{e_1^{p_1}}$
				\item if $x^{p''}\in dom{w_2}\backslash dom(w_1) then x^{p''}\notin fv{e_2^{p_2}}$
				\item if $x^{p''}\in dom{w'}\backslash dom(w_2') then x^{p''}\notin fv{e_3^{p_3}}$
			\end{itemize}
			\todo[inline]{I still need the rest to conclude the two following parts}
			\begin{description}
				\item[(3)] $(env,sto',w')\models\Gamma$
				\item[(4)] $(env,w_3,(L_3,V_3))\models(\Gamma,T_2)$
			\end{description}

		\item[\runa{Let}] Here $e^{p'}=\left[\mbox{let}\;x\;e_1^{p_1}\;e_2^{p_2}\right]^{p'}$, where
			\begin{itemize}
				\item $\Gamma,\Pi\vdash \left[\mbox{let}\;x\;e_1^{p_1}\;e_2^{p_2}\right]^{p'}:T,$, and 
				\item $env\vdash\left\langle \left[\mbox{let}\;x\;e_1^{p_1}\;e_2^{p_2}\right]^{p'},sto,w,p\right\rangle\rightarrow\left\langle v,sto',w',(L,V),p'\right\rangle$
			\end{itemize}
			and $(env,sto,w)\models\Gamma$.
			We must show that:
			\begin{itemize}
				\item $\Gamma,\Pi\vdash v:T$
				\item $(env,sto',w')\models\Gamma$
				\item $(env,w',(L,V))\models T$
			\end{itemize}
			To do this, we need to analyse the premises, where in the semantics we have:
			\begin{figure}[H]
				\setlength\tabcolsep{8pt}
				\begin{tabular}{l}
					\InfName{Let}\\[0.2cm]
						\inference[]
						{env\vdash \left\langle e_1^{p_1},sto,w,p \right\rangle \rightarrow \left\langle v_1,sto'',w'',(L',V'),p_1 \right\rangle &\\
						env[x\mapsto v_1]\vdash \left\langle e_2^{p_2},sto'',w_3,p_1 \right\rangle \rightarrow \left\langle v,sto',w',(L,V),p_2 \right\rangle}
						{env\vdash \left\langle \left[\mbox{let}\;x\;e_1^{p_1}\;e_2^{p_2}\right]^{p'},sto,w,p \right\rangle \rightarrow \left\langle v,sto',w',(L,V),p' \right\rangle}\\
						Where $w_3=w''[x^{p_1}\mapsto(L',V')]$\\
				\end{tabular}
			\end{figure}

			In the type system, we have to let rules, \runa{Let-1} and \runa{Let-2}, which differs in the typing of the first premise, namely when the value is a location.
			Thus, \runa{Let-1} is for when $v_1=\loc$ which corresponds when the type is $(\delta,\kappa)$ and $\kappa\neq\emptyset$, and \runa{Let-2} for other cases.

			When $v_1=\loc$, we have the following type rule
			\begin{figure}[H]
				\setlength\tabcolsep{8pt}
				\begin{tabular}{l}
					\runa{Let-1}\\[0.2cm]
						\inference[]
							{\Gamma;\Pi\vdash e_1^{p_1}:(\delta,\kappa) &\\
							\Gamma,x^{p_1}:(\delta,\kappa\cup\{x\});\Pi\vdash e_2^{p_2}:T}
							{\Gamma;\Pi\vdash [\mbox{let}\; x \; e_1^{p_1} \; e_2^{p_2}]^{p'}:T}\\[0.3cm]
						Where $\kappa\neq\emptyset$\\
				\end{tabular}
			\end{figure}
			For the first premise we can get $(env,sto'',w'')\models\Gamma$, as the dependency function is only extended.
			If we know $(env,sto'',(L',V'))\models(\delta,\kappa)$, we can then then get $\Gamma;\Pi\vdash\loc:(\delta,\kappa)$.
			Since the type is $(\delta,\kappa)$, we know that the type is either from \runa{Var}, \runa{Ref}, or \runa{Ref read} in $e_1^{p_1}$.

			In the second premise, we have $(env[x\mapsto v_1],sto'',w_3)\models\Gamma$, since $x^{p_1}\notin dom(\Gamma)$.
			We can then also get $(env[x\mapsto v_1],sto'',w_3)\models x^{p_1}:(\delta,\kappa\cup\{x\})$, since $(env,w_3,(L',V'))\models(\delta,\kappa)$, and $x$ is and alias $\loc$.
			
			Lastly, we can the get:
			\begin{itemize}
				\item $\Gamma,\Pi\vdash v:T$,
				\item $(env,sto',w')\models\Gamma$, and
				\item $(env,w',(L,V))\models T$
			\end{itemize}
			And the conclusion follows immediately.
			\bigskip

			When $v_1\neq\loc$, we have the following type rule:
			\begin{figure}[H]
				\setlength\tabcolsep{8pt}
				\begin{tabular}{l}
					\runa{Let-2}\\[0.2cm]
						\inference[]
							{\Gamma;\Pi\vdash e_1^{p_1}:T_1 &\\
							\Gamma,x^{p_1}:T_1;\Pi\vdash e_2^{p_2}:T}
							{\Gamma;\Pi\vdash [\mbox{let}\; x \; e_1^{p_1} \; e_2^{p_2}]^{p'}:T}\\
				\end{tabular}
			\end{figure}
			The proof follows similarly for the case where $v_1=\loc$.


		\item[\runa{Let rec}] Here $e^p=[\mbox{let rec}\;f\;e_1^{p'}\;e_2^{p''}]^p$

		\item[\runa{Case}] Here $e^p=[\mbox{case}\;e_1^{p_1}\;\pi^{p_2}]^p$
			\begin{figure}[H]
				\setlength\tabcolsep{8pt}
				\begin{tabular}{l}
					\runa{Case}\\[0.2cm]
						\inference[]
						{
							env \vdash \left\langle e_1^{p_1},sto,w,p \right\rangle \rightarrow \left\langle v_1,sto_1,w_1,(L_1,V_1),p_1 \right\rangle &\\
						env \vdash \left\langle (v_1,(L_1,V_1),\pi^{p_2}),sto_1,w_1,p_1 \right\rangle \rightarrow \left\langle v,sto',w',(L_2,V_2),p_2 \right\rangle
						}
						{env\vdash \left\langle \left[\mbox{case}\;e_1^{p_1}\;\pi^{p_2}\right]^{p},sto,w,p \right\rangle \rightarrow \left\langle v,sto',w',(L,V),p\right\rangle}
				\end{tabular}
			\end{figure}
			Where $(L,V)=(L_1\cup L_2,V_1\cup V_2)$
			And
			\begin{itemize}
				\item $\Gamma;\Pi\vdash [\mbox{case}\;e_1^{p_1}\;\pi^{p_2}]^p:T$
				\item $(w,env,sto)\models\Gamma$
			\end{itemize}
			We then need to show: \cat{(1)} $\Gamma;\Pi\vdash v:T$, \cat{(2)} $(w',env,sto)\models\Gamma$, and \cat{(3)} $(w',env,(L,V))\models T$.

			The type rule for \runa{Case} is as follows:
			\begin{figure}[H]
				\setlength\tabcolsep{8pt}
				\begin{tabular}{l}
					\runa{Case}\\[0.2cm]
						\inference[]
							{\Gamma;\Pi\vdash e^{p}:(\delta,\kappa) &\\
							\sigma(\Gamma,s_i,p,T');\Pi\vdash (s_i \; e^{p_i}):T_i\;\;\;(1\leq i\leq|\vec{\pi}|)}
							{\Gamma;\Pi\vdash [\mbox{case}\;e^{p} \vec{\pi}]^{p'}:T\sqcup(\delta,\kappa)}
				\end{tabular}
			\end{figure}
			Where $T=\bigcup_{i}T_i$.

			\begin{description}
				\item[(1)] By our induction hypothesis we get:
					\begin{itemize}
						\item $\Gamma;\Pi\vdash v_1:(\delta,\kappa)$
					\end{itemize}
						For the second premise, we know that the evaluation step to get $v$ must come from either \runa{Match 1} or \runa{Match 2}, as such there must be a $(s_i\;e_i^{p_i})\in\pi$ where the evaluation of $e_i^{p_i}$ results in $v$, for some $1\leq i\leq\pi$.
						As such, from the induction hypothesis, we know that:
						$$\Gamma;\Pi\vdash v:T_i$$
						Since the type for the \runa{Case} rule is an overapproximation, and $T_i\subseteq T$, we then get:
						$$\Gamma;\Pi\vdash v:T$$
				\item[(2)] From our induction hypothesis we have:
					\begin{itemize}
						\item $(w_1,env,sto)\models\Gamma$
						\item $(w_2,env,sto)\models\Gamma$
					\end{itemize}
					As such we can conclude that $(w',env,sto)\models\Gamma$ holds.
				\item[(3)] From our induction hypothesis we have:
					\begin{itemize}
						\item $(w',env,(L_1,V_1))\models (\delta,\kappa)$
						\item $(w',env,(L_2,V_2))\models T_i$
					\end{itemize}
					Where $T_i$ is for an $1\leq i\leq\pi$, where $(s_i e_i^{p_i})$ evaluates to $\left\langle v,sto',w',(L_2,V_2),p_2\right\rangle$.
					As such, we also have $$(w',env,(L,V))\models T\sqcup(\delta,\kappa)$$.
			\end{description}

		\item[\runa{Match 1}] Here $e^p=$
		\item[\runa{Match 2}] Here $e^p=$
		\item[\runa{Match $\perp$}] Here $e^p=$

		\item[\runa{Ref}] Here $e^p=[\mbox{ref}\;e_1^{p_1}]^p$ where:
			\begin{figure}[H]
				\setlength\tabcolsep{8pt}
				\begin{tabular}{l}
					\runa{Ref}\\[0.2cm]
						\inference[]
						{env \vdash \left\langle e_1^{p_1},sto,w,p \right\rangle \rightarrow \left\langle v,sto',w',(L,V),p_1 \right\rangle}
						{env\vdash \left\langle \left[\mbox{ref}\;e_1^{p_1}\right]^{p'},sto,w,p \right\rangle \rightarrow \left\langle \loc,sto'',w'',(\emptyset,\emptyset),p' \right\rangle}
				\end{tabular}
			\end{figure}
			Where $\loc=next$, $sto''=sto'[next\mapsto new(\loc),\loc\mapsto v]$ and $w''=w'[\loc^{p''}\mapsto (L,V)]$.
			And
			\begin{figure}[H]
				\setlength\tabcolsep{8pt}
				\begin{tabular}{l}
					\runa{Ref}\\[0.2cm]
						\inference[]
							{\Gamma;\Pi\vdash  e_1^{p_1}:T}
							{\Gamma,\nu x^{p'}:T;\Pi\vdash [\mbox{ref}\;e_1^{p_1}]^{p'}:(\emptyset,\kappa)}
				\end{tabular}
			\end{figure}
			Where $\kappa=\{\nu x\}$.
			And $(env,sto,w)\models\Gamma$.
			We abriviate the extension to $\Gamma$, such that $\Gamma'=\Gamma,\nu x^{p'}:T$.
			We then need to show: \cat{(1)} $\Gamma;\Pi\vdash\loc:(\emptyset,\kappa)$, \cat{(2)} $(env,sto'',w'')\models\Gamma'$, and \cat{(3)} $(env,w'',(L,V))\models (\emptyset,\kappa)$.

			\begin{description}
				\item[(1)] From the type rule \runa{Location}, we know that the type must be a base type, where $\kappa\neq\emptyset$.
					From the type rule \runa{Ref} we know that the type is $(\emptyset,\kappa)$, as such we get that 
					$$\Gamma;\Pi\vdash\loc:(\emptyset,\kappa)$$
				\item[(2)] From our induction hypothesis, we get: $(w',env,sto')\models\Gamma$.
					We then need to show for $w''$ and $\Gamma'$, where we know that $w''=w'[\loc^{p'}\mapsto(L,V)]$, $\Gamma'=\Gamma[\nu x^{p'}\mapsto T]$, and from the induction hypothesis we also know that $\Gamma;\Pi\vdash v:T$.
					We also know that $sto''=sto'[next\mapsto new(\loc),\loc\mapsto v]$, where the first extension is updating $next$ to a new free location and updating $\loc$ to map to $v$.
					As such, by \cref{def:EnvAgree} we get:
					$$(env, sto'',w'')\models\Gamma'$$
				\item[(3)] By \cref{def:TAgree}, we need to show that \cat{(3-1)} $(env,(\emptyset,\emptyset))\models\emptyset$ and \cat{(3-2)} $(env,w'')\models\kappa$.
					\begin{description}
						\item[(3-1)] Holds by \cref{def:DepAgree}.
						\item[(3-2)] Since $\loc$ is an unbound location in $env$, we then know that $\not\exists x\in dom(env).env(x)=\loc$.
							We then know that there must exists a $\kappa_i^0\in\kappa$ such that $\kappa_i^0\in\cat{IVar}$.
					\end{description}
			\end{description}

		\item[\runa{Ref read}] Here $e^p=$

		\item[\runa{Ref write}] Here $e^p=$
	\end{description}
\end{proof}
\end{document}
