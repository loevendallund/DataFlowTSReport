\documentclass[../../master.tex]{subfiles}
\begin{document}
\section{Notes}

\subsection{Basis}
%\begin{definition}[Type Base for dependencies]
%	The type base $\delta^0$
%\end{definition}

\begin{definition}[Type Base for aliasing]
	The type base $\kappa^0$ is a partition of $\cat{Var}\cup\cat{IVar}$, such that:
	$$\cat{Var}\subseteq\bigcup_{i\in I}(\kappa_i^0)$$
	and that $\kappa_i^0\neq\kappa_j^0$ for all $i\neq j$
\end{definition}

\subsection{Type environment}
\begin{definition}[Type Environment]
	The type environment $\Gamma$ is a partial function $\Gamma:Id^P\rightharpoonup TYPES$
\end{definition}

\begin{definition}[Approximated order of program points]
	The approximated order of program points $\Pi$, where:
	\begin{itemize}
		\item  $p\sqsubseteq p'\in\Pi$, if $p$ is an earlier or the same program point as $p'$
		\item The order of program points is transitive, such that if $p\sqsubseteq p'\in\Pi$ and $p'\sqsubseteq p''\in\Pi$ then $p\sqsubseteq p''\in\Pi$.
	\end{itemize}
\end{definition}

\subsection{Agreement}
\begin{definition}[Dependency environment agreement]
	Let $w$ be a dependency function, $env$ be an environment, and $\Gamma$ be a type environment.
	We say that:
	$$(w,env)\models\Gamma$$
	if 
	\begin{itemize}
		\item $\forall x^p\in dom(w).x^p\in dom(\Gamma)\Rightarrow w(x^p)=(L,V)\wedge\Gamma(x^p)=T.(w,env,(L,V))\models T$
		\item $\forall \loc^p \in dom(w).\exists\nu x^{p'}\in dom(\Gamma)\Rightarrow w(\loc^p)=(L,V)\wedge\Gamma(\nu x^{p'})=T.(w,env,(L,V))\models T$
	\end{itemize}
\end{definition}

\begin{definition}[Type agreement]
	Let $w$ be a dependency function, $env$ be an environment, $(L,V)$ be a dependency pair, and $T$ be a type.
	We say that:
	$$(w,env,(L,V))\models T$$
	if
	\begin{itemize}
		\item $T=T_1\rightarrow T_2$ then:
		\begin{itemize}
			\item $(w,env,(L,V))\models T_1\Rightarrow(w,env,(L,V))\models T_2$
		\end{itemize}
		\item $T=(\delta,\kappa)$ then:
		\begin{itemize}
			\item $(env,(L,V))\models\delta$
			\item $(w,env)\models\kappa$
		\end{itemize}
	\end{itemize}
\end{definition}

\begin{definition}[Dependency agreement]
	Let $env$ be an environment, $(L,V)$ be a dependency pair, and $\delta$ be a set of variables.
	We say that:
	$$(env,(L,V))\models\delta$$
	if
	\begin{itemize}
		\item $V\subseteq\delta$,
		\item For all $\loc^p\in L$ where $\exists x\in dom(env).env(x)=\loc$, we then have $\{x\in dom(env)\mid env(x)=\loc\}\subseteq \kappa_i^0$ for a $\kappa_i^0\in\delta$
		\item For all $\loc^p\in L$ where $\forall x\in dom(env).env(x)\neq \loc$, we then have a $\kappa_i^0\in\delta$, where for every $u\in\kappa_i^0$ is an internal variable, i.e., $u\in\cat{IVar}$
	\end{itemize}
\end{definition}

\begin{definition}[Alias agreement]
	Let $env$ be an environment, and $\kappa$ be an alias set.
	We say that:
	$$(env,(L,V))\models\kappa$$
	if
	\begin{itemize}
		\item For all $\loc^p\in dom(w)$ where $\exists x\in dom(env).env(x)=\loc$, we then have $\{x\in dom(env)\mid env(x)=\loc\}\subseteq \kappa_i^0$ for a $\kappa_i^0\in\kappa$
		\item For all $\loc^p\in dom(w)$ where $\forall x\in dom(env).env(x)\neq \loc$, we then have a $\kappa_i^0\in\kappa$, where for every $u\in\kappa_i^0$ is an internal variable, i.e., $u\in\cat{IVar}$
	\end{itemize}
\end{definition}

\subsection{Judgement}
\begin{definition}[Environment judgement]
	Let $env$ be an environment, $\Gamma$ be a type environment, and $\Pi$ be the approximated order of program points.
	We say that:
	$$\Gamma;\Pi\vdash env$$
	if 
	\begin{itemize}
		\item For all $x\in dom(env)$ and for all $x^p\in dom(\Gamma).\Gamma(x^p)=T_x$ then 
			$$\Gamma,\Pi\vdash env(x):T_x$$
	\end{itemize}
\end{definition}

\subsection{Type rules for values}
\begin{figure}[H]
	\setlength\tabcolsep{8pt}
	\begin{tabular}{l}
		\hline\\
		\runa{Constant}\\[0.4cm]
			\inference[]{}
				{\Gamma;\Pi\vdash  c:(\emptyset, \emptyset)}\\[1cm]

		\runa{Location}\\[0.4cm]
			\inference[]{}
				{\Gamma;\Pi\vdash  \loc:(\delta, \kappa)}\\[1cm]

		\runa{Closure}\\[0.4cm]
			\inference[]
				{
					\Gamma;\Pi\vdash env \\
					\Gamma,x^{p}:T_1;\Pi\vdash e^{p'}:T_2
				}
				{\Gamma;\Pi\vdash \left\langle x^{p}, e^{p'}, env \right\rangle^{p''}:T_1\rightarrow T_2}\\[1cm]

		\runa{Recursive closure}\\[0.4cm]
			\inference[]
				{
					\Gamma;\Pi\vdash env \\
					\Gamma,x^{p}:T_1,f^{p'}:T_1\rightarrow T_2;\Pi\vdash e^{p''}:T_2
				}
				{\Gamma;\Pi\vdash \left\langle x^{p}, f^{p'}, e^{p''}, env \right\rangle^{p_3}:T_1\rightarrow T_2}\\[1cm]

		\runa{Unit}\\[0.4cm]
			\inference[]{}
			{\Gamma;\Pi\vdash  ():(\delta, \kappa)}\\[0.5cm]
		\hline
	\end{tabular}
	\caption{Type rules for values}
	\label{fig:ValTypeRules}
\end{figure}

\subsection{Proof}
\begin{definition}[Substitutions]
	Suppose $e^p$ is an occurrence, then we say that
	$$e^p[x\mapsto v]$$
	is a substitution of $v$ for all occurrences of the variable $x$ on the structure of $e^p$ defined by:
	\begin{align*}
		c^p[x\mapsto v]&=c^p\\
		x^p[x\mapsto v]&=	\left\{\begin{matrix}
									y & \mbox{if}\;y \neq x\\ 
									v & \mbox{if}\;y = x\\ 
								\end{matrix}\right.\\
		[\lambda\;y\;e_1^{p'}]^p[x\mapsto v]&=
			\left\{\begin{matrix}
				[\lambda\;y\;e_1^{p'}]^p & \mbox{if}\;y = x\\ 
				[\lambda\;y\;e_1^{p'}[x\mapsto v]]^p & \mbox{if}\;y \neq x\\
			\end{matrix}\right.\\
		[e_1^{p'}\;e_2^{p''}]^p[x\mapsto v]&=[e_1^{p'}[x\mapsto v]\;e_2^{p''}[x\mapsto v]]^p\\
		[\mbox{let}\;y\;e_1^{p'}\;e_2^{p''}]^p[x\mapsto v]&=	
			\left\{\begin{matrix}
				[\mbox{let}\;y\;e_1^{p'}\;e_2^{p''}]^p & \mbox{if}\;y = x\\ 
				[\mbox{let}\;y\;e_1^{p'}[x\mapsto v]\;e_2^{p''}[x\mapsto v]]^p & \mbox{if}\;y \neq x\\ 
			\end{matrix}\right.\\
		[\mbox{let rec}\;f\;e_1^{p'}\;e_2^{p''}]^p[x\mapsto v]&=
			\left\{\begin{matrix}
				[\mbox{let rec}\;f\;e_1^{p'}\;e_2^{p''}]^p & \mbox{if}\;f = x\\ 
				[\mbox{let rec}\;f\;e_1^{p'}[x\mapsto v]\;e_2^{p''}[x\mapsto v]]^p & \mbox{if}\;f \neq x\\ 
			\end{matrix}\right.\\
		[\mbox{case}\;e^{p'}\;\pi^{p''}]^p[x\mapsto v]&=[\mbox{case}\;(e^{p'}[x\mapsto v])\;(\pi[x\mapsto v])]^p\\
		[\mbox{ref}\;e^{p'}]^p[x\mapsto v]&=[\mbox{ref}\;(e^{p'}[x\mapsto v])]^p\\
		[!e^{p'}]^p[x\mapsto v]&=[!(e^{p'}[x\mapsto v])]^p\\
		[e_1^{p'}\;:=\;e_2^{p''}]^p[x\mapsto v]&=[e_1^{p'}[x\mapsto v]\;e_2^{p''}[x\mapsto v]]^p\\
	\end{align*}
\end{definition}

\begin{definition}[Pattern substitutions]
	Suppose $\pi$ is patterns construct, we say that
	$$\pi[x\mapsto v]$$
	is a substitution of $v$ for all occurrences of the variable $x$ on the structure of $\pi$ defined by:
	\begin{align*}
		[(s\;e^{p'})\;\pi][x\mapsto v]&=[(s\;(e^{p'}[x\mapsto v]))\;(\pi[x\mapsto v])]\\
		[(s\;e^{p'})][x\mapsto v]&=[(s\;(e^{p'}[x\mapsto v]))]\\
	\end{align*}
\end{definition}

\todo[inline]{The type substitution lemma only works for placeholder variables and not internal variables, which is a bit of a problem. To handle internal variables, the value $v$ should be changed when meeting a \runa{Ref write} operation, and the substitution should be on \runa{Ref read}. This can be usefull to explain the \runa{Ref} and \runa{Ref write} rules as they extend upon the global $\Gamma$}
\begin{lemma}[Type substitution]
	Suppose that $e^p$ is an occurrence, that 
	$\Gamma,x^{p'}:T_1;\Pi\vdash e^p:T$ 
	and 
	$\Gamma;\Pi\vdash v:T_1$.
	Then we have that
	$\Gamma;\Pi\vdash e^p\begin{Bmatrix} ^v/_x \end{Bmatrix}$
	where $x^{p'}\not\in dom(\Gamma)$.
\end{lemma}
\begin{proof}
	The proof proceeds by induction on the height for the derivation tree of the judgement $\Gamma,\Pi\vdash e^p:T$.
	
	In the base case we have the \runa{Cons} and \runa{Var} rule:
	\begin{description}
		\item[\runa{Cons}] Here $e^p=c^p$, since $c^p\begin{Bmatrix} ^v/_x \end{Bmatrix}=c^p$ and we have $\Gamma;\Pi\vdash v:(\emptyset,\emptyset)$ this case follows immediately.

		\item[\runa{Var}] Here $e^p=x^p$, since $x^p\begin{Bmatrix} ^v/_x \end{Bmatrix}=v$ and we have $\Gamma;\Pi\vdash v:T$ this case follows immediately.
	\end{description}

	Next, follows the induction step:
	\begin{description}
		\item[\runa{Abs}] Here $e^p=[\lambda\;y\;e_1^{p'}]^p$, $x\neq y$, and $e^p\begin{Bmatrix} ^v/_x \end{Bmatrix}=[\lambda\;y\;e_1^{p'}]^p\begin{Bmatrix} ^v/_x \end{Bmatrix}$.
			By the induction hypothesis, we have $\Gamma,y:T_1;\Pi\vdash e_1^{p'}\begin{Bmatrix} ^v/_x \end{Bmatrix}$ where $x\notin dom(\Gamma)$.
			By using \runa{Abs}, we get $[\lambda\;y\;e_1^{p'}]^p\begin{Bmatrix} ^v/_x \end{Bmatrix}$

		\item[\runa{App}] Here $e^p=[e_1^{p'}\;e_2^{p''}]^p$ and $e^p\begin{Bmatrix} ^v/_x \end{Bmatrix}=[e_1^{p'}\;e_2^{p''}]^p\begin{Bmatrix} ^v/_x \end{Bmatrix}$.
			By the induction hypothesis, we have $\Gamma;\Pi\vdash e_1^{p'}\begin{Bmatrix} ^v/_x \end{Bmatrix}$ and $\Gamma;\Pi\vdash e_2^{p''}\begin{Bmatrix} ^v/_x \end{Bmatrix}$ where $x\notin dom(\Gamma)$.
			By using \runa{App}, we get $[e_1^{p'}\;e_2^{p''}]^p\begin{Bmatrix} ^v/_x \end{Bmatrix}$

		\item[\runa{Let-1}] Here $e^p=[\mbox{let}\;y\;e_1^{p'}\;e_2^{p''}]^p$, $x\neq y$, and $e^p\begin{Bmatrix} ^v/_x \end{Bmatrix}=[\mbox{let}\;y\;e_1^{p'}\;e_2^{p''}]^p\begin{Bmatrix} ^v/_x \end{Bmatrix}$.
			By the induction hypothesis, we have $\Gamma;\Pi\vdash e_1^{p'}\begin{Bmatrix} ^v/_x \end{Bmatrix}$ and $\Gamma,y:(\delta,\kappa\cup\{y\});\Pi\vdash e_2^{p''}\begin{Bmatrix} ^v/_x \end{Bmatrix}$ where $x\notin dom(\Gamma)$.
			By using \runa{Let-1}, we get $[\mbox{let}\;y\;e_1^{p'}\;e_2^{p''}]^p\begin{Bmatrix} ^v/_x \end{Bmatrix}$

		\item[\runa{Let-2}] Here $e^p=[\mbox{let}\;y\;e_1^{p'}\;e_2^{p''}]^p$, $x\neq y$, and $e^p\begin{Bmatrix} ^v/_x \end{Bmatrix}=[\mbox{let}\;y\;e_1^{p'}\;e_2^{p''}]^p\begin{Bmatrix} ^v/_x \end{Bmatrix}$.
			By the induction hypothesis, we have $\Gamma;\Pi\vdash e_1^{p'}\begin{Bmatrix} ^v/_x \end{Bmatrix}$ and $\Gamma,y:T_1;\Pi\vdash e_2^{p''}\begin{Bmatrix} ^v/_x \end{Bmatrix}$ where $x\notin dom(\Gamma)$.
			By using \runa{Let-2}, we get $[\mbox{let}\;y\;e_1^{p'}\;e_2^{p''}]^p\begin{Bmatrix} ^v/_x \end{Bmatrix}$

		\item[\runa{Let rec}] Here $e^p=[\mbox{let rec}\;f\;e_1^{p'}\;e_2^{p''}]^p$, $x\neq y$, and $e^p\begin{Bmatrix} ^v/_x \end{Bmatrix}=[\mbox{let rec}\;f\;e_1^{p'}\;e_2^{p''}]^p\begin{Bmatrix} ^v/_x \end{Bmatrix}$.
			By the induction hypothesis, we have $\Gamma;\Pi\vdash e_1^{p'}\begin{Bmatrix} ^v/_x \end{Bmatrix}$ and $\Gamma,f:T_1\rightarrow T_2;\Pi\vdash e_2^{p''}\begin{Bmatrix} ^v/_x \end{Bmatrix}$ where $x\notin dom(\Gamma)$.
			By using \runa{Let rec}, we get $[\mbox{let}\;y\;e_1^{p'}\;e_2^{p''}]^p\begin{Bmatrix} ^v/_x \end{Bmatrix}$

		\item[\runa{Case}] Here $e^p=[\mbox{case}\;e^{p'}\;\vec{\pi}]^p$, and $e^p\begin{Bmatrix} ^v/_x \end{Bmatrix}=[\mbox{case}\;e^{p'}\;\vec{\pi}]^p\begin{Bmatrix} ^v/_x \end{Bmatrix}$.
			By the induction hypothesis, we have $\Gamma;\Pi\vdash e^{p'}\begin{Bmatrix} ^v/_x \end{Bmatrix}$ and $\sigma(\Gamma,s_i,p',T');\Pi\vdash (s_i\;e_i^{p'})\begin{Bmatrix} ^v/_x \end{Bmatrix}$ for each $1\leq i \leq |\vec{\pi}|$, and where $x\notin dom(\Gamma)$.
			By using \runa{Case}, we get $[\mbox{case}\;e^{p'}\;\vec{\pi}]^p\begin{Bmatrix} ^v/_x \end{Bmatrix}$

		\todo[inline]{\runa{Match} is a bit weird, as a pattern is not an expression, should this be its own substitution lemma?}
		\item[\runa{Match}] Here $e^p=[(s\;e^{p'})]$, $s\neq x$, and $e^p\begin{Bmatrix} ^v/_x \end{Bmatrix}=[(s\;e^{p'})]\begin{Bmatrix} ^v/_x \end{Bmatrix}$.
		By the induction hypothesis, we have $\Gamma;\Pi\vdash e^{p'}\begin{Bmatrix} ^v/_x \end{Bmatrix}$, where $x\notin dom(\Gamma)$.
		By using \runa{Match}, we get $[(s\;e^{p'})]\begin{Bmatrix} ^v/_x \end{Bmatrix}$.

		\item[\runa{Ref}] Here $e^p=[\mbox{ref}\;e^{p'}]^{p}$ and $e^p\begin{Bmatrix} ^v/_x \end{Bmatrix}=[\mbox{ref}\;e^{p'}]^{p}\begin{Bmatrix} ^v/_x \end{Bmatrix}$.
		By the induction hypothesis, we have $\Gamma;\Pi\vdash e^{p'}\begin{Bmatrix} ^v/_x \end{Bmatrix}$, where $x\notin dom(\Gamma)$.
		By using \runa{Ref}, we get $[\mbox{ref}\;e^{p'}]^{p}\begin{Bmatrix} ^v/_x \end{Bmatrix}$.


		\item[\runa{Ref read}] Here $e^p=[!e^{p'}]^{p}$, and $e^p\begin{Bmatrix} ^v/_x \end{Bmatrix}=[!e^{p'}]^{p}\begin{Bmatrix} ^v/_x \end{Bmatrix}$.
		By the induction hypothesis, we have $\Gamma;\Pi\vdash e^{p'}\begin{Bmatrix} ^v/_x \end{Bmatrix}$, where $x\notin dom(\Gamma)$.
		By using \runa{Ref read}, we get $[\mbox{ref}\;e^{p'}]^{p}\begin{Bmatrix} ^v/_x \end{Bmatrix}$.

		\item[\runa{Ref write}] Here $e^p=[e_1^{p'}\;:=\;e_2^{p''}]^{p}$, and $e^p\begin{Bmatrix} ^v/_x \end{Bmatrix}=[e_1^{p'}\;:=\;e_2^{p''}]^{p}\begin{Bmatrix} ^v/_x \end{Bmatrix}$.
		By the induction hypothesis, we have $\Gamma;\Pi\vdash e_1^{p'}\begin{Bmatrix} ^v/_x \end{Bmatrix}$ and $\Gamma;\Pi\vdash e_2^{p''}\begin{Bmatrix} ^v/_x \end{Bmatrix}$, where $x\notin dom(\Gamma)$.
		By using \runa{Ref write}, we get $[e_1^{p'}\;:=\;e_2^{p''}]^{p}\begin{Bmatrix} ^v/_x \end{Bmatrix}$.
	\end{description}
\end{proof}

\begin{theorem}[Soundness of type system]
	Suppose $e^p$ is an occurrence where
	\begin{itemize}
		\item $\Gamma,\Pi\vdash e^p : T$, and 
		\item $env\vdash\left\langle e^p,w,sto,p'\right\rangle\rightarrow\left\langle v,w',sto',(L,V),p''\right\rangle$
	\end{itemize}
	and that $(w,env)\models\Gamma$.
	Then we have that:
	\begin{itemize}
		\item $\Gamma,\Pi\vdash v : T$
		\item $(w',env)\models\Gamma$
		\item $(env,(L,V))\models (\Gamma,T)$
	\end{itemize}
\end{theorem}
\begin{proof}
	The proof proceeds by induction on the height for the derivation tree for $env\vdash\left\langle e^p,w,sto,p'\right\rangle\rightarrow\left\langle v,w,sto,L,V,p\right\rangle$ and $\Gamma,\Pi\vdash e^p:T$
	In the base case we have the \runa{Cons} and \runa{Var} rule:
	\begin{description}
		\item[\runa{Cons}] Here $e^p=c^p$, where
			\begin{itemize}
				\item $\Gamma,\Pi\vdash c^p : (\emptyset,\emptyset)$, and 
				\item $env\vdash\left\langle c^p,w,sto,p'\right\rangle\rightarrow\left\langle c,w',sto',(\emptyset,\emptyset),p\right\rangle$
			\end{itemize}
			and $(w,env)\models\Gamma$.
			We then immediately get:
			\begin{itemize}
				\item $\Gamma,\Pi\vdash c : (\emptyset,\emptyset)$
				\item $(w,env)\models\Gamma$
				\item $(env,(\emptyset,\emptyset))\models (\Gamma,(\emptyset,\emptyset))$
			\end{itemize}
			Here, the conclusion is immediate, since there is no extensions to $w$ or $env$ and the type is $(\emptyset,\emptyset)$.

		\item[\runa{Var}] Here $e^p=x^p$, where
			\begin{itemize}
				\item $\Gamma,\Pi\vdash x^p : T\sqcup (\{x^p\},\emptyset)$, and 
				\item $env\vdash\left\langle x^p,w,sto,p'\right\rangle\rightarrow\left\langle v,w',sto',(\emptyset,\emptyset),p\right\rangle$
			\end{itemize}
			and $(w,env)\models\Gamma$.
			Since $(w,env)\models\Gamma$, and there is no update to $w$, we then immediately get:
			\begin{itemize}
				\item $\Gamma,\Pi\vdash v : T$
				\item $(w,env)\models\Gamma$
				\item $(env,(L,V)\models (\Gamma,T)$
			\end{itemize}
	\end{description}

	Next, follows the induction step:
	\begin{description}
		\item[\runa{Abs}] Here $e^p=[\lambda\;x\;e_1^{p'}]^p$, where
			\begin{itemize}
				\item $\Gamma,\Pi\vdash [\lambda\;x\;e^{p'}]^p : T_1\rightarrow T_2,$, and 
				\item $env\vdash\left\langle [\lambda\;x\;e^{p'}]^p,w,sto,p'\right\rangle\rightarrow\left\langle \left\langle x,e^{p'},env\right\rangle,w',sto',(\emptyset,\emptyset),p\right\rangle$
			\end{itemize}
			and $(w,env)\models\Gamma$.
			Since $(w,env)\models\Gamma$, and there is no update to $w$, we then immediately get:
			\begin{itemize}
				\item $(w,env)\models\Gamma$
				\item $(env,(\emptyset,\emptyset)\models (\Gamma,T)$
			\end{itemize}
			Next is for the value, where in the type rule for \runa{Abs}, we have:
			\begin{figure}[H]
				\setlength\tabcolsep{8pt}
				\begin{tabular}{l}
					\InfName{Abs}\\[0.2cm]
						\inference[]
						{\Gamma,x^{p'}:T_1;\Pi\vdash  e^{p}:T_2}
						{\Gamma;\Pi\vdash  [\lambda\;x.e^{p}]^{p'}:T_1\rightarrow T_2}\\
				\end{tabular}
			\end{figure}
			And we know by the semantics, that \runa{Abs} evaluates to a closure, as such we must get:
			\begin{figure}[H]
				\setlength\tabcolsep{8pt}
				\begin{tabular}{l}
					\runa{Closure}\\[0.4cm]
						\inference[]
							{
								\Gamma;\Pi\vdash env \\
								\Gamma,x^{p}:T_1;\Pi\vdash e^{p'}:T_2
							}
							{\Gamma;\Pi\vdash \left\langle x^{p}, e^{p'}, env \right\rangle^{p''}:T_1\rightarrow T_2}\\
				\end{tabular}
			\end{figure}
		\item[\runa{App}] Here $e^p=[e_1^{p'}\;e_2^{p''}]^p$ 

		\item[\runa{Let}] Here $e^p=[\mbox{let}\;x\;e_1^{p'}\;e_2^{p''}]^p$, where
			\begin{itemize}
				\item $\Gamma,\Pi\vdash [\mbox{let}\;x\;e_1^{p'}\;e_2^{p''}]^p : T,$, and 
				\item $env\vdash\left\langle [\lambda\;x\;e^{p'}]^p,w,sto,p'\right\rangle\rightarrow\left\langle \left\langle x,e^{p'},env\right\rangle,w',sto',(L,V),p\right\rangle$
			\end{itemize}
			and $(w,env)\models\Gamma$.
			First, we need to analyse the premises, where in the semantics we have:
			\begin{figure}[H]
				\setlength\tabcolsep{8pt}
				\begin{tabular}{l}
					\InfName{Let}\\[0.2cm]
						\inference[]
						{env\vdash \left\langle e_1^{p'},sto,w,p \right\rangle \rightarrow \left\langle v_1,sto',w'',(L',V'),p' \right\rangle &\\
						env[x\mapsto v_1]\vdash \left\langle e_2^{p''},sto',w_3,p' \right\rangle \rightarrow \left\langle v,sto'',w',(L,V),p'' \right\rangle}
						{env\vdash \left\langle [\mbox{let}\;x\;e_1^{p'}\;e_2^{p''}]^{p_3},sto,w,p \right\rangle \rightarrow \left\langle v,sto'',w_3,(L,V),p_3 \right\rangle}\\
						Where $w_3=w''[x^{p'}\mapsto(L',V')]$\\
				\end{tabular}
			\end{figure}

			In the type system, we have to let rules, \runa{Let-1} and \runa{Let-2}, which differs in the typing of the first premise, namely when there is aliasing and not.
			Thus, these two type rules are for when the value $v_1$ is $v_1=\loc$ or $v_1\neq\loc$, which corresponds to when the type is $(\delta,\kappa)$ and $\kappa\neq\emptyset$.

			When $v_1=\loc$, where have the following type rule
			\begin{figure}[H]
				\setlength\tabcolsep{8pt}
				\begin{tabular}{l}
					\runa{Let-1}\\[0.2cm]
						\inference[]
							{\Gamma;\Pi\vdash e_1^{p}:(\delta,\kappa) &\\
							\Gamma,x^{p'}:(\delta,\kappa\cup\{x\});\Pi\vdash e_2^{p}:T}
							{\Gamma;\Pi\vdash [\mbox{let}\; x \; e_1^{p} \; e_2^{p'}]^{p''}:T}\\[0.3cm]
						Where $\kappa\neq\emptyset$\\
				\end{tabular}
			\end{figure}
			Then, we can immediately get $(w'',env)\models\Gamma$, and thus we also get $(env,(L,V))\models(\Gamma,(\delta,\kappa))$
			We also know that the type for the value $v_1$ must be $\Gamma,\Pi\vdash v_1:(\delta,\kappa)$.
			
			In the second premise, we have $(w''[x^{p'}\mapsto(L',V')],env(x\mapsto v_1))\models(\Gamma,x^{p'}:(\delta,\kappa\cup\{x\}))$
$(w''[x^{p'}\mapsto(L',V')],env)\models(\Gamma,x^{p'}:(\delta,\kappa\cup\{x\}))$.
			We also know that the type of $v_1$, thus the extension in $w''$, $env$, and $\Gamma$ can be removed by substitution for $v_1$ in $e_2^{p''}$.
			Thus we can get $(env,(L,V))\models(\Gamma,(\delta,\kappa\cup\{x\}))$ and $\Gamma;\Pi\vdash v:T$, and the conclusion will then follow.


			When $v_1\neq\loc$, where have the following type rule
			\begin{figure}[H]
				\setlength\tabcolsep{8pt}
				\begin{tabular}{l}
					\runa{Let-2}\\[0.2cm]
						\inference[]
							{\Gamma;\Pi\vdash e_1^{p}:T_1 &\\
							\Gamma,x^p:T_1;\Pi\vdash e_2^{p}:T}
							{\Gamma;\Pi\vdash [\mbox{let}\; x \; e_1^{p} \; e_2^{p'}]^{p''}:T}\\
				\end{tabular}
			\end{figure}
			The proof follows similarly for the case where $v_1=\loc$.


		\item[\runa{Let rec}] Here $e^p=[\mbox{let rec}\;f\;e_1^{p'}\;e_2^{p''}]^p$
	\end{description}
\end{proof}
\end{document}
