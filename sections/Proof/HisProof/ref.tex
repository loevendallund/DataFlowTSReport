\item[\runa{Ref}] Here $e^{p'}=[\mbox{let}\;x\;e_1^{p_1}\;e_2^{p_2}]^{p'}$, where
\begin{figure}[H]
	\setlength\tabcolsep{8pt}
	\begin{tabular}{l}
		\InfName{Ref}\\[0.2cm]
		\inference[]
				{env \vdash \left\langle e_1^{p_1},sto,(w,\sqsubseteq_w),p \right\rangle \rightarrow \left\langle v,sto_1,(w_1,\sqsubseteq_w'),(L,V),p' \right\rangle}
				{env\vdash \left\langle \left[\mbox{ref}\;e_1^{p_1}\right]^{p'},sto,w,p \right\rangle \rightarrow \left\langle \loc,sto',(w',\sqsubseteq_w'),(\emptyset,\emptyset),p' \right\rangle}\\
	\end{tabular}
\end{figure}
Where $\loc=next$, $sto'=sto_1[next\mapsto new(\loc),\loc\mapsto v]$, and $w'=w_1[\loc^{p'}\mapsto (L,V)]$.
By virtue of our induction hypothesis, we can get the following from the premises:
\begin{description}
	\item[1)] if $y^{p''}\in dom(w')\backslash dom(w)$ then $y\notin fv(e_1^{p_1})$
\end{description}
By \cref{def:fv} we can then conclude that: if $y^{p''}\in dom(w')\backslash dom(w)$ then $y\notin fv(e^{p'})$
