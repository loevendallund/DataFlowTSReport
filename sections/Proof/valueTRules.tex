\documentclass[../../master.tex]{subfiles}
\begin{document}
\subsection{Type rules for values}
For the sake of proving the type system, we present type rules for values.
Where the type rules is given in \cref{fig:ValTypeRules}.

Since the type rules for closures and recursive closures also contain the environment, we also need to handle the environment.
As such we define the judgement for environments, where all bindings in the environment also have a binding in the type environment.

\begin{definition}[Environment judgement]\label{def:TEnv}
	Let $v_1,\cdots,v_n$ be values such that $\Gamma';\Pi\vdash v_i:T_i$, for $1\leq i\leq n$.
	Let $env$ be an environment where $env=[x_1\mapsto v_1,\cdots,x_n\mapsto v_n]$, $\Gamma$ be a type environment, and $\Pi$ be the approximated order of program points.
	We say that:
	$$\Gamma;\Pi\vdash env$$
	iff 
	\begin{itemize}
		\item For all $x_i\in dom(env)$ then $\exists x_i^p\in dom(\Gamma)$ where $\Gamma(x_i^p)=T_i$ then 
			$$\Gamma,\Pi\vdash env(x_i):T_i$$
	\end{itemize}
\end{definition}

\begin{description}
	\item[\runa{Constant}] type rule differs from the rule \runa{T-Const}, since most occurrences can evaluate to a constant and as such we know that its type should be a base type.
		Since constants can depend on other occurrences, we know that $\delta$ can be non-empty, but since constants can never be locations, we also know that it can never contain alias information, and as such $\kappa$ should be empty.

	\item[\runa{Location}] type rule represents locations, where we know that it must be a base type.
		Since locations can depend on other occurrences, we know that $\delta$ can be non-empty.
		Since locations can have alias information, and that a location is considered to always be an alias to itself, we know that $\kappa$ can never be empty, as it should always contain an internal variable. 

	\item[\runa{Closure}] type rule represents abstraction, and as such we know that it should have the abstraction type, $T_1\rightarrow T_2$.
		Since a closure contains an abstraction parameter, body, and the environment from when it were declared, we also need to handle those part in the type rule.
		The components of the closure is handled in the premises, where we type the environment according to \cref{def:TEnv}.
		We also type the body of the abstraction, we know that we need to update the type environment with the type $T_1$ for its parameter, and based on this we can then type the body to $T_2$.

	\item[\runa{Recursive closure}] type rules is similar to the \runa{Closure} rule, but since this is a recursive closure, we additionally need to update the type environment with the recursive abstractions name to the type of the recursive function.

	\item[\runa{Unit}] type rule simply have the base type, as it is not an abstraction and it also cannot have alias information, but it can depend on other occurrences.
\end{description}

\begin{figure}[H]
	\setlength\tabcolsep{8pt}
	\begin{tabular}{l}
		\runa{Constant}\\[0.2cm]
			\inference[]{}
				{\Gamma,\Pi\vdash  c:(\delta, \emptyset)}\\[1cm]

		\runa{Location}\\[0.2cm]
			\inference[]{}
				{\Gamma,\Pi\vdash  \loc:(\delta, \kappa)}\\
				Where $\kappa\neq\emptyset$\\[1cm]

		\runa{Closure}\\[0.2cm]
			\inference[]
				{
					\Gamma,\Pi\vdash env \\
					\Gamma[x^{p}:T_1],\Pi\vdash e^{p'}:T_2
				}
				{\Gamma,\Pi\vdash \left\langle x^{p}, e^{p'}, env \right\rangle^{p''}:T_1\rightarrow T_2}\\[1cm]

		\runa{Recursive closure}\\[0.2cm]
			\inference[]
				{
					\Gamma,\Pi\vdash env \\
					\Gamma[x^{p}:T_1,f^{p'}:T_1\rightarrow T_2],\Pi\vdash e^{p''}:T_2
				}
				{\Gamma,\Pi\vdash \left\langle x^{p}, f^{p'}, e^{p''}, env \right\rangle^{p_3}:T_1\rightarrow T_2}\\[1cm]

		\runa{Unit}\\[0.2cm]
			\inference[]{}
				{\Gamma,\Pi\vdash  ():(\delta,\emptyset)}\\[0.5cm]
	\end{tabular}
	\caption{Type rules for values}
	\label{fig:ValTypeRules}
\end{figure}
\end{document}
