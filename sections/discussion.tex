\documentclass[../../master.tex]{subfiles}
\begin{document}
\section{Discussion and conclussion}\label{sec:Conc}
In this paper, we have presented a type system for data-flow and alias analysis, that collects the variables and internal variables used to evaluate an occurrence.
\section{Future work}\label{sec:FW}
In this section we will present some areas which could be worked on.

\paragraph{Polymorphism}
One area to look at, is to introduce polymorphism to the type system which can reduce the slack of the type system.
One way is to introduce polymorphism to the base types, which would allow the typing of abstraction used at multiple places.

\paragraph{Extending locations}
Right now, locations are restricted upon, since it does not allow bindings to abstractions.
This would be a good place to extend, where there is two directions this could be handled.
The extension would follow Polymorphism, as either it should only allows for bindings to similar functions, or allows for free bindings.

\paragraph{Type inference}
Another area is to make a type inference algorithm, which can find the type information.
To make type inference for the type system would need to find an approximated order of program points, find a proper $\kappa_0$ and typing for abstractions, that is, find all the places where the parameter should be bound.

\paragraph{Extending with more language constructs}
As this is a small $\lambda$ calculus with a couple of interesting constructs, such as mutability, pattern matching, and local bindings.
An area to look at is to introduce more constructs to the language, such as lazy evaluation, constructors, and modules.
The lazy evaluation could either be through a wrapping construct or as a core part of the language.
Constructors could introduce user specified constructs, which also would extend upon the pattern matching.
Modules would wrap part of a program which could then be reused multiple places.
\end{document}
