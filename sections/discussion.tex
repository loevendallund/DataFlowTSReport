\documentclass[../../master.tex]{subfiles}
\begin{document}
\section{Conclusion}\label{sec:Conc}
In this paper, we have introduced a type system for local data-flow analysis for a language based on a subset of ReScript.
The type system we present here differs from other data-flow analysis techniques, that instead of solving constraints, gives a semantic analysis of a program.

In the type system, we have shown how to handle data-flow analysis for different language constructs, for pattern matching, local declarations, and referencing.
As pattern matching introduces branches to the language, we showed a sound over-approximation of how to handle these branches.
Additionally, since we also have mutability, through referencing, the approximation should also, in case of reading from a reference, get all places where a reference binding could exist in the type environment.
Since some branches could write to a reference, while others do not, it was important to consider each branch separately when reading from references.

\subsection{discussion}
The type system we have presented is for a small language without many constructs.
However, some interesting constructs were introduced to the language, such as mutability and pattern matching which introduces some challenges when trying to make a good approximation of data-flow.

The challenges from having both pattern matching and mutability introduced challenges, as each branch could not simply be thought of as locally, since reference operations introduce side effects.
To references, we represent them as internal variables, i.e., variables that do not exist in the syntax, and treating them as global information.
To handle the problem of branching, when reading from a reference, we look at each branch independently to find the information necessary.
\bigskip

Since we focused on a functional language, that is based on expressions, we focused the data-flow analysis on the flow of variables, i.e., which variables are used to evaluate an occurrence.
As the language is primarily a series of declaration of functions, variables, and references, this allows for analyzing where variables are used on which are useful evaluating an occurrence.

Additionally by representing referencing as internal variables, it allows for understanding of which references are used and where they are used in the occurrence.
This information can be used by compilers to make sure that references can be safely cleaned up, after the last place they were used.
The alias information also implies which aliases were used in the occurrence.
\bigskip

However, the type system introduced contains some restrictions, also called slack, for which occurrences it accepts.
As the type system does not allow for type polymorphism, the use cases for abstractions are restricted.
In one place, where abstractions are quite limited is when binding them to a local declaration, this local declaration cannot be used at multiple places, as this would mean it would contain occurrences at multiple program points.

Another area of the type system contains slack, is for references as abstractions cannot be bound to them.
Here, another issue occurs as the environment only contains local information, and an abstraction thus only knows about the variables known when it was declared.
As the type system is currently defined, the type environment should be bound with the abstraction, but the current type system does not allow this, as the type environment both includes local and global information.


\subsection{Future work}\label{sec:FW}
We will now introduce potential future work, as areas of improvement for the type system

\paragraph{Implementation of a type checker}
Implementing a type checker for the type system presented here would allow for testing how well the information is used.
It would also allow for comparing how well it performs, compared to other data-flow analysis techniques.

\paragraph{Polymorphism}
Introducing polymorphism would be an ideal place to extend upon the type system, as abstractions are restricted in the current type system.
Here, polymorphism for the base type, that is for $(\delta,\kappa)$, would allow for abstractions to be used multiple times in an occurrence, the input and output type would not be restricted from only allowing the exact same input type.
As such, consider the following occurrence:
\begin{lstlisting}[language=Caml, mathescape=true]
(let x ($\lambda$ y.y$^1$)$^2$ (x$^3$ (x$^4$ 1$^5$)$^6$)$^7$)$^8$
\end{lstlisting}
If polymorphism is introduced, occurrences like this could be defined, since when typing the applications, the type of the argument changes, since the occurrence $x^4$ is present in the second application.


\paragraph{Extending references}
References are defined currently in the type system, they cannot be bound to abstractions.
However, this would also introduce complications, as abstractions need the information known at the time they were declared.
Another complication would be that if different references had different types, e.g., if it had an abstraction type at one point and a base type at another point.
Here, either we should require references to always have the same type, e.g., with base type polymorphism.
Consider the following occurrence:
\begin{lstlisting}[language=Caml, mathescape=true]
((!(case 1 (1$^1$) (let z 5$^2$ (ref ($\lambda$ y.(PLUS z$^3$ y$^4$)$^5$)$^6$))$^7$)$^8$)$^9$ 5)$^{10}$
\end{lstlisting}
This occurrence would create a reference to a local abstraction which depends on the locally declared variable $z$ before reading from the reference and applying the constant to it.
In the semantics, the environment would be added to the abstraction closure, and when evaluating the body of the abstraction, in an application, it would use the environment in the closure.

\paragraph{Type inference}
Another area is to make a type inference algorithm, which can find the type information.
To make type inference for the type system would need to find an approximated order of program points, find a proper $\kappa_0$ and type for abstractions, that is, find all the places where the parameter should be bound.

\paragraph{Extending with more language constructs}
It would also be interesting to introduce more language constructs, as the language presented only contains a small amount of constructs, such as mutability and pattern matching.
Some interesting constructs to add could be records, constructors and deconstructors, modules, or lazy evaluation.
Here, lazy evaluation could take multiple forms, either by introducing it as a core part of the language, where every binding is lazy evaluated, or add special constructs for lazy evaluation.
Modules, on the other hand, would allow for wrapping an occurrence, or multiple occurrences into a module, which could then be used in multiple places.

\end{document}
