\documentclass[../../master.tex]{subfiles}
\begin{document}
\section{Occurrence dependency system}
This section will introduce the occurrence dependency system for the language presented in \cref{sec:lang}, to describe what occurrences are used to evaluate a term.
The system is based on the notion of type systems, and is therefore used to classifying the syntactic occurrences of used.
For this reason, some notations from the semantics needs to be represented differently, in particular locations as they are purely semantical.
We therefore introduce a new syntactic category of variables, $\AVar$, which we call internal variables and is ranged over by $\nu x,\;\nu y$.
We further denote the set of all variables as $Id=\Var\cup\IVar\cup\AVar$.

\subsection{Types}
Similar to type systems, we denote the syntactical category for types as \cat{Types} and is ranged over by $T$
A subset of of \cat{Types} is the set of base types, ranged over by the pair $(\delta,\kappa)$, where $\delta\subseteq Id^P$ is a set of variable occurrences and $\kappa\subseteq\Pow{\Pow{Id}}$.
Note that $\kappa$ will always contain at least one internal variable.

The composite type for abstractions, denoted as $T_1\rightarrow T_2$, represents a function that takes an input type $T_1$ and returns a value of type $T_2$.
The type of abstractions is denoted as $T_1\rightarrow T_2$, that 

The formation rules for types is thus given as:
$$T::=(\delta,\kappa)\mid T_1 \rightarrow T_2$$

\subsection{Bases}
In the semantics, presented in \cref{sec:sem}, we keep track of dependencies through the dependency function $w$, and value bindings through the environment and store.

The occurrence dependency system similarly needs to keep track of dependencies.
It is also, equally important, that the knowledge of which variables are bound to which internal variables.
We therefore introduce the notion of a type dependency base $\Gamma$, that keeps track of the type of variables, and a reference family $\Upsilon$ that keeps track of which variables shares internal variables.
%As the language contains the notion of pattern matching, the type can be non-deterministic as there can be multiple branches, therefore variables can be bound to multiple internal variables.

The type dependency base can thus be defined as:
\begin{definition}{[Type base $\Gamma$]}
	The type base $\Gamma$ is a partial function $\Gamma:Id^P\rightharpoonup TYPES$
\end{definition}

A lookup in the type base $\Gamma$ is of the form $\Gamma(x^p)=T$ for placeholder and imperative variables, and $\Gamma(\nu x^p)=T$ for internal variables.

\begin{definition}{[Extending $\Gamma$]}
	We write $\Gamma,x^p:T$ for the type base $\Gamma'$ defined by:
	\begin{align*}
		\Gamma'(y^{p'})=
		\left\{\begin{matrix}
			\Gamma(x^p) & \mbox{if}\;y^{p'} \neq x^{p}\\ 
			T & \mbox{if}\;y^{p'} = x^{p}
		\end{matrix}\right.
	\end{align*}
\end{definition}
The extension of $\Gamma$ for internal variables can be similarly defined.
\bigskip

\begin{definition}{[Reference family $\Upsilon$]}
	The  $\Upsilon$ is a family of sets, $\Upsilon=\Pow((\IVar\cup\AVar))$, where each partition is disjunct.
\end{definition}
We denote $\Upsilon(x)=\kappa$ as a partition $B\in\Upsilon$ where $x\in B$.
%We denote $\Upsilon(x)=\kappa$ as a lookup of all sets in $\Upsilon$ containing the variable $x$.

\begin{definition}{[Extending $\Upsilon$]}
	We write $\Upsilon,\kappa$ for $\Upsilon'$ defined by:
	\begin{align*}
		\Upsilon'(y)=
		\left\{\begin{matrix}
 			\Upsilon(y) & if y\not\in\kappa \\ 
 			\kappa & if y\in\kappa 
		\end{matrix}\right.
	\end{align*}
\end{definition}
\bigskip

The last base, $\Pi$, is the approximated order of program points, where we denote $p\sqsubseteq p'\in\Pi$ if the program point $p$ is an earlier or the same program point as $p'$.
The order of program points is also transitive, i.e., if $p\sqsubseteq p'\in\Pi$ and $p'\sqsubseteq p''\in\Pi$ then $p\sqsubseteq p''\in\Pi$.


\subsection{Judgement}
The type judgement is defined as:
$$\Gamma,\Upsilon,\Pi\vdash e^p: T$$
And should be read as, the occurrence $e^p$ has type $T$ given the dependency bindings $\Gamma$, the family of variables $\Upsilon$, and the approximated order of program points $\Pi$.

In the $VAR$ rule we make use of the union between a type and a base type, which we define as follows:

\begin{definition}{[Union of types]}
	Let $T$ be an arbitrary type and $(\delta,\kappa)$ be a base type, then the unification of these two types is as follows:
	\begin{equation*}
		T_1\cup T_2=
		\left\{\begin{matrix}
			\mbox{If } \; T_1=(\delta,\kappa) \;\mbox{and}\; T_2=(\delta',\kappa')  & \mbox{then} \; (\delta\cup\delta',\kappa\cup\kappa')\\
			\mbox{else if } \; T_1=T_1'\rightarrow T_1''\;\mbox{and}\;T_2=T_2'\rightarrow T_2'' & \mbox{then} \; T_1'\cup T_2'\rightarrow (T_1''\cup T_2'')
		\end{matrix}\right.
	\end{equation*}
\end{definition}

\begin{landscape}
\subsection{Example Abs}
Given the following program
$$(\lambda x.(x\;z))(\lambda y.y)$$
Where $z$ is bound to a constant integer and its type is thus $(\emptyset,\emptyset)$.

To type this program, we first construct the tree according to the rules:

\begin{figure}[H]
	\setlength\tabcolsep{8pt}
		\begin{tabular}{l}
		\inference[$APP$]
		{
			\inference[$ABS$]
			{
				\inference[$APP$]
				{
					\inference[$VAR$]{}{\Gamma';\Upsilon;\Pi\vdash x^{p}:\alpha:T'\rightarrow T''}
					\;\;\;
					\inference[$VAR$]{}{\Gamma';\Upsilon;\Pi\vdash z^{p'}:(\{z^{p'}\},\emptyset)}
				}
				{\Gamma,x:T_1;\Upsilon;\Pi\vdash [x^{p}\;z^{p'}]^{p''}:T''\{\alpha / (\{z^{p'}\},\emptyset)\}}
			}
			{\Gamma;\Upsilon;\Pi\vdash [\lambda\; x.([x^{p}\;z^{p'}]^{p''})]^{p_3}:x:T_1\rightarrow T''\{\alpha / (\{z^{p'}\},\emptyset)\}}
			\;\;
			\inference[$ABS$]
			{
					\inference[$VAR$]{}{\Gamma,y:T;\Upsilon;\Pi\vdash y^{p_3}:T}
			}
			{\Gamma';\Upsilon;\Pi\vdash \lambda y.[y^{p_3}]^{p_4})]^{p_5}:y:T\rightarrow T}
		}
		{\Gamma;\Upsilon;\Pi\vdash [[\lambda x.[x^{p}\;z^{p'}]^{p''}]^{p_3}\;[\lambda y.y^{p_4}]^{p_5}]^{p_6}:T''\{\alpha / (\{z^{p'}\},\emptyset)\}\{x / T\rightarrow T\}}\\
		Where $T_1=\alpha:T'\rightarrow T''$
	\end{tabular}

	\label{fig:AbsExample}
\end{figure}

After that, we can solve the intermediate types, since we now know the righthand type of the outermost application, and we will get the following:

\begin{figure}[H]
	\setlength\tabcolsep{8pt}
	\begin{tabular}{l}
		\inference[$APP$]
		{
			\inference[$ABS$]
			{
				\inference[$APP$]
				{
					\inference[$VAR$]{}{\Gamma';\Upsilon;\Pi\vdash x^{p}:\alpha:T\rightarrow T}
					\;\;\;
					\inference[$VAR$]{}{\Gamma';\Upsilon;\Pi\vdash z^{p'}:(\{z^{p'}\},\emptyset)}
				}
				{\Gamma,x:T\rightarrow T;\Upsilon;\Pi\vdash [x^{p}\;z^{p'}]^{p''}:T\{\alpha / (\{z^{p'}\},\emptyset)\}}
			}
			{\Gamma;\Upsilon;\Pi\vdash [\lambda\; x.([x^{p}\;z^{p'}]^{p''})]^{p_3}:x:(T\rightarrow T)\rightarrow T\{\alpha / (\{z^{p'}\},\emptyset)\}}
			\;\;
			\inference[$ABS$]
			{
					\inference[$VAR$]{}{\Gamma,y:T;\Upsilon;\Pi\vdash y^{p_3}:T}
			}
			{\Gamma';\Upsilon;\Pi\vdash \lambda y.[y^{p_3}]^{p_4})]^{p_5}:y:T\rightarrow T}
		}
		{\Gamma;\Upsilon;\Pi\vdash [[\lambda x.[x^{p}\;z^{p'}]^{p''}]^{p_3}\;[\lambda y.y^{p_4}]^{p_5}]^{p_6}:T\{\alpha / (\{z^{p'}\},\emptyset)\}\{x / T\rightarrow T\}}\\
	\end{tabular}
	\label{fig:Simple}
\end{figure}

Lastly, we can solve the last part, to get the prober type:

\begin{figure}[H]
	\setlength\tabcolsep{8pt}
	\begin{tabular}{l}
		\inference[$APP$]
		{
			\inference[$ABS$]
			{
				\inference[$APP$]
				{
					\inference[$VAR$]{}{\Gamma';\Upsilon;\Pi\vdash x^{p}:\alpha:T\rightarrow T}
					\;\;\;
					\inference[$VAR$]{}{\Gamma';\Upsilon;\Pi\vdash z^{p'}:(\{z^{p'}\},\emptyset)}
				}
				{\Gamma,x:T\rightarrow T;\Upsilon;\Pi\vdash [x^{p}\;z^{p'}]^{p''}:T\{\alpha / (\{z^{p'}\},\emptyset)\}}
			}
			{\Gamma;\Upsilon;\Pi\vdash [\lambda\; x.([x^{p}\;z^{p'}]^{p''})]^{p_3}:x:(T\rightarrow T)\rightarrow T\{\alpha / (\{z^{p'}\},\emptyset)\}}
			\;\;
			\inference[$ABS$]
			{
					\inference[$VAR$]{}{\Gamma,y:T;\Upsilon;\Pi\vdash y^{p_3}:T}
			}
			{\Gamma';\Upsilon;\Pi\vdash \lambda y.[y^{p_3}]^{p_4})]^{p_5}:y:T\rightarrow T}
		}
		{\Gamma;\Upsilon;\Pi\vdash [[\lambda x.[x^{p}\;z^{p'}]^{p''}]^{p_3}\;[\lambda y.y^{p_4}]^{p_5}]^{p_6}:(\{z^{p'}\},\emptyset)}\\
	\end{tabular}
	\label{fig:Simple}
\end{figure}
\subsection{Example case}
In the following example, the initial $\Gamma$ and $\Upsilon$ is given by:
\begin{align*}
	\Gamma &= [\nu x\mapsto(\emptyset,\emptyset),x\mapsto(\emptyset,\{\{\nu x\}\}),\nu y\mapsto(\emptyset,\emptyset),y\mapsto(\emptyset,\{\{\nu y\}\})]\\
	\Upsilon &=\{\{\nu x, x\},\{\nu y, y\}\}
\end{align*}
\begin{figure}[H]
	\setlength\tabcolsep{8pt}
	\begin{tabular}{l}
	\inference[$CASE$]
	{
		\inference[$Cons$]{}{\Gamma;\Pi\vdash 1:(\emptyset,\emptyset) }
		\;\;
		\inference[$Pattern$]
		{
			\inference[$Var$]{} {\Gamma;\Pi\vdash x:(\{x\},\{\nu x\})}
		}
		{\Gamma;\Pi\vdash [1\;(x)]:(\{x\},\{\nu x\}) }
		\;\;
		$Case_1$
		\;\;
		$Case_2$
	}
{\Gamma;\Pi\vdash [case\;(1)\;\{(1\;(x))\;(2\;(y))\;(3\;(ref\;(1)))\}]:(\{x,y\},\{\nu x, \nu y, \nu z\})}\\
\end{tabular}

	\label{fig:SimpleCase}
\end{figure}
This example will be used in the two following, and is refered as $caseEx$ with the type:
$$T=(\{x,y\},\{\{\nu x\}, \{\nu y\}, \{\nu z\}\})$$
\newpage

\subsection{Example let-case 1}
\begin{figure}[H]
	\setlength\tabcolsep{8pt}
	\begin{tabular}{l}
	\inference[$let$]
	{
		\inference[$Case$]{}
		{\Gamma;\Upsilon;\Pi\vdash [caseEx]:T}
		\;\;
		\inference[$Ref\;read$]
		{
			\inference[$Var$]{}{\Gamma;\Upsilon;\Pi\vdash z:T\cup(\{z\},\emptyset)}
		}
		{\Gamma,z:T;\Upsilon;\Pi\vdash [!z]:T'}
	}
{\Gamma;\Upsilon;\Pi\vdash [\mbox{let}\;z\;(caseEx)\;(!z)]:T'}\\
Where $T'=(\{x,y,z,\nu x,\nu y,\nu z\},\emptyset)$
\end{tabular}

	\label{fig:LetCase1}
\end{figure}

\subsection{Example let-case 2}
\begin{figure}[H]
	\setlength\tabcolsep{8pt}
	\begin{tabular}{l}
	\inference[$let$]
	{
		\inference[$Case$]{}
		{\Gamma;\Upsilon;\Pi\vdash [caseEx]:T}
		\;\;
		\inference[$Ref\;read$]
		{
			\inference[$Var$]{}{\Gamma;\Upsilon;\Pi\vdash x:(\{x\},\{\nu x, x\})}
		}
		{\Gamma,z:T;\Upsilon;\Pi\vdash [!x]:(\{x,\nu x\},\emptyset)}
	}
{\Gamma;\Upsilon;\Pi\vdash [\mbox{let}\;z\;(caseEx)\;(!x)]:T'}\\
Where $T'=(\{x,\nu x\},\emptyset)$
\end{tabular}

	\label{fig:LetCase1}
\end{figure}
\end{landscape}

\end{document}
